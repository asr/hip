% Created 2012-02-07 Tue 13:15
\documentclass[serif,professionalfont]{beamer}
\usepackage[utf8]{inputenc}
\usepackage[T1]{fontenc}
\usepackage{fixltx2e}
\usepackage{graphicx}
\usepackage{longtable}
\usepackage{float}
\usepackage{wrapfig}
\usepackage{soul}
\usepackage{textcomp}
\usepackage{marvosym}
\usepackage{wasysym}
\usepackage{latexsym}
\usepackage{amssymb}
\usepackage{hyperref}
\usepackage{mathpartir}
\usepackage{color}
\tolerance=1000
\usepackage{palatino,inconsolata,amsmath,array}
\providecommand{\alert}[1]{\textbf{#1}}

\usepackage{listings}

\lstnewenvironment{code}[1][]%
  {
   \noindent
   \minipage{\linewidth}
   \vspace{0.2\baselineskip}
%   \vspace{-0.4\baselineskip}
   \lstset{basicstyle=\ttfamily,
%           frame=single,
           language=Haskell,
           keywordstyle=\color{black},
           #1}}
  {%\vspace{-0.8\baselineskip}
   \endminipage}

\title{Proving Equational Haskell Properties Using Automated Theorem Provers}
\author{Dan Rosén}
\date{\today}

\begin{document}

\maketitle


\makeatletter
\newcommand*{\rom}[1]{\text{\footnotesize\expandafter\@slowromancap\romannumeral #1@.}}
\newcommand*{\romnodot}[1]{\text{\footnotesize\expandafter\@slowromancap\romannumeral #1@}}
\makeatother

%\newcommand\note[1]{\mbox{}\marginpar{\footnotesize\raggedright\hspace{0pt}\emph{#1}}}
%\newcommand\note[1]{}
\newcommand\PA{\mathcal{P\!A}}
\newcommand\hs[1]{\verb~#1~}
\newcommand\ts[1]{\verb~#1~}
\newcommand\fn[1]{\mathrm{#1}}
%\newcommand\fn[1]{#1}
\newcommand\ptr[1]{\fn{\operatorname{#1-ptr}}}
\newcommand\appfn{@}
\newcommand\app[2]{#1 \, \appfn \, #2}
\newcommand\appp[2]{(#1) \, \appfn \, #2}
\newcommand\ex[1]{\exists \, #1 \, . \,}
\newcommand\nexxx[3]{\nexists \, #1 , #2 , #3 . \,}
\newcommand\fa[1]{} % \forall \, #1 . \,}
\newcommand\faa[2]{} % \forall \, #1 , #2 . \,}
\newcommand\faaa[3]{} % \forall \, #1 , #2 , #3 . \,}
\newcommand\faaaaaa[6]{} % \forall \, #1 , #2 , #3 , #4 , #5 , #6 . \,}

\newcommand{\HRule}{\rule{\linewidth}{0.5mm}}%\usetikzlibrary {\trees,positioning,arrows}

\newcommand\fixb[0]{ ^{\bullet}}
\newcommand\fixw[0]{ ^{\circ}}

\newcommand\tofix[1]{#1^{\bullet}}
\newcommand\unfix[1]{#1^{\circ}}

\newcommand\append[0]{\texttt{\small{++}}}

\newcommand{\xsys}[2]{#1 \, xs \, #2 & = #1 \, ys #2}
\newcommand{\desca}[1]{  & \hspace{44.5mm}                            \{ \text{#1} \}}
\newcommand{\descra}[1]{ & \hspace{35mm} \Rightarrow     \hspace{4mm} \{ \text{#1} \}}
\newcommand{\descla}[1]{ & \hspace{35mm} \Leftarrow      \hspace{4mm} \{ \text{#1} \}}
\newcommand{\desclra}[1]{& \hspace{35mm} \Leftrightarrow \hspace{4mm} \{ \text{#1} \}}

\newcommand\lub[1]{\sqcup_{#1}}

\newcommand\defof[1]{definition of #1}

\newcommand\w[0]{\,\,}
\newcommand\eq[0]{ = }

\newcommand{\defBNF}[4] {\text{#1}\quad&#2&::=&\;#3&\text{#4}}
\newcommand{\defaltBNF}[2] {&&|&\;#1&\text{#2}}

\newcommand{\hstup}[2]{\hs{(} #1 \hs{,} #2 \hs{)}}

\newcommand{\nsqsubseteq}{\,\,\, /\!\!\!\!\!\!\sqsubseteq}

\newcommand{\bindname}{>\!\!>\!\!=}
\newcommand{\bind}[2]{#1 \bindname #2}
\newcommand{\bindp}[3]{\fn{bind'}(#1,#2,#3)}
\newcommand{\fork}[2]{\fn{fork}(#1,#2)}
\newcommand{\forkr}[1]{\fn{right}(#1)}
\newcommand{\forkl}[1]{\fn{left}(#1)}
\newcommand{\leaf}[1]{\fn{leaf}(#1)}
\newcommand{\unleaf}[1]{\fn{unleaf}(#1)}

\newcommand{\bindb}[2]{#1 \tofix{\bindname} #2}
\newcommand{\bindw}[2]{#1 \unfix{\bindname} #2}


\newcommand\Inf{\fn{Inf}}
\newcommand\Total{\fn{Total}}
\newcommand\Fin{\fn{Fin}}

\begin{frame}
  \frametitle{Behind the Scenes}
  \begin{itemize}
    \item Translation to first order logic\\
    \item Trying different induction techniques\\
    \item Running automated theorem provers\\
  \end{itemize}
\end{frame}


%% ADD OVERVIEW


\begin{frame}[fragile]
\frametitle{Translation of bind}
\label{sec-2}

\begin{code}
bind (Fork u v) f = Fork (bind u f) (bind v f)
bind (Leaf x)   f = f x
\end{code}

\begin{align*}
\onslide<2->{\rom{1} &&& \bind{\fork{u}{v}}{f} = \fork{\bind{u}{f}}{\bind{v}{f}}} \\
\onslide<3->{\rom{2} &&& \bind{\leaf{x}}{f}    = \app{f}{x}} \\
\onslide<5->{\rom{3} &&& t \neq \fork{\forkl{t}}{\forkr{t}} \wedge t \neq\leaf{\unleaf{t}} \rightarrow \bind{t}{f} = \bot}
\end{align*}

\begin{align*}
\onslide<4->{\rom{4} &&& \leaf{x}    \neq \fork{u}{v}} & \onslide<7->{\rom{7} &&& \forkl{\fork{u}{v}} = u}\\
\onslide<6->{\rom{5} &&& \leaf{x}    \neq \bot       } & \onslide<7->{\rom{8} &&& \forkr{\fork{u}{v}} = v}\\
\onslide<6->{\rom{6} &&& \fork{u}{v} \neq \bot       } & \onslide<7->{\rom{9} &&& \unleaf{\leaf{x}}   = x}
\end{align*}
\end{frame}

%% Describe this better

\begin{frame}[fragile]
\frametitle{Function pointers}
\begin{center}
$\rom{2} \quad \faa{x}{f} \bind{\leaf{x}}{f} = \app{f}{x}$

\vspace{2\baselineskip}

For each function in the theory, make a pointer with its axiom:

\begin{align*}
\app{\ptr{return}}{x}        & = \fn{return}(x) \\
\appp{\app{\ptr{bind}}{t}}{f} & = \bind{t}{f}
\end{align*}

\end{center}
\end{frame}

\begin{frame}[fragile]
  \frametitle{Limitations in the Haskell translation}
  \begin{itemize}
    \item Type classes
    \item Built-in types (\hs{Integer}, \hs{Int}, \hs{Char} $\ldots$)
    \item Sugar (list comprehensions, \hs{do}-notation)
    \item no \hs{seq}
  \end{itemize}
\end{frame}

%% Remove domino slide
%
%\begin{frame}
%  \frametitle{Induction}
%  \begin{center}
%    \includegraphics[height=6cm]{dominofalling} \\
%  \end{center}
%\end{frame}

\begin{frame}[fragile]
\frametitle{Supported Techniques}
\label{sec-4}
\begin{itemize}

\item Plain Equality \\
\item Structural Induction \\
\item Fixed Point Induction \\
\item Approximation Lemma

%\item Structural Induction\\
%\begin{code}
%prop_return_right t = t >>= return =:= t
%\end{code}
%
%\pause
%
%\item Approximation Lemma\\
%\begin{code}
%prop_return_left f x = return x >>= f =:= f x
%\end{code}
%
%\pause
%
%\item Fixed Point Induction\\
%\begin{code}
%prop_assoc t f g = (t >>= f) >>= g =:=
%                   t >>= (\x -> f x >>= g)
%\end{code}

\end{itemize} % ends low level
\end{frame}

\begin{frame}[fragile]
\frametitle{Plain Equality Example}

\begin{code}
prop_return_left f x = return x >>= f =:= f x
\end{code}

\end{frame}

\begin{frame}[fragile]
\frametitle{Structural Induction Example}

\begin{code}
prop_assoc t f g = (t >>= f) >>= g =:=
                   t >>= (\x -> f x >>= g)
\end{code}
\end{frame}

\begin{frame}[fragile]
\frametitle{Fixed Point Induction}
\begin{code}
bind' b (Fork u v) f = Fork (b u f) (b v f)
bind' b (Leaf x)   f = f x
\end{code}

\hs{bind = fix bind'}

\pause

\begin{mathpar}
  \inferrule*
    {
      P(\bot)
      \\
      P(b) \rightarrow P(\fn{bind'} \w b)
      \\
      P \, \mathrm{admissible}
    }
    { P(\fn{bind'}) }
\end{mathpar}

\pause

$$\bindp{b}{\fork{u}{v}}{f} = \fork{\appp{\app{b}{u}}{f}}{\appp{\app{b}{v}}{f}}$$

\end{frame}

\begin{frame}[fragile]
\frametitle{Fixed Point Induction Redux}

\begin{code}[mathescape]
Fork u v >>=$\fixb$ f = Fork (u >>=$\fixw$ f) (v >>=$\fixw$ f)
Leaf x   >>=$\fixb$ f = f x
\end{code}

\begin{mathpar}
  \inferrule*
     {
       P(\bot)
       \\
       P(\unfix{\fn{bind}}) \rightarrow P(\tofix{\fn{bind}})
       \\
       P \, \mathrm{admissible}
     }
     { P(\fn{bind}) }
\end{mathpar}

\end{frame}


\newcommand\lamptr[2]{(\appp{\app{\ptr{lam}}{#1}}{#2})}

\begin{frame}[fragile]
\frametitle{Bind Associativity with Fixed Point Induction}

\begin{code}
prop_assoc t f g = (t >>= f) >>= g =:=
                   t >>= (\x -> f x >>= g)
\end{code}

To show:

$$\bind{(\bindb{t}{f})}{g} = \bindb{t}{\lamptr{f}{g}}$$

where

$$\fn{lambda}(f,g,x) = \bind{\app{f}{x}}{g}$$

\pause

\vspace{2\baselineskip}

Cases: $t$ is one of $\bot$, $\leaf{x}$, $\fork{u}{v}$
\pause

\vspace{2\baselineskip}

$t \neq \fork{\forkl{t}}{\forkr{t}} \wedge t \neq\leaf{\unleaf{t}} \rightarrow \bindb{t}{f} = \bot$

\end{frame}

\begin{frame}[fragile]
\frametitle{Bind Associativity with Fixed Point Induction}

Case $\bot$. To show:
$$\bind{(\bindb{\bot}{f})}{g} = \bindb{\bot}{\lamptr{f}{g}}$$

\begin{align*}
lhs = & \; \bind{(\bindb{\bot}{f})}{g} \\
    = & \; \bind{\bot}{g}              \\
    = & \; \bot                        \\
    = & \; \bindb{\bot}{\lamptr{f}{g}} \\
    = & \; rhs
\end{align*}

\end{frame}

\begin{frame}[fragile]
\frametitle{Bind Associativity with Fixed Point Induction}

Case $\leaf{x}$. To show:
$$\bind{(\bindb{\leaf{x}}{f})}{g} = \bindb{\leaf{x}}{\lamptr{f}{g}}$$

where
$$\fn{lambda}(f,g,x) = \bind{\app{f}{x}}{g}$$

\begin{align*}
lhs = & \; \bind{(\bindb{\leaf{x}}{f})}{g} \\
    = & \; \bind{\app{f}{x}}{g}            \\
    = & \; \fn{lambda}(f,g,x)              \\
    = & \; \app{\lamptr{f}{g}}{x}          \\
    = & \; \bindb{\leaf{x}}{\lamptr{f}{g}} \\
    = & \; rhs
\end{align*}

\end{frame}

\begin{frame}[fragile]
\frametitle{Bind Associativity with Fixed Point Induction}

Case $\fork{u}{v}$. To show:
$$\bind{(\bindb{\fork{u}{v}}{f})}{g} = \bindb{\fork{u}{v}}{\lamptr{f}{g}}$$

Induction hypothesis:
$$\bind{(\bindw{t}{f})}{g} = \bindw{t}{\lamptr{f}{g}}$$


\begin{align*}
lhs = & \; \bind{(\bindb{\fork{u}{v}}{f})}{g} \\
    = & \; \fork{\bind{(\bindw{u}{f})}{g}}{\bind{(\bindw{v}{f})}{g}}\\
    = & \; \fork{\bindw{u}{\lamptr{f}{g}}}{\bindw{v}{\lamptr{f}{g}}} \\
    = & \; \bindb{\fork{u}{v}}{\lamptr{f}{g}} \\
    = & \; rhs
\end{align*}

\end{frame}


%\begin{frame}
%\frametitle{Properties with infinite values}
%map-iterate explanation
%\vspace{1cm}
%nothing to do induction on!

%\end{frame}

%\begin{frame}
%\frametitle{Approximation lemma}
%we approximate the result in the equality \\
%\vspace{1cm}
%IS: assume it holds for lists of length $n$, and show for length $n+1$
%\end{frame}
%
%\begin{frame}
%\frametitle{Fixpoint induction}
%over the recursive structure of the function \\
%\vspace{1cm}
%IS: assume for $\unfix{iterate}$, prove for $\tofix{iterate}$
%\end{frame}

%% DRAW THIS

\begin{frame}
\frametitle{Testsuite}
A mind map of the different parts of the test suite
\end{frame}


%%% REWRITE

\begin{frame}
\frametitle{Results}
\label{sec-5}


\begin{tabular}{ >{\small}r@{/}>{\small}l | >{\small}r@{/}>{\small}l | >{\small}r@{/}>{\small}l | >{\small}r@{/}>{\small}l | >{\small}r@{/}>{\small}l || >{\small}r@{/}>{\small}l }
\multicolumn{2}{>{\small}l|}{Theorem} & \multicolumn{2}{>{\small}l|}{plain} & \multicolumn{2}{>{\small}l|}{induction} & \multicolumn{2}{>{\small}l|}{approx} & \multicolumn{2}{>{\small}l||}{fixpoint} & \multicolumn{2}{>{\small}l}{Finite Thm.}  \\
214&540 & 74&214 & 124&214 & 145&214 & 26&214 & 111&540 \\
\end{tabular}
\end{frame}


\begin{frame}
\frametitle{Future Work: Lemma system - Totality}
\label{sec-6}


\centering
\hs{data Nat = Succ Nat | Zero}

\vspace{\baselineskip}

\begin{align*}
\rom{1} &&        & \neg \, \Total(\bot) \\
\rom{2} &&        & \Total(\fn{zero}) \\
\rom{3} && \fa{x} & \Total(x) \rightarrow \Total(\fn{succ}(x))
\end{align*}

\pause

\begin{equation*}
\faa{x}{y} \Total(x) \wedge \Total(y) \rightarrow x + y = y + x
\end{equation*}
\end{frame}
\begin{frame}
\frametitle{Future Work: Lemma system - Finiteness}
\label{sec-7}


\begin{center}
$\fa{xs} \Fin(xs) \rightarrow \fn{reverse}(\fn{reverse}(xs)) = xs$
\end{center}
\end{frame}
\begin{frame}
\frametitle{Future Work: Lemma system - Infiniteness}
\label{sec-8}


\begin{center}
$\fa{x} \Inf(x) \leftrightarrow (x = \fn{succ}(x))$
\end{center}

\vspace{1\baselineskip}

\pause

\begin{center}
$\Fin(x) \leftrightarrow \Total(x) \wedge \neg \Inf(x)$
\end{center}

\pause

\begin{align*}
\rom{1} && \faa{x}{y} & \Fin(x) \wedge \Fin(y)                & \leftrightarrow &&& \Fin(x + y) \\
\rom{2} && \faa{x}{y} & \Inf(x) \vee (\Fin(x) \wedge \Inf(y)) & \leftrightarrow &&& \Inf(x + y)
\end{align*}
\end{frame}

\begin{frame}[fragile]
\frametitle{Future Work: Lemma synthesis from QuickSpec}
\begin{verbatim}
== definitions ==
1: suc x := x+suc zero
== equations ==
1: y+x == x+y
2: y*x == x*y
3: x+zero == x
4: x*zero == zero
5: y+(x+z) == x+(y+z)
6: y*(x*z) == x*(y*z)
7: x+suc y == suc (x+y)
8: x*suc y == x+(x*y)
9: (x*y)+(x*z) == x*(y+z)
\end{verbatim}
\end{frame}

%% MOVE TO END
\begin{frame}
  \frametitle{Related Work}
  \begin{itemize}
    \item Zeno
    \item Combining Interactive and Automatic Reasoning about Functional Programs
          (Bove, Dybjer, Sicard)
  \end{itemize}
\end{frame}

\begin{frame}
\frametitle{Conclusion}

\begin{itemize}
\item FOL is expressible enough for various Haskell concepts; pattern
  matching, infinite data structures and higher order functions
\item Much thanks to referential transparency
\item Possible to prove properties without proving termination
\end{itemize}
\end{frame}


\begin{frame}
\frametitle{Old slides}
these are slides from the old presentation
\end{frame}


\begin{frame}
\frametitle{Future Work: GHC Core}
\label{sec-9}
\begin{itemize}

\item Mimic GHC Semantics\\
\label{sec-9-1}%
\item Type classes and desugaring for free\\
\label{sec-9-2}%
\end{itemize} % ends low level
\end{frame}

\begin{frame}
\frametitle{Name Contest!}
\label{sec-11}
\begin{itemize}

\item auhapp - AUtomated HAskell Property Prover\\
\label{sec-11-1}%
\item gluppe - GLUppe Proves Properties Equationally\\
\label{sec-11-2}%
\item glapp - GLapp Automatically Proves Properties\\
\label{sec-11-3}%
\item herp - Haskell Equational Reasoning Prover\\
\label{sec-11-4}%
\end{itemize} % ends low level
\end{frame}

\begin{frame}[fragile]
\frametitle{Dictionary passing}
\label{sec-12}


\begin{verbatim}
class Monoid a where
  mappend :: a -> a -> a
  mempty  :: a

mconcat :: Monoid a => [a] -> a
mconcat = foldr mappend mempty
\end{verbatim}

\pause

$\Rightarrow$

\begin{verbatim}
data Monoid a = Monoid { mappend :: a -> a -> a
                       , mempty  :: a }

mconcat :: Monoid a -> [a] -> a
mconcat m = foldr (mappend m) (mempty m)
\end{verbatim}
\end{frame}

\end{document}


%\begin{frame}
%\frametitle{HALT - HAskell to Logic Translator}
%target logic is untyped.
%
%handle constructors as functions, and use function pointers
%\end{frame}


%\begin{frame}
%  \frametitle{Motivation}
%  \begin{itemize}
%    \item Testing can only show presence of bugs, \\
%          proving shows absence of bugs
%    \item Theorem provers are good at proving
%    \item Infinite values from lazy data structures
%  \end{itemize}
%\end{frame}


%\begin{frame}
%  \frametitle{Induction step}
%  \begin{center}
%    \includegraphics[height=6cm]{dominostop}
%  \end{center}
%\end{frame}

%\begin{frame}[fragile]
%  \frametitle{Induction Example}
%  \begin{center}
%    \begin{verbatim}
%count :: Nat -> [Nat] -> Nat
%count n (x:xs) = if n == x then S (count n xs) else count n xs
%count n []    = Z
%
%prop_count :: Nat -> [Nat] -> [Nat] -> Prop Nat
%prop_count n xs ys
%  = count n xs + count n ys =:= count n (xs ++ ys)
%    \end{verbatim}
%  \end{center}
%\end{frame}
%
%\begin{frame}[fragile]
%  \frametitle{Infinite Lists}
%  \begin{verbatim}
%    natsFrom x = x : natsFrom (S x)
%  \end{verbatim}
%
%  \hs{natsFrom 0} is the infinite list \hs{0 : 1 : 2 : 3 : ...}
%  \\
%  How can we prove properties about infinite lists?
%\end{frame}
%
%\begin{frame}
%  \begin{center}
%    \includegraphics[height=8cm]{crashpdf}
%  \end{center}
%\end{frame}
%
%\begin{frame}[fragile]
%  \frametitle{Crashes}
%  \begin{center}



%  \Huge\Huge$$\bot$$
%  \end{center}
%  \begin{center}
%  ``bottom''
%  \begin{verbatim}
%         loop = loop
%         head []
%         undefined
%  \end{verbatim}
%  \end{center}
%\end{frame}

%\begin{frame}
%  \frametitle{Properties related to bottom}
%    \begin{itemize}
%      \item Monotonicity \\
%            $$x \sqsubseteq y \rightarrow f(x) \sqsubseteq f(y)$$ \\
%            $f$ cannot return more information from $\hs{1:}\bot$ than for $\hs{1:[]}$ \\
%\vspace{0.5cm}
%      \item Continuity \\
%            \hs{finite :: [a] -> Bool}
%  \end{itemize}
%\end{frame}

