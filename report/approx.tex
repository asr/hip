\section{Approximation Lemma}

The approximation lemma must be considered a standard technique for
proving properties about corecursive programs. Just like fixed point
induction it can be used with functions that are produce infinite
values, like \hs{repeat} and \hs{iterate}, out of abstract values
making structural induction impossible \cite{corecursive}.  The
approximataion lemma supersedes the classical take lemma
\cite{introfp} by being easier to apply and generalize: unlike the
take lemma, it can be applied to equalities of any polynomial
data type \cite{genapprox}. The definitions of \hs{take} and
\hs{approx}: % for lists can be viewed in figure \ref{fig:takeapprox}.

%% This is weird!

%\floatstyle{ruled}
%\newfloat{program}{thp}{lop}
%\floatname{program}{Program}

\note{how to put/float source code side by side?}
%\begin{program}
\begin{verbatim}
take :: Nat -> [a] -> [a]             approx :: Nat -> [a] -> [a]
take Zero    _      = []              approx (Suc n) []     = []
take (Suc n) []     = []              approx (Suc n) (x:xs) = x : approx n xs
take (Suc n) (x:xs) = x : take n xs
\end{verbatim}
%\caption{Definition of \hs{take} and \hs{approx} on lists, with Peano-\hs{Nat}s.}
%\label{fig:takeapprox}
%\end{program}
\note{write this as Haskell code? (those \hs{Nat} are actually $\mathbb{N}$)}

Whereas \hs{take} approximates a list and ends it with \hs{[]},
\hs{approx} ends it with $\bot$ since the \hs{Zero} case is
omitted. The idea of these techniques is then to show that show that
two lists are equal by showing that their prefix or approximation
coincides for all natural numbers. The approximation lemma is given thusly

\begin{equation}
\label{eq:approxeq}
xs \, = \, ys \quad \Leftrightarrow \quad \fa{n \in \mathbb{N}} \hs{approx} \, n \, xs = \hs{approx} \, n \, ys
\end{equation}

Equation \ref{eq:approxeq} quantifies over the \emph{true} natural
numbers, rather than the \emph{polluted} Haskell naturals. Showing an
equality then amounts to a proof by induction over natural numbers,
and the base case for $0$ is always true by reflexivity, as the
approximation is $\bot$.
The right to left implication is (trivially) true by substitution, and
the other direction hinges on the lemma that better and better
approximations form a chain with limit \hs{id}, as illustrated in
Equation \ref{eq:approxchain}.

\begin{equation}
\label{eq:approxchain}
\hs{approx} \, 0 \,
   \sqsubseteq \,
\hs{approx} \, 1 \,
   \sqsubseteq \,
\cdots
   \sqsubseteq \,
\hs{approx} \, n \,
   \sqsubseteq \,
\hs{approx} \, (\hs{Suc} \, n) \,
   \sqsubseteq \,
\cdots
   \sqsubseteq \,
\hs{id}
\end{equation}

The inclusions in \ref{eq:approxchain} are easily given by induction
on natural numbers and the limit by structural induction on lists.
For other polynomial data types, this lemma is established by
the structural induction induced on that data type.
The desired implication is then readily deduced:

\newcommand{\xsys}[2]{#1 \, xs \, #2 & = #1 \, ys #2}
\newcommand{\desca}[1]{  & \hspace{44.5mm}                              \{ \mathrm{#1} \}}
\newcommand{\descra}[1]{ & \hspace{35mm} \Rightarrow     \hspace{4mm} \{ \mathrm{#1} \}}
\newcommand{\descla}[1]{ & \hspace{35mm} \Leftarrow      \hspace{4mm} \{ \mathrm{#1} \}}
\newcommand{\desclra}[1]{& \hspace{35mm} \Leftrightarrow \hspace{4mm} \{ \mathrm{#1} \}}
\begin{align*}
\xsys{\fa{n} \hs{approx} \, n}{}            \\
\descra{limits}                             \\
\xsys{\lub{n} \, (\hs{approx} \, n}{)}      \\
\desclra{continuity \, of \, application}   \\
\xsys{\lub{n} \, (\hs{approx} \, n)}{}      \\
\desclra{Equation \, \ref{eq:approxchain}} \\
\xsys{\hs{id}}{}                            \\
\desclra{definition \, of \, \hs{id}}       \\
\xsys{}{}                                   \\
\end{align*}

\subsection{Example: Mirroring an Expression}

Consider these definitions of a modest but prototypical expression
data type, and its mirroring function:

\begin{verbatim}
data Expr = Add Expr Expr | Value Nat

mirror :: Expr -> Expr
mirror (Add e1 e2) = Add (mirror e2) (mirror e1)
mirror (Value n)   = Value n

prop_mirror_involutive :: Expr -> Prop Expr
prop_mirror_involutive e = e =:= mirror (mirror e)
\end{verbatim}

Two things to notice here is that the type \hs{Expr} does not have a
nullary constructor. Then the \hs{take} lemma would not be usable as
there is no such function over these expressions: the list version
returns the empty list \hs{[]} for the zero case, but there is no such
alternative for \hs{Expr} above. It is important that the limit of
approximations is the identity, and we cannot get this property when
trying to generalise the take lemma.

Furthermore, fixed point induction fails on this property: choosing
either or both occurrences of \hs{mirror} on the right side is
constant bottom for the base case, and the left side is the identity.


We shall now proceed to prove that \hs{mirror} is involutive by the
approximation lemma. The approximation function for \hs{Expr} is
automatically generated, by approximating each self-recursive
constructor, and hence \hs{Value}'s \hs{Nat} is not further approximated:

\begin{verbatim}
approx :: Nat -> Expr -> Expr
approx (Suc n) (Add e1 e2) = Add (approx n e1) (approx n e2)
approx (Suc n) (Value n)   = Value n
\end{verbatim}

\note{Use $\bot$ or bottom in this text?}
Indeed, we also get a third case for $\bot$ which states that the
approximation of $\bot$ is, quite unsurprisingly, $\bot$.
As always in proofs by approximation lemma, we proceed by induction
over natural numbers, and the base case is always trivial: true by
reflexivity as both sides are $\bot$. The step case - which indeed is
the only proof obligation in any proof of this kind - is to prove
this:

\begin{equation*}
\fa{e}  approx \, (\hs{Suc} \, n) \, e = approx \, (\hs{Suc} \, n) \, (mirror \, (mirror \, e))
\end{equation*}

An important property of the induction hypothesis is the universal
quantification of the expression $e$, unlike the fixed natural number
$n$:

\begin{equation*}
\fa{e}  approx \, n \, e = approx \, n \, (mirror \, (mirror \, e))
\end{equation*}

The proof is by case exhaustion. The case for \hs{Value} and $\bot$
are trivial: \hs{mirror} is strict in its first argument, and
mirroring \hs{Value} twice is the identity, so these cases are both
true by reflexivity. The \hs{Add} case is ever so slightly more
elaborate, and with names shortened to \hs{app} and \hs{mir} the
reasoning is as follows:

\note{$(\hs{Suc} \, n)$ or $(n + 1)$?}
\newcommand{\Adds}[2]{\hs{Add} \, #1 e_1 #2 \, #1 e_2 #2}
\newcommand{\Approxn}[0]{\hs{app} \, n \,}
\newcommand{\ApproxSucn}[0]{\hs{app} \, (\hs{Suc} \, n) \,}
\newcommand{\mirmir}[0]{\hs{mir} \, (\hs{mir} \, }
\begin{align*}
\faa{e_1}{e_2}&  \ApproxSucn (\Adds{}{})  = \ApproxSucn (\mirmir \Adds{}{} ))                                                                   \\
                                                                                 \desclra{\defof{\hs{mirror}}}                                   \\
\faa{e_1}{e_2}&  \ApproxSucn (\Adds{}{})  = \ApproxSucn (\Adds{(\mirmir}{))})                                                                    \\
                                                                                \desclra{\defof{\hs{approx}}}                                    \\
\faa{e_1}{e_2}&  \Adds{(\Approxn}{)}      = \Adds{(\Approxn(\mirmir}{)))}                                                                        \\
                                                                                \desclra{induction \, hypothesis \, twice \, (e_1 \, and \, e_2)} \\
\faa{e_1}{e_2}&  \Adds{(\Approxn}{)}      = \Adds{(\Approxn}{)}                                                                                  \\
                                                                                \desca{reflexivity}                                              \\
\end{align*}

\subsection{Approximation Lemma is Fixpoint Induction}

This technique is already simple and widely applicable, however it can
further be simplified. Implementing it in this form relies on the
auxiliary structure of Peano natural numbers which also needs to be
added to the theory. This can be removed by the observation that is
\note{What is a better name for these functions than \hs{id}?}
expressed as fixed point induction over a recursive form of \hs{id}.

\begin{verbatim}
id :: [a] -> [a]              id :: Expr -> Expr
id []     = []                id (Add e1 e2) = Add (id e1) (id e2)
id (x:xs) = x : id xs         id (Value n)   = Value n
\end{verbatim}

Each \hs{id} function constructed in this way is indeed an identity
function, equivalent to the implementation \hs{id x = x} if we
disregard time and space complexity. Now, to prove

\begin{equation*}
e_1 \, = \, e_2
\end{equation*}

we simply use fixed point induction to prove

\begin{equation*}
\hs{id} \, e_1 \, = \, \hs{id} \, e_2
\end{equation*}

where \hs{id} is a such a specialized recursive identity function over
the data type of the equality. With the same translation of recursive
functions as in the fixed point section, the axioms for \hs{id} for, say
lists becomes:

\begin{align*}
            & \tofix{id}(\hs{[]}) \,   = \, \hs{[]}                                                           \\
\faa{x}{xs} & \tofix{id}(x\hs{:}xs) \, = x\hs{:}\unfix{id}(xs)                                                \\
\fa{xs}     & \tofix{id}(xs)           = \bot  \leftarrow    xs \neq [] \wedge xs \neq head(xs)\hs{:}tail(xs) \\
\end{align*}

The step case in induction $P(\unfix{id}) \rightarrow P(\tofix{id})$
is then exactly the same strength as the approximation lemma with
natural numbers.

\subsection{Implementation} The implementation of approximation lemma
was the simplest to implement, after definitional equality. First,
from the type signature from the property, such a recursive identity
function as above is generated for the data type of the equality. Then
the lemma $P(\unfix{id})$ is added to the theory, with $P$
instantiated to the universally quantified equality, and the
conjecture is $P(\tofix{id})$. The base case need not be proven,
$P(\bot)$ is always true since it evaluates to $\bot=\bot$.

\subsection{Future Work: Total Approximation Lemma}

It is also possible to adjust the approximation lemma to prove
properties that are true for total and potentially infinite objects,
but false for objects with partial values. One such property is the
idempotence of \hs{nub}. Here is a version of \hs{nub} on booleans,
and the said property about it:

\begin{verbatim}
nub :: [Bool] -> [Bool]
nub (True :True :xs) = nub (True:xs)
nub (False:False:xs) = nub (False:xs)
nub (x:xs)           = x:nub xs
nub _                = []

prop_nub_idem :: [Bool] -> Prop [Bool]
prop_nub_idem xs = nub (nub xs) =:= nub xs
\end{verbatim}

Consider the list \hs{True:False:}$\bot$. One application of \hs{nub}
gives \hs{True:}$\bot$, and two gives $\bot$ immediately. In spite of
this, the property is a truism for finite as well as infinite lists,
provided that there are not bottoms.

The way to enable the approximation lemma to prove such properties is
to add predicates of totality, and add axioms like the following to
the theory:

\note{How to make $Total$ look like one word? The kerning for $T$ in
  $Total$ looks incorrect}
\begin{align*}
            & \neg \, Total(\bot) \\
            & Total(\hs{[]}) \\
\faa{x}{xs} & Total(x) \wedge Total(xs) \rightarrow Total(x\hs{:}xs) \\
\fa{xs}     & Total(xs) \rightarrow Total(\hs{nub} \, xs) \\
\end{align*}

To use fixed point induction \hs{Total} needs to be an admissible
predicate. But this is indeed so, for each model with infinite lists,
we have that all such lists are total.  However, that a total argument
to \hs{nub} yields a total value also needs to be proven. Furthermore,
the totality property for the cons case of lists is debatable: should
the consed element also be total?

The proof obligation is then:

\begin{mathpar}
  \inferrule*
     {
        \fa{xs} Total(xs) \Rightarrow \unfix{id}(\hs{nub} (\hs{nub} \, xs)) = \unfix{id}(\hs{nub} \, xs)
     }
     {
        \fa{xs} Total(xs) \Rightarrow \tofix{id}(\hs{nub} (\hs{nub} \, xs)) = \tofix{id}(\hs{nub} \, xs)
     }
\end{mathpar}

It seems like you have to add
$Total(x) \rightarrow Total(\tofix{id}(x))$ and
$Total(x) \rightarrow Total(\unfix{id}(x))$ to the theory, too.

\subsection{Uncategorized / old}

After fixed point induction since it is an easy consequence of fixed
point induction, and how we removed the auxiliary structure of natural
numbers. This makes it equivalent to fixed point induction of id on
both sides (or does it?)

Approximation lemma is a generalization of the take lemma, and its
first form is used for properties about infinite and partial lists,
but it is easily generalized to other recursive data types.
In particular, all polynomial data types — for example, any
sum-of-products data type — can be defined in this way.
This result generalizes
to mutually recursive, parameterised, exponential and nested datatypes, but
for simplicity we only consider polynomial datatypes in this article.
\cite{genapprox}

How to use approximation lemma on exponential, mutually recursive and
nested data types? This would be nice.