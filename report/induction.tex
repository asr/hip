\section{Structural induction}
\label{sec:induction}

Properties that are true merely by rewriting the definitions are
neither abundant nor especially interesting. The fundamental concept
of constructors and pattern matching in Haskell is closely related to
induction. The values of an argument with a concrete type can only
range over the data type's different constructors. Such a case
analysis combined with induction is especially strong.

Induction is a very fundamental concept of mathematics and is directly
or indirectly an axiom in axiomatizations of mathematics. We shall
take the Peano Arithmetic, $\PA$, axioms for true. This theory has
the natural numbers as standard model with a small vocabulary
consisting only of the constant $0$, the unary successor function $s$,
and binary plus and multiplication. Induction in $\PA$ allows proving
properties that hold for all natural numbers. The first proof
obligation in a proof by induction is to prove that the property holds
for 0. For the second it is if it holds for an arbitrary then it must
hold for its successor. The induction schema is formally written as this:

\begin{mathpar}
  \inferrule* % [Left=PA-Ind]
     {
       \overbrace{P(0)}^{\mathrm{base}}
       \\
       \overbrace{
           \fa{x}
                 \underbrace{P(x)}_{\mathrm{hypothesis}}
              \rightarrow
                 \underbrace{P(s(x))}_{\mathrm{conclusion}}
       }^{\mathrm{step}}
     }
     { \fa{x} P(x) }
\end{mathpar}

Different parts of this rule has been highlighted that are common
names for induction proofs: the induction base, the induction step
with the assumed hypothesis and the obligated conclusion.  This is a
axiom schema since it is not possible to quantify over the predicate
$P$ in FOL. Rather, it is a infinite set of axioms, one for each
(well-formed) formula instantiated in place for $P$. Generally, ATPs
do not instantiate schemas themselves but it has to be done manually,
with an appropriate formula for $P$. In this thesis, the predicates
will mainly be equalities that depend on the argument to $P$. For
example, if you want to prove that the functions $f$ and $g$ are
equal, you can then define $P(x)$ to $f(x)=g(x)$.

To prove a property with induction in Haskell, induction can be
carried out on the number of constructors in a value. It is possible
to get many base cases, one for each non-recursive constructor and
then a step case for each recursive constructor. As an example, we can
consider one of the simplest recursive data structure in Haskell, the
Peano Natural numbers in Haskell, defined as
\hs{data Nat = Zero | Succ Nat}. This axiom schema is used and
follows the same structure as in $\PA$:

\begin{mathpar}
  \inferrule*
     {
       P(\hs{Zero})
       \\
       \fa{x} P(x) \rightarrow P(\hs{Succ} \w x)
     }
     { \fa{x} x \text{ finite and total} \rightarrow P(x) }
\end{mathpar}

Any data type gives rise to induction schemata, and this is called
structural induction. These are the axiom schemata for lists and the
\hs{Tree} type defined in \ref{sec:treetrans}:

\begin{mathpar}
  \inferrule* % [left=List-Ind]
     {
       P(\hs{[]})
       \\
       \faa{x}{xs} P(xs) \rightarrow P(x\hs{:}xs)
     }
     { \fa{xs} xs \text{ finite and total} \rightarrow P(xs) }

  \inferrule* % [left=Tree-Ind]
     {
       P(\hs{Empty})
       \\
       \faaa{l}{x}{r} P(l) \wedge P(r) \rightarrow P(\hs{Branch} \w l \w x \w r)
     }
     { \fa{t} t \text{ finite and total} \rightarrow P(t) }
\end{mathpar}

For such simple indexed sum of products data types the translation is
straightforward: for a data type $T$, with a constructor $K$ you get
all recursive arguments of $T$ to $K$ as hypotheses.

\subsection{Partial and Infinite Values}
\label{sec:admissible}

To use induction to prove properties about infinite and partial
values, we need to use predicates that preserves $\omega$-chains of
$\sqsubseteq$. Such chains are described in more detail in
section \ref{sec:domaintheory}, an example of such a chain can for
instance be this one:

\begin{equation*}
\bot \sqsubseteq
\hs{Succ} \w \bot \sqsubseteq
\hs{Succ} \w (\hs{Succ} \w \bot) \sqsubseteq
\hs{Succ} \w (\hs{Succ} \w (\hs{Succ} \w \bot)) \sqsubseteq
\cdots
\end{equation*}

This chain has the limit \hs{inf}, defined as \hs{inf = Succ inf}. If
$P$ is an admissible predicate, then if $P$ holds for every element in
that chain then $P(\hs{inf})$. In general terms, preserving
$\omega$-chains means that for every chain $\langle x_n \rangle_{n \in
  \omega}$} this holds:

\begin{equation*}
(\fa{n} P(x_n)) \Rightarrow P(\lub{n \in \omega}(x_n))
\end{equation*}

The properties that we are concerned about are equality, and equality
of continuous functions is easily proved to admissible. Assume two
such functions $f$ and $g$, and define the predicate as $P(x)
:\Leftrightarrow f(x) = g(x)$. Take any chain $\langle x_n \rangle_{n
  \in \omega}$, and assume $P$ holds for every element in the chain,
then:

\begin{align*}
\desca{assumption}               \\
& \fa{n} P(x_n)                    \\
\desclra{definition}             \\
& \fa{n} f(x_n) = g(x_n)           \\
\descra{limits}                  \\
& \lub{n}(f(x_n)) = \lub{n}(g(x_n))  \\
\desclra{continuity}             \\
& f(\lub{n}(x_n)) = g(\lub{n}(x_n))  \\
\desclra{definition}             \\
& P(\lub{n}(x_n))
\end{align*}

This result generalizes to equalities of compositions of continuous
functions. One way to approach this is to use that the continuous
functions over a complete partial order form a category with products
and exponentials which has the necessary components of composition and
application.

In short: if you want to show that an admissible property holds for
partial and infinite values you also need to consider $\bot$ as a
constructor for the data type. As an example, the induction principle
for possibly partial and infinite lists is:

\begin{mathpar}
  \inferrule* % [left=List-Ind]
     {
       P(\bot)
       \\
       P(\hs{[]})
       \\
       \faa{x}{xs} P(xs) \rightarrow P(x\hs{:}xs)
       \\
       P \text{ admissible}
     }
     { \fa{xs} P(xs) }
\end{mathpar}


\subsection{Generalizations}
\label{sec:genind}

The structural induction schemata we have seen so far only unses each
constructor once and that does not work for proving properties about
functions defined with recursion with a bigger depth. For instance,
the induction on Peano numbers in Haskell is adjusted to a constructor
depth two in this way:

\begin{mathpar}
  \inferrule* % [left=Nat-Ind-Depth-Two]
     {
       P(\bot)
       \and
       P(\hs{Zero})
       \and
       P(\hs{Succ Zero})
       \\
       \fa{x} P(x) \wedge P(\hs{Succ} \w x) \rightarrow P(\hs{Succ} \w (\hs{Succ} \w x))
     }
     { \fa{x} P(x) }
\end{mathpar}

Given how many constructors you maximally want to unroll for your data
type, you have to prove the property for all combinations or
constructors up to including that limit, but for an induction step
with a conclusion with $i$ constructors, the induction hypothesis is
the conjunction of all combinations with strictly less than $i$
constructors.

\newpage

Say you want to prove a property that has a list of natural numbers as
property, with up to two constructors, these are the required
induction steps:

\begin{align*}
\rom{1} &&                                                                & P(\hs{[]}) \\
\rom{2} &&                                                                & P(\hs{Zero}\hs{:}\hs{[]}) \\
\rom{3} && \fa{x} P(x\hs{:}\hs{[]}) \rightarrow                        \, & P(\hs{Succ} \w x\hs{:}\hs{[]}) \\
\rom{4} && \faaa{x}{y}{zs} P(x\hs{:}zs) \wedge P(y\hs{:}zs)\rightarrow \, & P(x\hs{:}y\hs{:}zs) \\
\end{align*}

Similarly, this is used to use induction on more than one arguments,
by using the tuple constructor. Induction on two natural numbers is this:

\begin{align*}
\rom{1} &&                                                                                        & P(\hs{Zero}     ,\hs{Zero}     ) \\
\rom{2} && \fa{x} P(x,\hs{Zero})                                                   \rightarrow \, & P(\hs{Succ} \w x,\hs{Zero}     ) \\
\rom{3} && \fa{y} P(\hs{Zero},y)                                                   \rightarrow \, & P(\hs{Zero}     ,\hs{Succ} \w y) \\
\rom{4} && \faa{x}{y} P(x,y) \wedge P(\hs{Succ} \w x,y) \wedge P(x,\hs{Succ} \w y) \rightarrow \, & P(\hs{Succ} \w x,\hs{Succ} \w y) \\
\end{align*}

This can be used to show the commutativity of plus for natural numbers
defined recursively on the first argument. The predicate is defined as
$P(x,y) \Leftrightarrow \faa{x}{y} x + y = y + x$. Two lemmas are
needed if you do this with induction on the first argument:
\begin{itemize}
  \item $\fa{x} x + \hs{Zero} = x$. This corresponds the hypothesis
    $P(x,\hs{Zero})$ above.
  \item $\faa{x}{y} \hs{Succ} \w x + y = x + \hs{Succ} \w y$. This is
    derivable from two of the hypotheses in the last step above:
    \begin{align*}
    \hs{Succ} \w x + y   & = \{\mathrm{definition \w of \w \mathit{+}}\} \\
    \hs{Succ} \w (x + y) & = \{\mathrm{hypothesis \w \mathit{P(x,y)}}\} \\
    \hs{Succ} \w (y + x) & = \{\mathrm{definition \w of \w \mathit{+}}\} \\
    \hs{Succ} \w y + x   & = \{\mathrm{hypothesis \w \mathit{P(x,\hs{Succ} \w y)}}\} \\
    x + \hs{Succ} \w y
    \end{align*}
\end{itemize}

The commutativity of plus does only hold for total values since we have
$$\bot \eq \bot + \hs{Succ Zero} \, \neq \, \hs{Succ Zero} + \bot \eq \hs{Succ} \w \bot$$

This kind of unrolling data types also work well with mixed data
types, a standard example is the definition of $\mathbb{Z}$ defined
with a disjoint union as $\mathbb{N} + \mathbb{N}$, or in Haskell

\begin{verbatim}
data Z = Pos Nat | Neg Nat
\end{verbatim}

A value of $\hs{Pos} \w x$ denotes the integer $+\, x$ and $\hs{Neg} \w y$
the integer $-(1+y)$. This avoids having two values denoting
$0$. With the above technique, the induction principle for total
\hs{Z} with two constructors is:

\begin{mathpar}
  \inferrule* % [left=Nat-Ind-Depth-Two]
     {
       P(\hs{Pos Zero})
       \and
       \fa{x} P(\hs{Pos} \w x) \wedge P(\hs{Neg} \w x) \rightarrow P(\hs{Pos} \w (\hs{Succ} \w x))
       \\
       P(\hs{Neg Zero})
       \and
       \fa{x} P(\hs{Pos} \w x) \wedge P(\hs{Neg} \w x) \rightarrow P(\hs{Neg} \w (\hs{Succ} \w x))
     }
     { \fa{x} P(x) }
\end{mathpar}

\subsection{Skolemized Hypotheses}

The universally quantified variables are skolemized, a new constant is
introduced for each and that means that they can be referred to in
different axioms added in the theory. For example, when adding the
\hs{Branch} step for an induction proof over the \hs{Tree} data type,
we could run the theorem checker like this:

\begin{equation*}
T \vdash \faaa{l}{x}{r} P(l) \wedge P(r) \rightarrow P(\hs{Branch} \w l \w x \w r)
\end{equation*}

Rather, skolemize each universally quantified variable and prove this:

\begin{equation*}
T , P(l) , P(r) \vdash P(\hs{Branch} \w l \w x \w r)
\end{equation*}

This prevents long formulas should there be many hypotheses.  It also
makes the implementation of the simple induction technique with one
argument and one constructor straightforward. For each constructor $C$
with $n$ arguments, we make a new call to a theorem prover with $n$
Skolem variables $x_1 \cdots x_n$ and try to prove
$P(x_1,\cdots,x_n)$.  Additionally, for each argument with assigned
variable $x_i$, if the type of this variable is the same as the
constructor $C$'s, $P(x_i)$ is added as an axiom to the theory: this
an induction hypothesis.

The predicate is inlined: there is no introduction of $P$ and its
definition, it is replaced with the property. Suppose we want to prove
the property
$\fa{t} \hs{mirror} \w (\hs{mirror} \w t) = t$ where $t : \hs{Tree a}$,
the \hs{Branch} case generates this call to the theorem prover:

\begin{align*}
\mathcal{T} \, & , \, \hs{mirror} \w (\hs{mirror} \w x_1) \eq x_1 \\
            \, & , \, \hs{mirror} \w (\hs{mirror} \w x_3) \eq x_3 \\
               &\vdash \hs{mirror} \w (\hs{mirror} \w (\hs{Branch} \w x_1 \w x_2\w x_3) \eq \hs{Branch} \w x_1 \w x_2\w x_3
\end{align*}

Here $\mathcal{T}$ is the theory for this Haskell program: it includes the
definition of $\hs{mirror}$, along with axioms that $\hs{Branch}$,
$\hs{Empty}$ and $\bot$ are disjoint.  Hypotheses for $x_1$ and $x_3$
are added since the first and the third argument for the \hs{Branch}
constructor is \hs{Tree a}. For $x_2$, nothing is added since the type
of this is simply \hs{a}.  For this property, in total three
invocation of the theorem prover is made, one for each constructor.

\begin{comment}
    \note{I don't elaborate on this too much since the current
      implementation generates tree of a given depth instead of a given
      number of constructors. Specifying constructors could potentially
      give you less granularity: I need to investigate this}
    To generate theories for the approach in \ref{sec:genind} an algorithm
    was written to generate trees with a number of constructors. For the
    commutativity of plus example, you would start with the expression
    \hs{(x :: Nat,y :: Nat)}, and for each occurrence of a typed variable,
    you replace it with its constructors, and combine. Then an other
    algorithm takes a typed expression, like
    \hs{(Succ (x :: Nat),Succ (y :: Nat))} and returns all expressions
    with less constructors you can generate with the correct type with
    the typed variables, here \hs{(Nat,Nat)}. Then you proceed as in the
    simple case: one invocation to the theorem prover for each step,
    variables are skolemized and the predicate inlined.

    This leads to a combinatorial explosion for a lot of data types. If
    you need to prove something with length $n$ lists, you get a induction
    conclusion $P(x_1\hs{:}x_2\hs{:}\cdots\hs{:}x_n\hs{:}xs)}$, and as there
    are $n!$ combinations of $x_1 \cdots x_n$, making this approach
    unusable for high $n$. A better way would probably to just give you
    all sub trees instead of all possible trees: again, this of course
    depends on what you need to show.
\end{comment}

\subsection{Future work}
\label{sec:futind}

\paragraph{Exponential Types} It is also possible to get structural
induction for exponential data types. An example is higher-order
abstract syntax, another is the Brouwer ordinals defined by
\citet{dixonphd} as:

\begin{code}
data Ord = Zero | Succ Ord | Lim (Nat -> Ord)
\end{code}

The induction schema for this data type is:

\begin{mathpar}
  \inferrule* % [left=Ord-Ind]
     {
       P(\bot)
       \\
       P(\hs{Zero})
       \\
       \fa{x} P(x) \rightarrow P(\hs{Succ} \w x)
       \\
       \fa{f} (\fa{x} P(f \w x)) \rightarrow P(\hs{Lim} \w f)
     }
     { \fa{x} P(x) }
\end{mathpar}

This was never implemented since the lack of examples with exponential
data types.

\paragraph{Lexicographic Induction} In the rest of this section we
will mainly be discuss induction in $\PA$, again with $0$ as zero and
the successor function $s$.  If you need induction on two variables,
to show $\faa{x}{y} P(x,y)$, an other way is to use lexicographic
induction. First you do induction on $x$ to get the proof obligations
$\fa{y} P(0,y)$ and
$(\fa{y} P(x,y)) \rightarrow (\fa{y} P(s(x),y))$.
Notice that in the step case, $y$ is universally quantified in both
the hypothesis and in the conclusion. In both the base and in the step
you can now do induction on $y$ to get this inference rule:

\begin{mathpar}
  \inferrule* % [left=Nat-Ind-Depth-Two]
     {
       P(0,0)
       \and
       \inferrule{P(0,y)}{P(0,s(y))}
       \and
       \inferrule{\fa{y'} P(x,y')}
                 {P(s(x),0)}
       \and
       \inferrule{\fa{y'} P(x,y') \and P(s(x),y)}
                 {P(s(x),s(y))}
     }
     { \faa{x}{y} P(x,y) }
\end{mathpar}

This is not yet implemented since it was not clear which examples
would benefit from it, and you get many combinations of how to apply
this principle: in the example above we first did induction on $x$,
and then on $y$, but we could just as well do $y$ then $x$. If a
property has $n$ variables that could be used for induction, and you
use all variables, you have $n!$ ways of doing lexicographic
induction. If you want all subsets of this, you get even more
combinations. The simplest solution to this problem, as well as the
problem with how many constructors you want to use in the generalized
induction is to allow the user to annotate in the source code the
desired way to do induction: furthermore, it does not blend well that
this should be a fully automated way of proving Haskell properties.


\paragraph{Automated Depth}
Another way to do this is to encode
different induction ways by means of axioms in the theory itself. A
simple example of this is to prove a property $P$ on one Peano natural
number, and choose how many base cases in form of $P(0)$, $P(1)$, up
to $P(n)$ to get a stronger induction hypothesis $P(x)\wedge
P(x+1)\wedge \cdots\wedge P(x+n-1)\rightarrow P(x+n)$. It is possible
to axiomatize such a ``machine'' in first order logic as this:

\newcommand{\QZ}[2]{\inferrule* [left=q-0]{ }{Q(#1,#2)}}
\newcommand{\QS}[3]{\inferrule* [left=q-s]{P(#2) \and #3}{Q(#1,#2)}}
\newcommand{\IPZ}[2]{\inferrule* [left=+-0]{#1}{#2}}
\newcommand{\IPS}[2]{\inferrule* [left=+-s]{#1}{#2}}

\begin{mathpar}
  \QZ{0}{x}
  \and
  \QS{s(n)}{x}{Q(n,s(x))}
  \and
  \IPZ{ }{0 + x=x}
  \and
  \IPS{ }{s(n) + x=s(n+x)}
  \\
  \inferrule*{\ex{n} Q(n,0) \wedge
                     \fa{x} Q(n,x) \rightarrow P(n+x)}
             {\fa{x} P(x)}
\end{mathpar}

The $n$ in the existential quantifier is how deep the induction needs
to go. $Q$ is a predicate which both gives a suitable induction base
and hypothesis for this $n$. For the base case, you need to prove
$P(0)$, $P(1)$, up to $P(n-1)$. In the induction step, you will have
$P(x)$, $P(x)$, up to $P(x+n-1)$ as hypothesis, since it is the
antecedent in the implication. The consequent is $P(n+x)$, so we need
to have the suitable axioms for $+$ in the theory. The degenerate case
come from $n=0$:

\begin{mathpar}
  \inferrule*{
     \QZ{0}{0}
     \and
     \fa{x}      \QZ{0}{x}
     \rightarrow \IPZ{P(x)}{P(0+x)}
    }
    {\fa{x} P(x)}
\end{mathpar}

\newpage
A little bit of expanding is necessary for $n=2$. It looks like this

\begin{mathpar}
    \inferrule*
      {
        \QS{2}{0}{\QS{1}{1}{\QZ{0}{2}}}
       \\
       \fa{x}      \QS{2}{x}{\QS{1}{s(x)}{\QZ{0}{s(s(x))}}}
       \rightarrow \IPS{
                        \IPS{
                              \IPZ{P(s(s(x)))}
                                  {P(s(s(x+0)))}
                            }
                            {P(1+s(x))}
                       }
                       {P(2+x)}
      }
      {
        \inferrule*
          {P(0) \and P(1) \and \fa{x} P(x) \wedge P(s(x)) \rightarrow P(s(s(x)))}
          {\fa{x} P(x)}
      }
\end{mathpar}

The complexity is greatly increased with more irregular data types
than natural numbers, and it is unclear how well theorem provers would
be able to handle this.

