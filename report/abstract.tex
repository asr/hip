\thispagestyle{empty}

\section*{\centering \begin{normalsize}Abstract\end{normalsize}}

\begin{quotation}
\noindent
This Master thesis gives a new tool to automatically verify equational
properties written in the functional programming language Haskell,
with the aim to be able to reason about infinite and partial values
available in Haskell from general recursion and lazy data
structures. The novelty of this approach is to use automated theorem
provers for first order logic by means of a translation from Haskell
to first order theories. The properties are instantiated with
different induction techniques applicable to non strict functional
languages: structural induction, fixed point induction and the
approximation lemma. As the target logic is untyped, Haskell features
such as pattern matching and higher order functions needs to be dealt
with special care. The results from using the tool on a test suite are
convincing as the automated provers quickly deduce theorems for a
variety of properties. To be able to compete on fair grounds with
contemporary tools a system for adding lemmas would be required, which
turned out to be difficult as theorems for finite and infinite values
do not coexist peacefully out of the box.  Our conclusion is that
first order logic is able to express the various constructs in Haskell
and that theorem provers perform well on the generated theories, and
this without having to prove termination.
\end{quotation}
