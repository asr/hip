\chapter{Results and Discussion}
\label{ch:results}

This chapter describes the test suite used in this project. A large
test with five theorem provers was carried out on this test suite, and
this section describes the setup and results.

\section{Test Suite}

To carry out a throughout testing of the system a large test suite was
developed and collected from other sources.  It consists of over 25
Haskell source files, and all together includes over 500 equality
properties.  The files of can be divided into groups with different
aims of complexity, suitable proving methods and different aspects of
Haskell. All files of the test suite are briefly described in Table
\ref{tbl:testsuite}.

\begin{table}[p]
\centering
\begin{tabular}{>{\small}l | >{\small}l | >{\small}p{6cm} }
Group              & File                   & Description \\
\hline
Fundamental        & $\ts{Bool                  }$ & Simple properties about and, or and not \\
                   & $\ts{Expr                  }$ & Expressions with mirror, size and eval \\
                   & $\ts{Functions             }$ & Function composition, currying \\
                   & $\ts{Integers              }$ & Integers as a disjoint union of Nats \\
                   & $\ts{Reverse               }$ & Reverse with and without accumulator \\
                   & $\ts{Queues                }$ & Queues with $O(1)$ pop and enqueue \\
Infinite values    & $\ts{Infinite              }$ & Infinite list and tree properties \\
                   & $\ts{Sequences             }$ & Infinite sequences \\
                   & $\ts{Streams               }$ & Streams from \citep{streams} \\
Miscellaneous      & $\ts{PatternMatching       }$ & \hs{||} and \hs{mirror} in different ways \\
                   & $\ts{Fix                   }$ & Even and odd defined using \hs{fix} \\
                   & $\ts{Tricky                }$ & Properties that hold for total infinite lists \\
Monads             & $\ts{MonadEnv              }$ & The environment monad\\
                   & $\ts{MonadMaybe            }$ & The maybe monad \\
                   & $\ts{MonadState            }$ & The state monad \\
Natural numbers    & $\ts{Nat                   }$ & Natural numbers from the Zeno test suite \\
                   & $\ts{Nat2ndArg             }$ & Induction on the second argument \\
                   & $\ts{NatAcc                }$ & Accumulator definitions\\
                   & $\ts{NatDouble             }$ & Depth two definitions \\
                   & $\ts{NatDoubleSlow         }$ & Depth two, recursion in one depth \\
                   & $\ts{NatStrict             }$ & Strict in both arguments \\
                   & $\ts{NatSwap               }$ & Arguments swapped in the recursive call \\
Other Work         & $\ts{IWC                   }$ & Inductive challenge problems\footnote{http://www.csc.liv.ac.uk/~lad/research/challenges} \\
                   & $\ts{ProductiveUseOfFailure}$ & From \citep{productiveuse} \\
                   & $\ts{ZenoLists             }$ & The list part of Zeno's test suite \\
                   & $\ts{Ordinals              }$ & Ordinals as defined by \cite{dixonphd} \\
Sorting            & $\ts{InsertionSort         }$ & Insertion sort \\
                   & $\ts{MergeSort             }$ & Merge sort from \hs{Data.List} \\
\end{tabular}

\caption{Description of the files in the test suite, by group.}
\label{tbl:testsuite}
}
\end{table}

\section{Setup}

This section describes the setup for the exhaustive run of the test
suite.  Five theorem provers were used: Vampire 0.6, E 1.0-004 Temi,
prover9 2009-02A \citep{prover9}, SPASS 3.5 and equinox
6.0.1alpha. Each prover invocation had an own thread on a 2.40GHz
Intel Xeon CPU. All properties were tested for each theorem prover,
and also one run which tried every invocation with all provers. The
time limit for each invocation was 5 seconds.

To make this tractable, some restrictions were needed. The maximum depth
and induction variables for structural induction was 2, and to prevent
combinatorial explosion the number of induction steps was limited to
20, and a maximum of 500 hypotheses were generated. For fixed point
induction there is a similar problem since there are many possible
candidate functions, so the maximum number of fixed functions were 100
for each property.

\section{Results for Provers and Proof Techniques}

In this section we present how well the different provers succeeded on
the big run of the test suite. It is important to distinguish between
the value domains the different techniques work for. Successes for
methods which handle all values are called Theorems, and for methods
that only work on finite and total values are called Finite Theorems.

To compare the provers and the different proof methods on this test
suite, the number of proved properties for each prover and proof
method is illustrated in Table \ref{tbl:proved}. The entries in the
row captioned Theorem shows how many theorems were proved out of the
total. The following four columns for definitional equality,
structural induction, approximation lemma and fixpoint induction tells
how many properties out of those could be proved with this method. The
following two columns are similar, though for finite theorems where the
only applicable method is structural induction.

\begin{table}[h]
\centering
%\documentclass{article}
%\usepackage{fullpage}
%\usepackage{courier}

%\usepackage{amssymb}
%\usepackage{longtable}
%
%\begin{document}
%
%\title{Results}
%\maketitle

%\section*{Total}
\begin{tabular}{>{\small}l || >{\small}r@{/}>{\small}l | >{\small}r@{/}>{\small}l | >{\small}r@{/}>{\small}l | >{\small}r@{/}>{\small}l | >{\small}r@{/}>{\small}l || >{\small}r@{/}>{\small}l | >{\small}r@{/}>{\small}l}
Prover & \multicolumn{2}{>{\small}l|}{Theorem} & \multicolumn{2}{>{\small}l|}{defn.eq.} & \multicolumn{2}{>{\small}l|}{induction} & \multicolumn{2}{>{\small}l|}{approx} & \multicolumn{2}{>{\small}l||}{fixpoint} & \multicolumn{2}{>{\small}l|}{Finite Thm.} & \multicolumn{2}{>{\small}l}{induction} \\
\hline
E           & 203&540 & 68&203 & 120&203 & 136&203 & 23&203 &  99&540 &  99&99\\
prover9     & 194&540 & 69&194 & 114&194 &  92&194 &  8&194 &  93&540 &  93&93\\
SPASS       & 198&540 & 65&198 & 122&198 & 111&198 & 21&198 &  94&540 &  94&94\\
Vampire     & 208&540 & 71&208 & 122&208 & 133&208 & 25&208 & 104&540 & 104&104\\
equinox     &  78&540 & 15&78  &  57&78  &  56&78  &  3&78  &  39&540 &  39&39\\
\hline
all provers & 214&540 & 74&214 & 124&214 & 145&214 & 26&214 & 111&540 & 111&111\\
\hline
\end{tabular}

%\end{document}

\caption{Number of proved properties per prover and proof method.
         Only the Theorem is counted for properties proved as both
         Theorems and Finite Theorems.
\label{tbl:proved}
}
\end{table}

\paragraph{Discussion}

In general, all provers tested in this experiment but equinox
performed almost equally well on this problem set. For those four in
the top the differences are small, but the most successful is
Vampire. It is closely followed by E, which performs slighly better on
the approximation lemma. One of the few notable differences is that
prover9 performs worse proofs by the approximation lemma and fixed
point induction.

A striking observation is that the approximation lemma is the most
successful of proving techniques for Theorems. It is also the easiest
to formulate and implement. These good results for approximation
lemma motivates the search for a reformulation of that holds for
finite or total infinite values.

Structural induction could become even stronger if the combinatorial
explosion when generating induction hypotheses could be
restricted. Currently, it creates all smaller trees than the induction
conclusion, but another alternative is to use all subtrees.

Definitional equality, written as plain in the table, is only
applicable for properties with only abstract arguments. There are some
notable files in the test suite with these, \ts{Functions}, \ts{MonadEnv} and
\ts{MonadState}, and most successes come from there. Should the translation
be changed so that it also applicable on concrete arguments, it would
also be able to prove properties from for example \ts{Bool}.


\section{Results per File}

To compare the different kinds of properties from the files of the
test suite the results per file can be viewed in Table
\ref{tbl:allstats}.

\begin{table}[h]
%\centering
%\documentclass{article}
%\usepackage{fullpage}
%\usepackage{courier}
%\usepackage{amssymb}
%\usepackage{longtable}
%
%\begin{document}
%
%\title{Results}
%\maketitle
%
%\section*{ home dan remote run 4442 evpsx statistics.data}
\begin{tabular}{>{\footnotesize}l || >{\footnotesize}c | >{\footnotesize}c | >{\footnotesize}c | >{\footnotesize}c | >{\footnotesize}c || >{\footnotesize}c | >{\footnotesize}c}
File                       & Thm & defn.eq. & ind & approx & fixpoint & Fin.Thm. & ind \\
\hline
$\ts{Bool                    }$   & 15/23 & 2/15 & 13/15 & 15/15 &  & 8/23 & 8/8\\
$\ts{Expr                    }$   & 4/8 &  & 4/4 & 3/4 & 1/4 &  & \\
$\ts{Fix                     }$   & 7/7 &  & 7/7 & 1/7 &  &  & \\
$\ts{Functions               }$   & 15/19 & 15/15 &  &  &  & 4/19 & 4/4\\
$\ts{IWC                     }$   & 2/7 &  & 2/2 & 1/2 & 2/2 &  & \\
$\ts{Infinite                }$   & 11/15 &  & 5/11 & 10/11 & 9/11 &  & \\
$\ts{InsertionSort           }$   &  &  &  &  &  & 1/12 & 1/1\\
$\ts{Integers                }$   & 7/14 &  & 7/7 & 6/7 &  & 6/14 & 6/6\\
$\ts{MergeSort               }$   &  &  &  &  &  &  & \\
$\ts{MonadEnv                }$   & 18/18 & 18/18 &  &  &  &  & \\
$\ts{MonadMaybe              }$   & 10/10 & 2/10 & 8/10 & 10/10 &  &  & \\
$\ts{MonadState              }$   & 33/48 & 25/33 &  & 31/33 &  &  & \\
$\ts{Nat                     }$   & 11/33 &  & 11/11 & 6/11 & 3/11 & 15/33 & 15/15\\
$\ts{Nat2ndArg               }$   & 4/18 &  & 4/4 & 4/4 & 1/4 & 5/18 & 5/5\\
$\ts{NatAcc                  }$   & 2/18 &  & 2/2 & 2/2 &  & 1/18 & 1/1\\
$\ts{NatDouble               }$   & 2/18 &  & 2/2 & 2/2 &  & 8/18 & 8/8\\
$\ts{NatDoubleSlow           }$   & 2/18 &  & 2/2 & 2/2 &  & 8/18 & 8/8\\
$\ts{NatStrict               }$   & 4/18 &  & 4/4 & 3/4 &  & 6/18 & 6/6\\
$\ts{NatSwap                 }$   & 4/18 &  & 4/4 & 3/4 &  & 1/18 & 1/1\\
$\ts{Ordinals                }$   & 2/6 &  & 1/2 & 1/2 & 1/2 &  & \\
$\ts{PatternMatching         }$   & 9/15 & 4/9 & 5/9 & 7/9 &  & 6/15 & 6/6\\
$\ts{ProductiveUseOfFailure  }$   & 6/85 &  & 6/6 & 6/6 & 6/6 & 21/85 & 21/21\\
$\ts{Queues                  }$   & 14/23 & 7/14 & 7/14 & 9/14 &  & 2/23 & 2/2\\
$\ts{Reverse                 }$   & 9/17 & 1/9 & 8/9 & 9/9 & 1/9 & 1/17 & 1/1\\
$\ts{Sequences               }$   &  &  &  &  &  &  & \\
$\ts{Streams                 }$   & 1/3 &  &  & 1/1 &  &  & \\
$\ts{Tricky                  }$   &  &  &  &  &  & 4/8 & 4/4\\
$\ts{ZenoLists               }$   & 22/54 &  & 22/22 & 13/22 & 2/22 & 14/54 & 14/14\\
\hline
Total                      & 214/540 & 74/214 & 124/214 & 145/214 & 26/214 & 111/540 & 111/111\\
\end{tabular}
%\end{document}

\caption{Number of proved properties per file and proof method, using
  all provers. An empty cell means nothing proved.
\label{tbl:allstats}
}
\end{table}

\paragraph{Discussion}

These results shreds some light over the success of approximation
lemma: it is applicable for non recursive structures. That enables it
to prove properties from \ts{Bool} and \ts{MonadMaybe}, as well as on
the tuple result in \ts{MonadState}. On the other hand, fixed point
induction is only for recursive functions. It performs well on the
tests suited for it, notably the Infinite file, but also surprisingly
on the Theorems from \ts{ProductiveUseOfFailure} which are only
contains properties about lists and natural numbers.

The two files \ts{Nat} and \ts{ZenoLists} are taken from the Zeno test
suite \citep{zeno}. Zeno accomplishes to prove all but two properties,
and the situation is a bit different here. If the theorems and finite
theorems are summed, the result is 26/33 for \ts{Nat} and 36/54 for
\ts{ZenoLists}. The explanation is simple, our tool neither use nor
discover lemmas, whereas Zeno can. A similar result can be found in
the test suite file \ts{ProductiveUseOfFailure} from the paper with
the same name \citep{productiveuse}. The tests from that paper
includes the lemmas discovered by their tool, and most of the theorems
we can prove are those lemmas. Lemmas are crucial in many files,
examples are \ts{InsertionSort} and \ts{Reverse}. It is also needed to
prove more sophisticated properties not widely included in this test
suite.

\section{Proving Time}

Most invocations to the theorem provers are solved in a few
milliseconds, and few can take up to 500 ms. SPASS had the largest
success time of all invocation: about 1500ms, but that was the only success
over one second. The cumulative amount of success times can be viewed
in Figure \ref{fig:provingtime}. Each method can give rise to
several different ways of proving a property, an example is fixed
point induction on different functions. These are called parts. A part
can consist of one or many invocations to a theorem prover, an example
is the base and the different step cases in a proof by structural
induction. These invocations are called particles. All particles from
a part that ended in success are counted in the figure.

\paragraph{Discussion}

The theorem provers are well suited for these kinds of problems, with
some variance already discussed. This motivates using automated
theorem provers to prove properties about Haskell programs. For the
specific setup, 5 seconds were used as timeout, but one secound should
be sufficient, possibly less.

\begin{figure}[h]
\centering
\begin{tikzpicture}
\begin{axis}[xlabel={time (ms)},ylabel=quantity,
             enlargelimits=false,ymax=3000,ymin=0,
             width=10cm,
             xmin=0,xmax=500]
\addplot [draw=black] table[x=time,y=qty]  {resultdata/theoremTimes.data} ;
\end{axis}
\end{tikzpicture}
\caption{Cumulative amount of successes over time, using all provers
  and methods. Only successes from a method resulting in a theorem are counted.
\label{fig:provingtime}
}
\end{figure}



