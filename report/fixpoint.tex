\section{Fixed point induction}
\label{sec:fixpoint}

Induction is applicable when arguments are of a concrete type, such as
lists or natural numbers. There are also properties where all
arguments are of abstract types. The canonical example is the
map-iterate property:

\begin{equation*}
\faa{(f : a \rightarrow a)}{(x : a)} \hs{map} \w f \w (\hs{iterate} \w f \w x) \eq
           \hs{iterate} \w f \w (f \w x)
\end{equation*}

Here $f$ is an abstract function $a \rightarrow a$, and $x$ is
something of type $a$. This example is further investigated in Section
\ref{sec:mapiter} below, but it is already clear that we cannot proceed to
prove this with structural induction.

Enter fixed point induction. It allows a way of performing induction
on the recursive structure of the program. In short, if the property
regards a function $f$, the hypothesis is that the property holds for all
the recursive calls in the definiton of $f$, and the goal is to prove
that it holds for $f$.

\begin{comment}
It
is an early example of a technique from domain theory, attributed to
Scott and de Bakker,
\note{Citation needed: there is a book called
  Mathematical Theory of Program Correctness by Jaco de Bakker that
  could be appropriate if found}
and sometimes called Scott induction
or computational induction.  \cite{domains}
\end{comment}

The least fixed point for a function can be found in Haskell with the
function \hs{fix}, which can simply be defined as:

\begin{code}
fix :: (a -> a) -> a
fix f = f (fix f)
\end{code}

This function solves the equation $x = f \w x$, since substituting $x$
for $\hs{fix} \w x$: the left side evaluates to $f \w (\hs{fix} \w f)$
in one step, which is then equal to the right side. This is the origin
of the name of the combinator \hs{fix}: this is a fixed point of the
equation.  Any self-recursive function can be rewritten in terms of
\hs{fix}. Recall the \hs{Prelude} function \hs{repeat}, which makes an
infinite list of the same element \hs{repeat x = x : repeat}. Its
transformation to \hs{fix}-style is this:

\begin{code}
repeat x = fix r
  where r i = x : i
\end{code}

Computing \hs{repeat x}, we get the following unfolds:
\begin{equation*}
  \hs{repeat x}
= \hs{fix r}
= \hs{x:fix r}
= \hs{x:x:fix r}
= \hs{x:x:x:fix r}
  \cdots
\end{equation*}
So \hs{fix (x:)} is the infinite list of \hs{x}. The translation of a
self-recursive function to be defined in terms of \hs{fix} is
mechanical. Assume $f$ is defined with arguments $\overline{x}$ and
has a body $e$ that uses both itself and its arguments, let us write
this as $e(f,\overline{x})$. Then the translation is this:

\begin{equation*}
f \w \overline{x} \eq e(f,\overline{x})
\w \Leftrightarrow \w
f \eq \hs{fix} \w (\lambda \w f' \w \overline{x} \w \rightarrow \w e(f',\overline{x}))
\end{equation*}

Another more domain theoretic approach is to say that
$\hs{fix} \w f \eq \lub{n}(f^n \bot)$, where $f^n \bot$ is $n$ applications of $f$:
\begin{equation*}
f^n \bot \eq \underbrace{f (f (\cdots (f}_{n \w \mathrm{copies \w of} \w f}} \bot) \cdots))
\end{equation*}
This corresponds to a potentially infinite, countable unrolling of $f$.
It is easy to verify that $\langle f^n \bot\rangle_{n\in\omega}$ is a
$\sqsubseteq$-chain by induction on $n$, and that this is the least
pre-fixed point of $f$ is also showed by induction: Assume there
is another pre-fixed point $\theta$, thus satisfying
$\theta \eq f \w \theta$. The base case is
$\bot \eq f^0 \bot \sqsubseteq \theta$, trivially satisified since
$\bot$ is the least element. For the step case, assume that
$f^n \bot \sqsubseteq \theta$, and we get the conclusion
$f^{n+1} \bot = f (f^n \bot) \sqsubseteq f \w \theta = \theta$ as desired.
Fixpoint induction proves properties about a function written in terms
of \hs{fix}, and its inference rule is this:

\begin{mathpar}
  \inferrule*
     {
       P(\bot)
       \\
       P(x) \rightarrow P(f \w x)
       \\
       P \w \mathrm{admissible}
     }
     { P(\fn{fix} f) }
\end{mathpar}


Here it is important that $P$ is \emph{admissible} , meaning that for
all $\sqsubseteq$-chains of length $\omega$, if the property holds for
all elements in the chain it must necessary hold for its limit, futher
described in Section \ref{sec:admissible}, where a proof is given that
the properties we are interested in, universally quantified equalities
of continuous functions, are admissible predicates.

An interesting property of fixed point induction is that it does not
care about types: indeed, it works in an untyped setting. In addition,
it can exploit strange recursive structures of the function. A caveat
is that it can only prove properties that must hold for infinite and
partial values.

The proof that fixed point induction relies on the fact that
$\lub{n}(f^n \w \bot) \eq \hs{fix} \, f$, where $f^n$ is $n$
self-applications of $f$. This is true since \hs{fix} is defined as
$f$ self-applied to it self. Apart from this, the proof only uses
induction over natural numbers and that $f^0 \w \bot \eq \bot$, and
it is of course important that $P$ is admissible. See proof below:

\begin{align*}
P(\bot) & \wedge \fa{x} P(x) \rightarrow P(f x) \\
\desclra{$f^0 \w \bot \eq \bot$} \\
P(f^0 \w \bot) & \wedge \fa{x} P(x) \rightarrow P(f x) \\
\descra{quantifying} \\
P(f^0 \w \bot) & \wedge \fa{n} P(f^n \w \bot) \rightarrow P(f^{n+1} \w \bot) \\
\desclra{induction} \\
\fa{n} & P(f^n \w \bot) \\
\desclra{\textit{P} \w admissible} \\
& P(\lub{n}(f^n \w \bot)) \\
\desclra{definition \w of \w \hs{fix}} \\
& P(\hs{fix} \w f) \\
\end{align*}

One reason to introduce fixed point induction is to avoid the natural
numbers in $\fa{n} P(f^n \bot)$  to prove $P(\hs{fix} \w f)$.

\subsection{Example: map-iterate}
\label{sec:mapiter}

For properties that do not have any arguments with a concrete type,
structural induction is not applicable. The Haskell function
\hs{iterate} is a that makes an infinite list from a seed, by repeated
application of a function, i.e \hs{iterate f x} is the list
 \hs{x:f x:f (f x):}$\cdots$. It is related to Haskell function
 \hs{map} in the map-iterate property, stated as follows:

\begin{equation*}
\faa{f}{x} \hs{map} \w f \w (\hs{iterate} \w f \w x) \eq
           \hs{iterate} \w f \w (f \w x)
\end{equation*}

\noindent
With their standard definitions given in \ref{code:mapiterate} below.

\begin{figure}[h!]
\centering
\begin{minipage}[b]{6cm}
\begin{code}
map :: (a -> b) -> [a] -> [b]
map f (x:xs) = f x : map f xs
map f [] = []
\end{code}
\end{minipage}
\hspace{10pt}
\begin{minipage}[b]{6cm}
\begin{code}[mathescape]
iterate :: (a -> a) -> a -> [a]
iterate f x = x : iterate f (f x)
$\newline$
\end{code}
\end{minipage}
\caption{Definition of \texttt{map} and \texttt{iterate}
\label{code:mapiterate}
}
\end{figure}

The behavior of \hs{map} is to apply a function to every element of a
list. We see that we cannot use structural induction here, since both
$f$ and $x$ are abstract, but this can be proved by fixpoint induction
on \hs{iterate}. First, we rewrite this function in terms of \hs{fix}:

\begin{code}
iterate = fix iter
iter i f x = x : i f (f x)
\end{code}

The predicate $P$ from fixpoint induction is $P(g) \w \Leftrightarrow
\w \faa{f}{x} \hs{map} \w f \w (g \w f \w x) \eq g \w f \w (f \w x) $. If we
prove the base case and step case we can then conclude
$P(\hs{fix iter})$, and that is by definition $P(\hs{iterate})$.

The base case is $P(\bot)$. Since \hs{map} is strict in its second
argument, it is both the left side and right side evaluate to $\bot$.
The for the step case we have to show
$P(\hs{i}) \rightarrow P(\hs{iter i})$. We start from the induction
hypothesis and work towards the goal as follows:

\begin{align*}
\w \faa{f}{x} \hs{map} \w f \w (\hs{i} \w f \w x) & \eq \hs{i} \w f \w (f \w x) \\
\descra{generalizing $x$ to $f \w x$} \\
\w \faa{f}{x} \hs{map} \w f \w (\hs{i} \w f \w (f \w x)) & \eq \hs{i} \w f \w (f \w (f \w x)) \\
\descra{substitutivity} \\
\w \faa{f}{x} f \w x \hs{:} \hs{map} \w f \w (\hs{i} \w f \w (f \w x)) & \eq f \w x \hs{:} \hs{i} \w f \w (f \w (f \w x)) \\
\desclra{\defof{\texttt{map}}} \\
\w \faa{f}{x} \hs{map} \w f \w (x \hs{:} \hs{i} \w f \w (f \w x)) & \eq f \w x \hs{:} \hs{i} \w f \w (f \w (f \w x)) \\
\desclra{\defof{\texttt{iter}}} \\
\w \faa{f}{x} \hs{map} \w f \w (\hs{iter} \w \hs{i} \w f \w x) & \eq \hs{iter} \w \hs{i} \w f \w (f \w x) \\
\end{align*}

As discussed earlier, the $P$ used is admissible since it is an
universally quantified equality. Hence, fixpoint induction gives us the
\hs{map}-\hs{iterate} property.


\begin{comment}
To illustrate why it is important that the property $P$ is admissible,
we shall consider an example
consider the predicate P to be “is not infinite”, and then you can
prove for a lot of functions that they return finite objects. For
instance, define this function:

CHANGE THIS

  Instead do the finite list predicate, and use $\neq
  \hs{False}$. This then servers for an example why inequality
  predicates are not admissible!

\begin{code}
listrec :: ([a] -> [a]) -> [a] -> [a]
listrec i [] = []
listrec i (x:xs) = x : i xs
\end{code}

Then define
$P(f) \Leftrightarrow \fa{x} ``f(x) \w \mathrm{is \w not \w infinite}"$,
and proceed to prove $P(\hs{fix listrec})$ by fixed point induction. The
base case $P(\bot)$ succeeds, since $\bot$ is not infinite, and if we
assume $P(\hs{i})$, we have no problem proving $P(\hs{listrec i})$.
Hence $P(\hs{fix listrec})$, and since \hs{fix listrec} is essentially
a linear identity function on lists, we have ``proved'' that all lists
are finite (but possibly partial).

The error is as promised that $P$ is not admissible: for the sequence
\begin{equation*}
\bot \sqsubseteq
\hs{0:}\bot \sqsubseteq
\hs{0:1:}\bot \sqsubseteq
\hs{0:1:2:}\bot \sqsubseteq
\cdots
\end{equation*}
$P$ holds for all elements but $P$ does not hold for its limit \hs{[0..]}.

\end{comment}

\begin{comment}
\subsection{Mutually Recursive Functions}

You can also mechanically transform mutually recursive functions to be
defined in terms of \hs{fix}. The functions \hs{even} and \hs{odd}
defined below, which determines if a \hs{Nat} is even, and odd,
respectively, are straightforwardly written by mutual recursion:

\begin{code}
even :: Nat -> Bool           odd :: Nat -> Bool
even Z     = True             odd Z     = False
even (S x) = odd x            odd (S x) = even x
\end{code}

To write these functions in terms of fix, as an additional argument,
the take a tuple of ``non-recursive'' copies of themselves.

\begin{code}
evenToFix :: (Nat -> Bool,Nat -> Bool) -> Nat -> Bool
evenToFix (evenUnFix,oddUnFix) Z     = True
evenToFix (evenUnFix,oddUnFix) (S x) = oddUnFix x

oddToFix :: (Nat -> Bool,Nat -> Bool) -> Nat -> Bool
oddToFix (evenUnFix,oddUnFix) Z     = True
oddToFix (evenUnFix,oddUnFix) (S x) = evenUnFix x
\end{code}

Here the prefix \hs{ToFix} means that it is a function subject to be
\hs{fix}-ed, and \hs{UnFix} means that it is the ``non-recursive''
function. The functions above can now be \hs{fix}-ed by giving the
tuple as an argument to both of them:

\begin{code}
even',odd' :: Nat -> Bool
(even',odd') = fix (\t -> (evenToFix t,oddToFix t))
\end{code}

This encoding makes \hs{even'} denotationally equal to \hs{even} and
the same relation hols for \hs{odd'} and \hs{odd}.
\end{comment}

\subsection{Simplification}

The mechanical translations introduced above for self-recursive
functions and mutually recursive functions makes a new function with
an additional argument, the ``non-recursive'' version of itself. By
the translation to FOL that is used, this would introduce a new
argument as a ``function pointer'' and introduce uses of $\appfn$,
which gives unnecessary overhead to the automated theorem
provers.

This is another approach. It avoids introducing these function
pointers and the additional argument to every function. Given a
function $f$ with arguments $\overline{x}$ defined as this:

\begin{equation*}
f \, \overline{x} = e(\overline{x},f)
\end{equation*}

Two new constants are introduced, $\tofix{f}$ and $\unfix{f}$
and this definition:

\begin{equation*}
\tofix{f} \, \overline{x} = e(\overline{x},\unfix{f})
\end{equation*}

\noindent
The empty circle $\unfix{}$ describes that this function is empty
(lacks implementation,) and the filled circle $\tofix{}$ means that
this function has an implementation.

Now we can get a simplified fixpoint schema:

\begin{mathpar}
  \inferrule*
     {
       P(\bot)
       \\
       P(\unfix{f}) \rightarrow P(\tofix{f})
       \\
       P \, \mathrm{admissible}
     }
     { P(f) }
\end{mathpar}

\noindent
The empty $\unfix{f}$ does not have any implementation. But it has
something much better, namely the induction hypothesis. The induction
conclusion is to prove the property for $\tofix{f}$, where the
recursive call to $f$ is replaced with $\unfix{f}$. We do this
simplification since it is better suited for the theorem provers.

\newpage
This also works for several functions at the same time, possibly
mutually recursive:

\begin{mathpar}
  \inferrule*
     {
       P(\bot,\bot)
       \\
       P(\unfix{f},\unfix{g}) \rightarrow P(\tofix{f},\tofix{g})
       \\
       P \, \mathrm{admissible}
     }
     { P(f,g) }
\end{mathpar}

This translation needs to be carried out with some care, since for $f
\, \overline{x} = e(\overline{x},f)$, it is also possible that $f$ is
called in bodies of other functions. These are of two kinds: either
this function is also called from $f$, making it recursive, or another
function which is not called from $f$, but makes use of $f$
anyway. The first example, with a recursive call, the body needs to be
edited so $f$ becomes translated (to $\bot$, $\unfix{f}$ or
$\tofix{f}$), and the second case should use the original $f$. The
transitive clousure of the call graph is calculated, and every
appropriate calls of $f$ are replaced.

\subsection{Erroneous Use of Fixed Point Induction}

The importance that the predicate $P$ is admissible is illustrated in
this example. The function \hs{finite} below returns \hs{True} on
finite lists and $\bot$ on partial lists.

\begin{code}
finite :: [a] -> Bool
finite []     = True
finite (x:xs) = finite xs
\end{code}

Is it possible to use \hs{finite} to prove that \emph{all} lists in
Haskell are either finite or partial? Let the predicate be
$P(f) \Leftrightarrow \fa{xs} f \w xs \neq \hs{False}$, and proceed by fixed
point induction. This is the proof plan with the predicate inlined:

\begin{mathpar}
  \inferrule*
     {
       \fa{xs} \bot \w xs \neq \hs{False}
       \\
       \fa{xs}             \unfix{\hs{finite}} \w xs \neq \hs{False}
               \rightarrow \tofix{\hs{finite}} \w xs \neq \hs{False}
     }
     { \fa{xs} \hs{finite} \w xs \neq \hs{False} }
\end{mathpar}

\noindent
The base case is trivial, as anything applied to bottom is $\bot$ and
$\bot \neq \hs{False}$. The step case also succeeds; trivially if $xs$
is the empty list, by the hypothesis if $xs$ is non-empty, and if $xs$
is bottom for similar reasons as in the base case.

The error is as promised that $P$ is not admissible. See the
$\sqsubseteq$-chain below:
\begin{equation*}
\bot \sqsubseteq
\hs{0:}\bot \sqsubseteq
\hs{0:1:}\bot \sqsubseteq
\hs{0:1:2:}\bot \sqsubseteq
\cdots
\end{equation*}
$P$ holds for all elements in the chain, but $P$ does not hold for its
limit \hs{[0..]}. This example also serves as a proof that inequality
in general is not admissible.


\subsection{Candidate Selection}

Faced with the following property saying that if you drop $n$ elements
from a list the length of this is the same as the length of the
original list minus $n$, which functions should one do fixed point
induction on?

\begin{verbatim}
prop_length_drop :: [a] -> Nat -> Prop Nat
prop_length_drop xs n = length (drop n xs) =:= length xs - n
\end{verbatim}

The answer here is to do fixed point induction on \hs{drop}, and on
\hs{-}. So far no better way to tackle this is used than to try fixed
point induction on all subsets of recursive functions mentioned in the
property.

\subsection{Future Work}

Just as with structural induction, it is also possible to use fixed
point in more than one ``depth'', giving for instance this inference
rule:

\begin{mathpar}
  \inferrule*
     {
       P(\bot)
       \\
       P(f \w \bot)
       \\
       P(x) \wedge P(f \w x) \rightarrow P(f \w (f \w x))
       \\
       P \w \mathrm{admissible}
     }
     { P(\fn{fix} f) }
\end{mathpar}

It is also possible to use such an encoding as in ``Automated depth''
in Section \ref{sec:futind} to let the theorem prover determine the
depth. As an example, the map-iterate property impossible to show with
\hs{map} redefined to \hs{doublemap}, defined below, with ordinary one
depth fixed point induction.

\begin{verbatim}
doublemap :: (a -> b) -> [a] -> [b]
doublemap f []       = []
doublemap f [x]      = [f x]
doublemap f (x:y:zs) = f x : f y : doublemap f zs
\end{verbatim}

\noindent
While \hs{doublemap} is behaviorally equivalent to \hs{map} on total
lists, it makes the induction hypothesis in fixed point induction too
weak.


An issue with the candidate selection is that is some selections are
quite stupid, for instance doing fixating functions on only one side
of the equality. A heuristic to find good candidates would be beneficial.