\chapter{Technical Part}

Implementation technicalities.

\section{Translation to FOL}

The goal of this translation is to enable equational reasoning of
Haskell programs, in FOL. It is here referred to as a translation, but
it could also in be regarded as compilation.

Equational reasoning is traditional in proving corrected of Haskell
programs, but it assumes that a simple denotational semantics exists,
and there is not even a formal semantics for the language
\cite{chasingbot}.

% Whoa, much in this section
\subsection{Data types}

Haskell's data types and its consequence pattern matching is
translated by giving each constructor a function or constant symbol.

Consider this examples of binary trees in Haskell style:

\begin{verbatim}
data Tree a = Empty | Branch (Tree a) a (Tree a)
\end{verbatim}

We get one constant $\fn{Empty}$ for \hs{Empty}, and one trinary
function $\fn{Branch}$ for \hs{Branch}. Normally, it is custom
to write functions in logic with lowercase letters, but this
convention is disregarded here. The values of these constructors are
distinct, so the following disjointedness axioms are added to the
theory of a Haskell program with this data type.

\begin{align*}
\fn{\bot} &  \neq \fn{Empty}\\
 \forall \, l \,  x \,  r \,  . \,  \fn{\bot} &  \neq \fn{Branch}(l,x,r)\\
 \forall \, l \,  x \,  r \,  . \,  \fn{Empty} &  \neq \fn{Branch}(l,x,r)\\
\end{align*}

Notice that in Haskell, each data type also has an extra value, bottom, which
can come from the functions \hs{undefined} or \hs{error}, as well as
irrefutable pattern matches but also non-terminating programs. The constant
$\bot$ captures this notion and is naturally also distinct from the values
of \hs{Empty} and \hs{Branch}.

\subsection{Injective constructors and projections}

We also want injectivity of constructors, for example if we have the
cons constructor, \hs{:} in Haskell, and \hs{x:xs = y:ys} then
\hs{x = y} and \hs{xs = ys}. As we will see later, we also want the
projections of the left and right sub tree, and the value in a Branch.
It turns out that the projections imply the injectivity. For the Branch
constructor of the Tree example, we get the following projections:

\begin{align*}
 \forall \, l \, x \, r \, . \,  \fn{Branch_{0}}(\fn{Branch}(l,x,r)) &  = l\\
 \forall \, l \, x \, r \, . \,  \fn{Branch_{1}}(\fn{Branch}(l,x,r)) &  = x\\
 \forall \, l \, x \, r \, . \,  \fn{Branch_{2}}(\fn{Branch}(l,x,r)) &  = r\\
\end{align*}

These ternary functions are bluntly named by indexing on the projected
coordinate, but could just as well have more descriptive names as
$\fn{left}$, $\fn{value}$ and $\fn{right}$.

Now, these projections imply injective constructors. Assume we have
$\fn{Branch}(l,x,r) = \fn{Branch}(l',x',r')$ then the first
projection, $\fn{Branch_0}$, gives us that $l=l'$. Analogously,
and the second and the third give $x=x'$ and $r=r'$, respectively.

\subsection{Translation of functions}

Translating functions that do not use pattern matching is
straightforward. Consider this Haskell definition of a singleton tree:

\begin{verbatim}
leaf :: a -> Tree a
leaf x = Branch Empty x Empty
\end{verbatim}

We simply introduce a new function symbol, $\fn{leaf}$, and due
to referential transparency we can turn the definition into an
equality with quantified variables:

\begin{align*}
 \forall \, x \, . \, \fn{leaf}(x) &  = \fn{Branch}(\fn{Empty},x,\fn{Empty})\\
\end{align*}

\subsection{Pattern matching}

Let's now consider \hs{leaf}'s partial inverse, \hs{top} that
yields the top element of the tree if there is one, or is undefined
otherwise:

\begin{verbatim}
top :: Tree a -> a
top (Branch _ x _) = x
\end{verbatim}

The function call of \hs{top Empty} would yield a run time error
since this pattern is not covered, equivalent to an undefined
value. Indeed, an equivalent formulation would be to cover the
\hs{Empty} case by \hs{undefined} or a helpful message from
\hs{error}, but the run time error is still remains, and this
behavior is modeled by the $\bot$ value in the theory. Thus the
translation to an equality is as follows: if the argument is
constructed with \hs{Branch}, it is equivalent to the top value,
otherwise, it is $\bot$:

\begin{align*}
 \forall \, l \, x \, r \, & . \, \fn{top}(\fn{Branch}(l,x,r) = x\\
 \forall \, t \, & . \, t \neq
p \fn{Branch}(\fn{Branch_{0}}(t),\fn{Branch_{1}}(t),\fn{Branch_{2}}(t))
 \rightarrow \, \fn{top}(t)  = \fn{\bot}\\
\end{align*}

Here the projections functions come in handy. Indeed, an equivalent
but skolemized formulations of the second formulas are:

\begin{align*}
 \forall \, t \, & . \neg (\exists \, l \, x \, r . \, t =
 \fn{Branch}(l,x,r))
 \rightarrow \, \fn{top}(t) = \bot \\
 \forall \, t \, & . (\forall \, l \, x \, r . \, t \neq
 \fn{Branch}(l,x,r))
 \rightarrow \, \fn{top}(t) = \bot \\
\end{align*}

But as we saw earlier, projections also imply injectivity so this is
the approach used here but the choice is of little importance (or is
it? benchmark!!)

\subsection{Overlapping patterns}

Overlapping patterns need to be removed, otherwise we could easily get
an inconsistent theory, consider

\begin{verbatim}
overlap :: Bool -> Bool
overlap True = True
overlap True = False
\end{verbatim}

Certainly, we cannot translate this to
\begin{align*}
\fn{overlap}(\fn{True}) & = \fn{True} \\
\fn{overlap}(\fn{True}) & = \fn{False} \\
\forall \, b \, . \, b \neq True & \rightarrow \fn{overlap}(b) = \bot \\
\end{align*}

Transitivity of equality then yields $\fn{True} = \fn{False}$,
and this together with the axioms of disjoint constructors gives a
contradiction.

In Haskell, pattern matching is done from top to bottom of the
definition, making the second match of True to never occur. Thus, the
translation to FOL also removes all subsequent patterns that are
instances of any pattern above.

\subsection{Nested patterns and bottoms}

The translation also handles patterns in more than one depth. At every
location in a pattern where a constructor is matched against, a
pattern with bottom at that spot is also added, defined to
bottom. This definition that balances a tree to the left is defined
with pattern matching on depth two:

\begin{verbatim}
unbalance :: Tree a -> Tree a
unbalance (Branch (Branch l x r) y r') = unbalance (Branch l x (Branch r y r'))
unbalance (Branch l x r)               = Branch l x (unbalance r)
unbalance Empty                        = Empty
\end{verbatim}

If we could see the bottoms in Haskell, the definition would look like this:

\begin{verbatim}
unbalance :: Tree a -> Tree a
unbalance (Branch (Branch l x r) y r') = unbalance (Branch l x (Branch r y r'))
unbalance (Branch Bottom _ _)          = Bottom
unbalance (Branch l x r)               = Branch l x (unbalance r)
unbalance Empty                        = Empty
unbalance Bottom                       = Bottom
\end{verbatim}

And such an addition of bottoms is made by the translation.
\note{This writing actually discovered a bug. Report the solution and how it was tested}

\subsection{Guards}

Guards are not much of a complication. Either the guard expression is
\hs{True}, then that branch is picked. If the expression returns
bottom, then for this argument, the function is bottom. Care needs to
be taken when looking ``upwards'' the branches, to not collide with
the guards.
\note{Add example}

\subsection{Functions as arguments}

In Haskell, functions readily take other functions as arguments, and
functions can also be partially applied. To get the same behavior in
logic, each function gets a \emph{function pointer}, and a new binary
function is added to the language, written infix with $\fn{@}$.
\note{Should $\fn{@}$ be written infix? $\fn{app}$ prefix is also a viable option}
For instance, the if there is a binary function plus then a constant
called plus-ptr is added to the theory and this axiom:

\begin{equation*}
\forall \, x \, y \, . \, (\fn{plus.ptr} \, @ \,  x) \, @ \,  y = \fn{plus}(x,y)
\end{equation*}

When a function is only partially applied, or a function argument is
applied, $\, @ \, $ is used. Consider the Prelude function \hs{iterate}

\begin{verbatim}
iterate :: (a -> a) -> a -> [a]
iterate f x = x : iterate f (f x)
\end{verbatim}

It is translated with $\, @ \, $ in the following way, with \hs{:} written infix:

\begin{equation*}
\forall \, f \, x \, . \, \fn{iterate}(f,x) = x : \fn{iterate}(f,f \, @ \,  x)
\end{equation*}

Should a function not get all its arguments, appropriate use of $\, @ \, $ is
added, as in this function which increments all elements of the list
by one using \hs{map}:

\begin{verbatim}
incr = map (plus one)
\end{verbatim}

As incr is also written point-free or eta-reduced, \hs{map} is also
partially applied. This is the translated axiom:

\begin{equation*}
\fn{incr} = \fn{map.ptr} \, @ \,  (\fn{plus.ptr} \, @ \,  \fn{one})
\end{equation*}

If \hs{incr} is applied to an argument $xs$, then \hs{incr} is
applied to more arguments then it takes, so we add $\, @ \,$ so the
corresponding formula becomes $\fn{incr} \, @ \, xs$, and by
equational substitution from the definition of incr we get
$(\fn{map.ptr} \, @ \, (\fn{plus.ptr} \, @ \, \fn{one}))
\, @ \, xs$ and the axiom of $\fn{map.ptr}$ then equals this to
$\fn{map}(\fn{plus.ptr} \, @ \, \fn{one},xs)$.

\subsection{The intermediate language}

Everything becomes top level definitions, \hs{let} and \hs{where} are
floated to the top, with the free variables added as extra parameters,
as well as lambda functions, the same approach was taken for extra or
nested case-expressions. Extra care needs to be taken for mutually
recursive \hs{let} and \hs{where} functions, which potentially need's
each other's free variables as arguments.
\note{Incomplete, add examples}

Functions do not do any pattern matching on their arguments directly,
but in a case statement that is the entire body of the function. The
branch expressions are just constructor or function application.
\note{Add BNF}

Type-signatures and data types are also supported since this
information is needed for different proof techniques.

\subsection{Haskell coverage}

Haskell is a big language, and translating it all in one go is a
daunting task. Therefore, some restrictions were settled to be able to
focus on proving rather than translating.
\note{Maybe move this to background?}
The goal was to add enough of the Haskell language to enable to prove
interesting properties, but much of the widely available sugar in
Haskell was omitted since it does not add extra
expressibility. Therefore there is no support for
list comprehensions, do-notation, pattern bindings and type classes .

A more serious omitted feature is the lack of built-in types like \hs{Int},
\hs{Integer}, \hs{Double}, \hs{Char}, etc.

(Higher-kinded type variables are currently not supported)

\subsection{Uncategorized}

\begin{itemize}

\item $\checkmark$ Describe (and motivate here?) the intermediate language

\item $\checkmark$ Pattern-matching and bottoms

\item $\checkmark$ Higher order functions and function pointers

\item $\checkmark$ Axioms of disjointedness

\item $\checkmark$ Axioms of projections and injectivity of constructors

\item Extensional equality and application of bottom

\item Remove unnecessary definitions for a given proof

\end{itemize}

\section{Proof techniques}

Describe how properties are entered in the code, and that they are
Quick Check testable

\subsection{Definitional Equality}

Only on non-concrete types

Prove monad laws for the environment monad for instance

Mention (the unimplemented \hs{Prop (a -> b)} - \hs{a -> Prop b}
isomorphism and its relation to seq and extensional equality)

\subsection{Structural induction}

Simple induction to more complex structural induction and its
implementation.

\subsection{Fixed point induction}

Describe how and why it works

Fixed point induction on several functions and subsets of the existing
functions

\subsection{Approximation lemma}

After fixed point induction since it is an easy consequence of fixed
point induction, and how we removed the auxiliary structure of natural
numbers. This makes it equivalent to fixed point induction of id on
both sides (or does it?)
