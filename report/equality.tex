\section{Definitional Equality}
\label{sec:equality}

Some properties cannot or need not use induction or some more
sophisticated technique, since they are true by definition. Examples
are properties for fully polymorphic functions such as the definition
of \hs{id} in the SK-calculus, here

\begin{code}
s f g x = f x (g x)
k x y = x
id x = x

prop_skk_id :: Prop (a -> a)
prop_skk_id = s k k =:= id
\end{code}

Then, the generated conjecture is simply

\begin{equation*}
\app{ (\app {\ptr{s}} {\ptr{k}} )
    }{\ptr{k}} = \ptr{id}
\end{equation*}

Another example where this is useful is to prove functor and monad
laws for the environment monad.

\subsection{Extensional Equality and seq}

To prove the previous property we also need to have extensional
equality, postulated with the following axiom

\begin{equation*}
\faa{f}{g} (\fa{x} \app{f}{x} = \app{g}{x}) \rightarrow f = g
\end{equation*}

\noindent
which identifies function pointers and functions composed with $@$.
One problem with extensional equality in Haskell, is that the presence
of \hs{seq} breaks it. \hs{seq} is a built in function with the
following behavior:

\begin{code}[mathescape]
seq :: a -> b -> b
seq $\bot$ y = $\bot$
seq x y = y$, \qquad x \neq
\end{code}

It forces the first argument to weak head normal form and returns the
second. For our purposes, it is only important if the first argument
is $\bot$, then the function also returns this as it is strict in its
first argument. With \hs{seq} it is possibly to distinguish between
these two functions, which otherwise are observationally equal:

\begin{code}[mathescape]
f = $\bot$
g = \x -> $\bot$
\end{code}

Because \hs{seq f ()} evaluates to $\bot$, and \hs{seq g ()} evaluates
to \hs{()}, but on any argument \hs{f} and \hs{g} gets, they both
return $\bot$. Here we also need an extra axiom, which says that
anything applied to $\bot$ is $\bot$:

\begin{equation*}
\fa{x} \app{\bot}{x} = \bot
\end{equation*}

However, \hs{seq} is the only function that can tell such functions
apart, so we will ignore its presence in Haskell.  In the future,
there could be added as a switch \hs{--enable-seq}, which removes
extensional equality.

Furthermore, if we assume we have extensional equality we also have
the property that \hs{Prop (a -> b)} is equivalent to
\hs{a -> Prop b}, by letting the property have an extra argument that
is applied to the left and right hand side of the equality. This has
two benefits. Firstly, it can trigger other proof methods should \hs{a}
or \hs{b} be concrete types (the former for induction and the latter
for approximation lemma). Secondly, it does not need to use the
extensionality axiom introduced above which introduces extra steps in
the proof search.

\subsection{Future Work: Concrete Concerns}

This only works on non-concrete types because of the way bottoms are
added. One example when this is a problem with is this plausible
definition of boolean or, and the property of right identity:

\begin{code}
False || a = a
True  || _ = True

prop_or_right_identity :: Bool -> Prop Bool
prop_or_right_identity x = x || False =:= x
\end{code}

The translation of \hs{||} makes any element in the domain that is not
the introduced constants $\fn{false}$ or $\fn{true}$ for \hs{Bool}'s
constructors, equal $\bot$. Now consider the translation of the
property:

\begin{equation*}
\fa{x} x \, \fn{||} \, \fn{false} = x
\end{equation*}

Now this is false: take a model with another element $\star$ in the
domain:

$$\star \, \fn{||} \, \fn{false} = \bot$$

The consequence of this is that the proof principle of definitional
equality is only used on abstract types, rather than polymorphic, as
they cannot be strict. Do not fear: the property above is trivially
proved by induction. For the \hs{Bool} type and other non recursive
data types, induction degenerate into mere case enumeration.


