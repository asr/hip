\chapter{Haskell to First Order Logic}

To enable automated theorem provers to do equational reasoning of
Haskell programs a translation to first order logic is needed. It is
here referred to as a translation, but it could also be regarded as a
compilation. The idea is to use constants and functions in first order
logic to correspond to constructors and functions, and arguments to
functions need to be universally quantified. We shall try to do a
na\"{\i}ve attempt of a translation with this ideas and see how far it
takes us.

\section{Na\"{\i}ve Translation}

We will use a data type of binary trees with an element at every
branch, and consider some examples of functions defined on it. This
is the Haskell definition of the data type we will be using:

\begin{code}
data Tree a = Fork (Tree a) a (Tree a) | Leaf
\end{code}

\noindent
With the idea above, occurrences of the \hs{Fork} constructor in the
source code should be translated to a logic function $\fn{fork}$, and
similarly a constant for \hs{Leaf}. How should we then translate the
\hs{singleton} function, defined below?

\begin{code}
singleton :: a -> Tree a
singleton x = Fork Leaf x Leaf
\end{code}

\noindent
Following our intuition we make an universal quantification for
\hs{x}, and a new logic function for \hs{singleton}. The result
is this axiom:
\begin{equation*}
\fa{x} \fn{singleton}(x) = \fn{fork}(\fn{leaf},x,\fn{leaf})
\end{equation*}

\noindent
So far so good, but what if someone wants to prove that \hs{singleton x}
is a \hs{Leaf}? With only this axiom in the theory, it would be
possible: there are models with only one element where \hs{Leaf} is
equal to \hs{Fork}. Indeed, we will need to add axioms that values
created from the different constructors are unequal. We will call
those disjoint constructor axioms, and for the \hs{Tree} data type, we
get this axiom:
\begin{equation*}
\faaa{l}{x}{r} \fn{leaf} \neq \fn{fork}(l,x,r)
\end{equation*}

Constructors should also be injective to get regular models, and
adding such axioms are straightforward. Since only \hs{Fork} has
arguments, this injectivity axiom is needed:
\begin{equation*}
\faaaaaa{l_0}{l_1}{x_0}{x_1}{r_0}{r_1} \fn{fork}(l_0,x_0,r_0) \eq
\fn{fork}(l_1,x_1,r_1) \rightarrow l_0 \eq l_1 \wedge x_0 \eq x_1 \wedge r_0 \eq r_1
\end{equation*}

For the \hs{mirror} function, which recursively mirrors the left sub
tree with the right and vice-versa, we follow our intuition to
translate the pattern matching and get these two axioms\footnote
{Axioms are enumerated by Roman numerals to tell them apart.}:

\begin{code}
mirror :: Tree a -> Tree a
mirror (Fork l x r) = Fork (mirror r) x (mirror l)
mirror Leaf         = Leaf
\end{code}
\begin{align*}
\rom{1} && \faaa{l}{x}{r} & \fn{mirror}(\fn{fork}(l,x,r)) \eq \fn{fork}(\fn{mirror}(r),x,\fn{mirror}(l)) \\
\rom{2} &&                & \fn{mirror}(\fn{leaf}) \eq \fn{leaf}
\end{align*}

\noindent
A problem with this translation is that there are no axioms for other
arguments of $\fn{mirror}$ than leafs and forks, and we have models that
include other values than leafs and forks. Another problem is
encountered for \hs{singleton}'s left inverse, \hs{top}, code below,
which returns the top element of a \hs{Tree}. This is a partial
function since it has no pattern for the \hs{Leaf} constructor.

\begin{code}
top :: Tree a -> a
top (Fork l x r) = x
\end{code}

The translation must capture the pattern match failure that results
from trying to evaluate top applied to a leaf. We conclude that
this na\"{\i}ve translation does not take us further, but we shall see
in the next section how to fix this.

\section{Bottom and Pattern Matching}

In domain theory there is a concept of bottom, denoted $\bot$. It is
used for the least defined value: pattern match failures, use of
\hs{error} and \hs{undefined} in the source code, but also for
non-terminating programs.  \note{Refer to a short introduction of
  domain theory at the end of this chapter (or elsewhere, appendix?)}
For \hs{top} the idea is to add an axiom so that $\fn{top}$ of
anything that is not a \hs{Fork} is bottom. This is an example of such
an axiomatization:

\begin{align*}
\rom{1} \qquad & \faaa{l}{x}{r} \fn{top}(\fn{fork}(l,x,r)) \eq x \\
\rom{2} \qquad & \fa{t}         (\nexxx{l}{x}{r} \fn{fork}(l,x,r)) \eq t) \rightarrow \fn{top}(t) \eq \bot
\end{align*}

Most theorem provers would as a preprocessing step \note{citation
  needed}skolemize the existential quantification in the second
axiom. A new unary function would be introduced for $l$, $x$ and $r$,
depending on $i$, an arbitrary choice of names are $\fn{top}$ appended
to the original variable. The axiom then looks like
this\footnote{Lambda functions bind as far to the right as possible,
  and this thesis uses the same convention for quantifiers.}:
\begin{align*}
\rom{2}' \qquad & \fa{t} \fn{fork}(\fn{topl}(t),\fn{topx}(t),\fn{topr}(t))) \neq t \rightarrow \fn{top}(t) \eq \bot
\end{align*}

For another function, like \hs{mirror} above, one of the skolemized
functions could be called $\fn{mirrorl}$. Since axioms of injective
constructors are also added, a theorem prover could, in some steps,
conclude that $\faaa{l}{x}{r} \fn{mirrorl}(\fn{fork}(l,x,r)) \eq
\fn{topl}(\fn{fork}(l,x,r)) \eq l$. But what happens if we introduce
such skolemized ``selector'' functions for every constructor manually?
For the \hs{Fork} constructor call them $\fn{fork_0}$, $\fn{fork_1}$
and $\fn{fork_2}$, and their axioms are:
\begin{align*}
\rom{1} \qquad \faaa{l}{x}{r} \fn{fork_{0}}(\fn{fork}(l,x,r)) & \eq l\\
\rom{2} \qquad \faaa{l}{x}{r} \fn{fork_{1}}(\fn{fork}(l,x,r)) & \eq x\\
\rom{3} \qquad \faaa{l}{x}{r} \fn{fork_{2}}(\fn{fork}(l,x,r)) & \eq r
\end{align*}

\noindent
The translation of \hs{top} with these selector functions is:
\begin{align*}
\rom{1} \qquad & \faaa{l}{x}{r} \fn{top}(\fn{fork}(l,x,r)) \eq x \\
\rom{2} \qquad & \fa{t}         (\fn{fork}(\fn{fork_0}(t),\fn{fork_1}(t),\fn{fork_2}(t))) \neq t) \rightarrow \fn{top}(t) \eq \bot
\end{align*}

\noindent
Another nice side effect of writing in this skolemized selector style
is that implies injective constructors. Assume we have
$\fn{fork}(l_0,x_0,r_0)=\fn{fork}(l_1,x_1,r_1)$ then the first
projection, $\fn{fork_0}$, gives us that $l_0=l_1$. Analogously, and
the second and the third give $x_0=x_1$ and $r_0=r_1$,
respectively. Thus selector axioms are added in place of of
injectivity axioms.

With the bottom constant in the theory, the axioms disjointedness are
effected by this. It can be seen as an implicit constructor for every
data type. For the \hs{Tree} data type the axioms are:

\begin{align*}
\rom{1} \qquad & \faaa{l}{x}{r} \fn{fork}(l,x,r) \neq \fn{leaf} \\
\rom{2} \qquad & \faaa{l}{x}{r} \fn{fork}(l,x,r) \neq \bot      \\
\rom{3} \qquad & \bot \neq \fn{leaf}
\end{align*}

Now we have a good idea how to translate pattern matching, but
in Haskell we can pattern match almost everywhere! How would we
proceed to translate a function like this, taken from the
implementation of \hs{scanr} from the \hs{Prelude}?

\begin{code}
scanr             :: (a -> b -> b) -> b -> [a] -> [b]
scanr f q0 []     =  [q0]
scanr f q0 (x:xs) =  f x q : qs
                     where qs = scanr f q0 xs
                           q = case qs of
                                 q : _ -> q
\end{code}

\noindent
There is both pattern matching on the direct arguments, but also
pattern matching in a case statements in the where function
\hs{q}. There can also be pattern matching in lambdas. To help with
these difficulties, we define an intermediate language in the next
section.

\section{The Intermediate Language}

To address the difficulties of pattern matching elsewhere than the
arguments of a function, a small intermediate language was designed
that can only do pattern matching at a very controlled location: in a
case statement that is the entire body of a function, and all arms are
just simple expressions consisting of function and constructor
applications and variables. Haskell is translated to this, and pattern
matching at other locations is translated to this in a new top level
definition. Functions definined in let and where are raised to the top
level, with the necessary variables in scope as additional
arguments. The same is done for sections and lambda functions. The BNF
for the language is this:

\begin{equation*}
\begin{aligned}
\text{Variables} \quad & x \\
\text{Functions} \quad & f \\
\text{Constructors} \quad & C \\
\text{Type variables} \quad & \tau \\
\text{Type constructors} \quad & T \\
\defBNF{Declarations}{decl}{ f \; \overline{x} \; \hs{=} \; body}{function declaration} \\
    \defaltBNF{f \; :: \; t}{type signature} \\
    \defaltBNF{\hs{data} \; T \; \overline{\tau} \; \hs{=} \; \overline{C \; \overline{t}}}{data type declaration} \\
\defBNF{Function body}{body}{\hs{case} \; e \; \hs{of} \; \overline{alt}}{case body} \\
    \defaltBNF{e}{expression body} \\
\defBNF{Expressions}{e}{x}{variable} \\
    \defaltBNF{f \; \overline{e}}{function application} \\
    \defaltBNF{C \; \overline{e}}{constructor application} \\
\defBNF{Alternative}{alt}{pat \rightarrow e}{branch without guard} \\
    \defaltBNF{pat \; \hs{|} \; e \rightarrow e}{branch with guard} \\
\defBNF{Pattern}{p}{x}{pattern variable} \\
    \defaltBNF{C \; \overline{p}}{constructor pattern} \\
\defBNF{Types}{t}{\tau}{type variable} \\
    \defaltBNF{t \; \rightarrow \; t}{function type} \\
    \defaltBNF{T \; \overline{\tau}}{type constructor application} \\
\defBNF{Programs}{prog}{\overline{decl}}{} \\
\end{aligned}
\end{equation*}

This language is a strict subset of Haskell. Repeated entities are
notated with an $\overline{\text{overline}}$.  Data declarations are
added for disjointedness and selector axioms, and type signatures are
just skipped in the translation, but the proof techniques introduced
later use this information.

A function is just a function name with a number of variables, and
then a function body, which is either an expression of variables,
functions and constructors, or a case statements with an expression
scrutinee. Branches consists of a pattern, possibly with nested uses
of constructors, and an optional guard, and in the arm is an
expression.

Now we need to distinguish between two translations: the intermediate
translation from Haskell to the intermediate language, and the logic
translation from this language to first order logic.

\section{The Intermediate Translation}

After this section, we will only concentrate on the logic translation.

\paragraph{Argument pattern matching} A function that does pattern matching will be translated to one that
takes in unmatched arguments and with a case in the body. The
\hs{mirror} function above is thus translated to this:

\begin{code}
mirror :: Tree a -> Tree a
mirror t = case t of
   Fork l x r -> Fork (mirror r) x (mirror l)
   Leaf       -> Leaf
\end{code}

\noindent
If you do pattern matching on several arguments, the scrutinee in the
case will be a tuple of all the arguments.

\paragraph{Local definitions} Where-clauses and let-expressions are
raised to the top level, with all necessary variables as
arguments. This example of an accumulator definition of multiplication
of Peano natural numbers needs such a rewrite:

\begin{code}
(*) :: Nat -> Nat -> Nat
x * y = go Zero x
  where
    go acc Zero    = acc
    go acc (Suc n) = go (acc + y) n
\end{code}

\noindent
The \hs{go} function has the \hs{y} in scope but not as argument so it
is appended to the arguments to the top level lifted version of \hs{go}:

\begin{code}
go acc Zero    y = acc
go acc (Suc n) y = go (acc + y) n y

x * y = go Zero x y
\end{code}

\noindent
Finally \hs{go} is translated using a case expression:

\begin{code}
go acc x y = case x of
     Zero  -> acc
     Suc n -> go (acc + y) n y
\end{code}

A similar translation is done for let expressions.

\paragraph{Lambda functions} These are translated to top level
definitions. Take this example of defining \hs{fmap} in terms of the
functions from the \hs{Monad} type class:

\begin{code}
fmap' :: Monad m => (a -> b) -> m a -> m b
fmap' f m = m >>= \x -> return (f x)
\end{code}

\noindent
In the lambda, \hs{f} is a free variable so it becomes an argument to
the new top level function called \hs{lambda} below:

\begin{code}
lambda f x = return (f x)

fmap' :: Monad m => (a -> b) -> m a -> m b
fmap' f m = m >>= lambda f
\end{code}

A similar translation is done for sections.

\section{Pattern Matching Revisited}

\paragraph{Overlapping patterns} First of all, overlapping patterns need to be removed, otherwise we
easily get an inconsistent theory, consider

\begin{code}
overlap :: Bool -> Bool
overlap b = case b of
              True -> True
              True -> False
\end{code}

Certainly, this cannot be translated to:
\begin{align*}
\rom{1} \qquad & \fn{overlap}(\fn{true}) = \fn{true} \\
\rom{2} \qquad & \fn{overlap}(\fn{true}) = \fn{false} \\
\rom{3} \qquad & \fa{b} b \neq \fn{true} \rightarrow \fn{overlap}(b) = \bot
\end{align*}

Starting from the immediate truth $\fn{overlap}(\fn{true}) =
\fn{overlap}(\fn{true})$, transitivity of the equalities in the axioms
$\romnodot{1}$ and $\romnodot{2}$ gives the equality $\fn{true} =
\fn{false}$. This together with the axiom from disjoint constructors,
$\fn{true} \neq \fn{false}$, gives a contradiction.

In Haskell, pattern matching is done from top to bottom of the
definition, making the second match of \hs{True} to never occur. Thus,
the translation to FOL also removes all subsequent patterns that are
instances of any pattern above.

\paragraph{Nested patterns and bottoms} The translation also handles
patterns in more than one depth. At every location in a pattern where
a constructor is matched against, a pattern with bottom at that spot
is also added, defined to bottom. This is a definition for a balancing
function for Red-Black trees\footnote{From
  http://www.cs.kent.ac.uk/people/staff/smk/redblack/TypedExist.hs}:

\begin{code}
balance :: RR a [b] -> a -> RR a [b] -> Red Black a b
balance (R a) y (R b) = R(B a,y,B b)
balance (C a) x b     = balanceR a x b
balance a     x (C b) = balanceL a x b
\end{code}

\noindent
For the sake of readability it is not presented as a case expression
pattern matching on a tuple, thought this is what the intermediate
translation would transform it to.

For each matched constructor, we need to add a new match to bottom,
which evaluates to bottom. Unnecessary bottoms can be carelessly added
since overlapping patterns are removed \emph{afterwards}. Furthermore,
a wild pattern is added at the end that goes to bottom in case there
are other constructors for the data type not mentioned in the
patterns.

No type information is needed to do this: it is merely an
inspection. Could the bottoms be seen in this Haskell definition it
would be this after the insertion of bottoms and removal of
overlapping patterns:

\begin{code}[mathescape]
balance :: RR a [b] -> a -> RR a [b] -> Red Black a b
balance (R a) y (R b) = R(B a,y,B b)
balance $\bot$      _ _     = $\bot$
balance _     _ $\bot$      = $\bot$
balance (C a) x b     = balanceR a x b
balance a     x (C b) = balanceL a x b
balance _     _ _     = $\bot$
\end{code}

Haskell's behavior of matching patterns from top to bottom is
justified with implications ensuring the \emph{upward agreement}.


\paragraph{Guards} Guards are not much of a complication. Either the guard expression is
\hs{True}, then that branch is picked. If the expression returns
bottom, then for this argument, the function is bottom. Care needs to
be taken when looking ``upwards'' the branches, to not collide with
the guards.
\note{Add example}

\section{Functions as Arguments}

In Haskell, functions readily take other functions as arguments, and
functions can also be partially applied. To get the same behavior in
logic, each function gets a \emph{function pointer}, and a new binary
function is added to the language, written infix with $\appfn$.  For
instance if there is a binary function \hs{plus} then a constant
called $\fn{plus.ptr}$ is added to the theory and this axiom:

\begin{equation*}
\faa{x}{y}  \app{(\app{\fn{plus.ptr}}{x})}{y} = \fn{plus}(x,y)
\end{equation*}

When a function is only partially applied, or a function argument is
applied, $\appfn$ is used. Consider this Prelude function \hs{iterate}:

\begin{code}
iterate :: (a -> a) -> a -> [a]
iterate f x = x : iterate f (f x)
\end{code}

It is translated with $\appfn$ in the following way, with the cons
constructor \hs{:} written infix:

\begin{equation*}
\forall \, f \, x \, . \, \fn{iterate}(f,x) = x : \fn{iterate}(f,\app{f}{x})
\end{equation*}

Should a function not get all its arguments, appropriate use of $\, @ \, $ is
added, as in this function which increments all elements of the list
by one using \hs{map}:

\begin{code}
incr = map (plus one)
\end{code}

As \hs{incr} is written $\eta$-reduced, \hs{map} is
only partially applied, this is the translated axiom:

\begin{equation*}
\fn{incr} = \app{\fn{map.ptr}}{(\app{\fn{plus.ptr}}{\fn{one}})}
\end{equation*}

If \hs{incr} is applied to an argument $xs$, then \hs{incr} is applied
to more arguments than it takes, so we add $\appfn$ so the
corresponding formula becomes $\app{\fn{incr}}{xs}$, and by equational
substitution from the definition of $\fn{incr}$ we get
$\app{(\app{\fn{map.ptr}}{(\app{\fn{plus.ptr}}{\fn{one}})})}{xs}$ and
the axiom of $\fn{map.ptr}$ then equals this to
$\fn{map}(\app{\fn{plus.ptr}}{\fn{one}},xs)$.

\section{Summary}

We summarize the translation in Table~\ref{tab:transtable}.

\begin{table}[h]
  \centering
  \begin{tabular}{|l|l|}
    \hline
    Haskell                    & First Order Logic \\
    \hline
    function                   & function or constant \\
    constructor                & function or constant \\
    data type                  & disjoint constructors and selector axioms \\
    pattern matching           & overlap removal, bottoms insertion, upward agreement \\
    guards                     & equality to true and bottom and upward agreement \\
    partial application        & $\appfn$ on pointer constant \\
    partially applied function & pointer constant and $\appfn$ rule \\
    sections, lambdas, let     & new functions with variables in scope as arguments \\
    \hline
  \end{tabular}
  \caption{Translation of different Haskell constructs
    \label{tab:transtable}
  }
\end{table}

%Equational reasaoning is traditional in proving corrected of Haskell
%programs, but it assumes that a simple denotational semantics exists,
%and there is not even a formal semantics for the language
%\cite{chasingbot}.
%

% Remove unnecessary definitions for a given proof

\section{Domain Theory}

This section explains some key concepts in domain theory: monotonicity
and continutity. It shows some nice partial orderings for some
domains. It also gives an example how monotonicity can be used to
check the translation for bugs, and explains why it is problematic to
do for contiunity.

\section{Future Work}

Haskell is a big language, and translating it all in one go is a
daunting task. Therefore, some restrictions were settled to be able to
focus on proving rather than translating.  \note{Maybe move this to
  background?}  The goal was to add enough of the Haskell language to
enable to prove interesting properties, but much of the widely
available sugar in Haskell was omitted since it does not add extra
expressibility. Therefore there is no support for list comprehensions,
do-notation, pattern bindings and type classes .

A more serious omitted feature is the lack of built-in types like \hs{Int},
\hs{Integer}, \hs{Double}, \hs{Char}, etc.

Higher-kinded type variables are currently not supported.

There are still parts of Haskell that are not supported. List
comprehensions and do-notation can be added by its rewriting rules.
\hs{Type} definitions should be unrolled , so they could be used in
the signature for properties. Type classes is probably the most
interesting thing to add, and an approach would be to use dictionary
passing, and inline for concrete types. However, more type information
would be needed but it is possible that much of it could be extracted
from example GHC.

Syntactic features for controlling lazy and strict evaluation like
irrefutable patterns, \hs{seq} and bang patterns, and richer pattern
matching in form of pattern bindings are discussed below, but it
should be noted that it is already possible to prove a lot of
interesting Haskell properties, it is far from able to prove things
about bigger Haskell projects which usually use a richer part of the
language.

\subsection{Irrefutable Patterns and Pattern Bindings}

Irrefutable patterns can be defined in terms of strict projections,
not unlike those that already exist (\hs{fst}, \hs{snd}, \hs{head},
\hs{fromJust}, and so on.) Each irrefutable pattern is translated to
a constant, and when you use the variables in the pattern, you
translate it to appropriate use of strict projections. One example is
the translation of the \hs{uncurry} function:

\begin{code}
uncurry f ~(x,y) = f x y    <=>  uncurry f t = f (fst t) (snd t)
\end{code}

\noindent
The irrefutable pattern \verb:~(x,y): is replaced with the new constant
\hs{t}, and in the body of the function, \hs{x} is replaced with the
strict projection \hs{fst t}, and similarly for \hs{y}.

Top level patterns, more specifically called pattern bindings, can
also be written in terms of such strict projections. The whole pattern
is replaced with a constant, and when the variables from the pattern
are used, you again replace it with strict projections. This is how it
could look for a simple \hs{fromJust . lookup} implementation:

\begin{code}
unsafeLookup x xs = v           <=>   unsafeLookup x xs = fromJust t
  where Just v = lookup x xs            where t = lookup x xs
\end{code}

The strict projections would not rely on the user having \hs{fst} or
\hs{fromJust} in scope, they can automatically be generated by
inspection of the data type definition.

\subsection{Bang Patterns and seq}

The encoding for bang patterns and \hs{seq} is also straightforward,
say you want to define seq with bang patterns, you would have

\begin{code}
seq :: a -> b -> b
seq !x y = y
\end{code}

The axioms needs to ensure that if \hs{x} evaluates to $\bot$, then
\hs{seq x} also evaluates to $\bot$. The two axioms for this functions are:
\begin{align*}
\rom{1} \qquad & \fa{y}    seq(\bot,y) \eq \bot \\
\rom{2} \qquad & \faa{x}{y} x \neq \bot \rightarrow seq(x,y) \eq y
\end{align*}

Either you implement bang patterns in this fashion, or you do the same
translation as GHC for bang patterns: for each strict variable, you
add a \hs{seq} for that variable for the expression of that pattern,
and you simply add the axioms for \hs{seq} to the theory if the
program uses it or bang patterns. This also works for data types with
strictness fields.

\subsection{Pattern Guards}

Patterns guards are a GHC specific extension to Haskell, that allows
you to pattern match against values returned from a function in a
guard. An example is this elaboration of the \hs{lookup} function from
the \hs{Prelude}, which applies a function to the found element:

\begin{code}
transformLookup :: k -> [(k,v)] -> (k -> v -> b) -> Maybe b
transformLookup k xs f | Just v <- lookup k xs = Just (f k v)
                       | otherwise             = Nothing
\end{code}

\noindent
If the lookup returns \hs{Just}, you already have the value \hs{v}
bound and can use it in the expression for this pattern. This is not
at all unlike the normal guards, they are a special case of pattern
guards: the guard \hs{f x | p x} is expressed as
\hs{f x | True <- p x}. The translation of guards currently checks if
\hs{p x} is \hs{True}, and then ``picks'' this branch, or $\bot$ and
then ``returns'' $\bot$. You could just as well do this for other
constructors, you just need to add bottoms in the guard branches just
as is currently done for ordinary patterns.

