\documentclass{report}
\usepackage[utf8]{inputenc}
\usepackage{amssymb}
\usepackage{amsmath}
\usepackage{graphicx}
\usepackage{mathpartir}
\usepackage[margin=1.5in]{geometry}
\usepackage{parsetree}
\usepackage{verbatim}
\usepackage{subfig}
\usepackage{wrapfig}
\usepackage{tipa}
\usepackage{textcomp}
\usepackage{float}
%\usepackage{courier}

\newcommand\note[1]{\mbox{}\marginpar{\footnotesize\raggedright\hspace{0pt}\emph{#1}}}
\newcommand\hs[1]{\verb~#1~}
\newcommand\fn[1]{\mathrm{#1}}
\newcommand\ptr[1]{\fn{#1.ptr}}
\newcommand\app[2]{#1 \, @ \, #2}
\newcommand\fa[1]{\forall \, #1 \, . \,}
\newcommand\faa[2]{\forall \, #1 \, #2 \, . \,}
\newcommand\faaa[3]{\forall \, #1 \, #2 \, #3 \, . \,}

\newcommand\tofix[1]{#1_{\mathrm{tofix}}}
\newcommand\unfix[1]{#1_{\mathrm{unfix}}}
\newcommand\comment[1]{}

\newcommand{\xsys}[2]{#1 \, xs \, #2 & = #1 \, ys #2}
\newcommand{\desca}[1]{  & \hspace{44.5mm}                            \{ \mathrm{#1} \}}
\newcommand{\descra}[1]{ & \hspace{35mm} \Rightarrow     \hspace{4mm} \{ \mathrm{#1} \}}
\newcommand{\descla}[1]{ & \hspace{35mm} \Leftarrow      \hspace{4mm} \{ \mathrm{#1} \}}
\newcommand{\desclra}[1]{& \hspace{35mm} \Leftrightarrow \hspace{4mm} \{ \mathrm{#1} \}}

\newcommand\lub[1]{\sqcup_{#1}}
\newcommand\defof[1]{definition \, of \, #1}

\newcommand\w[0]{\,\,}
\newcommand\eq[0]{=}

\begin{comment}
\begin{document}

\chapter{Proof Methods}

\noindent
\section{Fixed point induction}
\label{sec:fixpoint}

Induction is applicable when arguments are of a concrete type, such as
lists or natural numbers. There are also properties where all
arguments are of abstract types. The canonical example is the
map-iterate property:

\begin{equation*}
\faa{(f : a \rightarrow a)}{(x : a)} \hs{map} \w f \w (\hs{iterate} \w f \w x) \eq
           \hs{iterate} \w f \w (f \w x)
\end{equation*}

Here $f$ is an abstract function $a \rightarrow a$, and $x$ is
something of type $a$. This example is further investigated in Section
\ref{sec:mapiter} below, but it is already clear that we cannot proceed to
prove this with structural induction.

Enter fixed point induction. It allows a way of performing induction
on the recursive structure of the program. In short, if the property
regards a function $f$, the hypothesis is that the property holds for all
the recursive calls in the definiton of $f$, and the goal is to prove
that it holds for $f$.

\begin{comment}
It
is an early example of a technique from domain theory, attributed to
Scott and de Bakker,
\note{Citation needed: there is a book called
  Mathematical Theory of Program Correctness by Jaco de Bakker that
  could be appropriate if found}
and sometimes called Scott induction
or computational induction.  \cite{domains}
\end{comment}

The least fixed point for a function can be found in Haskell with the
function \hs{fix}, which can simply be defined as:

\begin{code}
fix :: (a -> a) -> a
fix f = f (fix f)
\end{code}

This function solves the equation $x = f \w x$, since substituting $x$
for $\hs{fix} \w x$: the left side evaluates to $f \w (\hs{fix} \w f)$
in one step, which is then equal to the right side. This is the origin
of the name of the combinator \hs{fix}: this is a fixed point of the
equation.  Any self-recursive function can be rewritten in terms of
\hs{fix}. Recall the \hs{Prelude} function \hs{repeat}, which makes an
infinite list of the same element \hs{repeat x = x : repeat}. Its
transformation to \hs{fix}-style is this:

\begin{code}
repeat x = fix r
  where r i = x : i
\end{code}

Computing \hs{repeat x}, we get the following unfolds:
\begin{equation*}
  \hs{repeat x}
= \hs{fix r}
= \hs{x:fix r}
= \hs{x:x:fix r}
= \hs{x:x:x:fix r}
  \cdots
\end{equation*}
So \hs{fix (x:)} is the infinite list of \hs{x}. The translation of a
self-recursive function to be defined in terms of \hs{fix} is
mechanical. Assume $f$ is defined with arguments $\overline{x}$ and
has a body $e$ that uses both itself and its arguments, let us write
this as $e(f,\overline{x})$. Then the translation is this:

\begin{equation*}
f \w \overline{x} \eq e(f,\overline{x})
\w \Leftrightarrow \w
f \eq \hs{fix} \w (\lambda \w f' \w \overline{x} \w \rightarrow \w e(f',\overline{x}))
\end{equation*}

Another more domain theoretic approach is to say that
$\hs{fix} \w f \eq \lub{n}(f^n \bot)$, where $f^n \bot$ is $n$ applications of $f$:
\begin{equation*}
f^n \bot \eq \underbrace{f (f (\cdots (f}_{n \w \mathrm{copies \w of} \w f}} \bot) \cdots))
\end{equation*}
This corresponds to a potentially infinite, countable unrolling of $f$.
It is easy to verify that $\langle f^n \bot\rangle_{n\in\omega}$ is a
$\sqsubseteq$-chain by induction on $n$, and that this is the least
pre-fixed point of $f$ is also showed by induction: Assume there
is another pre-fixed point $\theta$, thus satisfying
$\theta \eq f \w \theta$. The base case is
$\bot \eq f^0 \bot \sqsubseteq \theta$, trivially satisified since
$\bot$ is the least element. For the step case, assume that
$f^n \bot \sqsubseteq \theta$, and we get the conclusion
$f^{n+1} \bot = f (f^n \bot) \sqsubseteq f \w \theta = \theta$ as desired.
Fixpoint induction proves properties about a function written in terms
of \hs{fix}, and its inference rule is this:

\begin{mathpar}
  \inferrule*
     {
       P(\bot)
       \\
       P(x) \rightarrow P(f \w x)
       \\
       P \w \mathrm{admissible}
     }
     { P(\fn{fix} f) }
\end{mathpar}


Here it is important that $P$ is \emph{admissible} , meaning that for
all $\sqsubseteq$-chains of length $\omega$, if the property holds for
all elements in the chain it must necessary hold for its limit, futher
described in Section \ref{sec:admissible}, where a proof is given that
the properties we are interested in, universally quantified equalities
of continuous functions, are admissible predicates.

An interesting property of fixed point induction is that it does not
care about types: indeed, it works in an untyped setting. In addition,
it can exploit strange recursive structures of the function. A caveat
is that it can only prove properties that must hold for infinite and
partial values.

The proof that fixed point induction relies on the fact that
$\lub{n}(f^n \w \bot) \eq \hs{fix} \, f$, where $f^n$ is $n$
self-applications of $f$. This is true since \hs{fix} is defined as
$f$ self-applied to it self. Apart from this, the proof only uses
induction over natural numbers and that $f^0 \w \bot \eq \bot$, and
it is of course important that $P$ is admissible. See proof below:

\begin{align*}
P(\bot) & \wedge \fa{x} P(x) \rightarrow P(f x) \\
\desclra{$f^0 \w \bot \eq \bot$} \\
P(f^0 \w \bot) & \wedge \fa{x} P(x) \rightarrow P(f x) \\
\descra{quantifying} \\
P(f^0 \w \bot) & \wedge \fa{n} P(f^n \w \bot) \rightarrow P(f^{n+1} \w \bot) \\
\desclra{induction} \\
\fa{n} & P(f^n \w \bot) \\
\desclra{\textit{P} \w admissible} \\
& P(\lub{n}(f^n \w \bot)) \\
\desclra{definition \w of \w \hs{fix}} \\
& P(\hs{fix} \w f) \\
\end{align*}

One reason to introduce fixed point induction is to avoid the natural
numbers in $\fa{n} P(f^n \bot)$  to prove $P(\hs{fix} \w f)$.

\subsection{Example: map-iterate}
\label{sec:mapiter}

For properties that do not have any arguments with a concrete type,
structural induction is not applicable. The Haskell function
\hs{iterate} is a that makes an infinite list from a seed, by repeated
application of a function, i.e \hs{iterate f x} is the list
 \hs{x:f x:f (f x):}$\cdots$. It is related to Haskell function
 \hs{map} in the map-iterate property, stated as follows:

\begin{equation*}
\faa{f}{x} \hs{map} \w f \w (\hs{iterate} \w f \w x) \eq
           \hs{iterate} \w f \w (f \w x)
\end{equation*}

\noindent
With their standard definitions given in \ref{code:mapiterate} below.

\begin{figure}[h!]
\centering
\begin{minipage}[b]{6cm}
\begin{code}
map :: (a -> b) -> [a] -> [b]
map f (x:xs) = f x : map f xs
map f [] = []
\end{code}
\end{minipage}
\hspace{10pt}
\begin{minipage}[b]{6cm}
\begin{code}[mathescape]
iterate :: (a -> a) -> a -> [a]
iterate f x = x : iterate f (f x)
$\newline$
\end{code}
\end{minipage}
\caption{Definition of \texttt{map} and \texttt{iterate}
\label{code:mapiterate}
}
\end{figure}

The behavior of \hs{map} is to apply a function to every element of a
list. We see that we cannot use structural induction here, since both
$f$ and $x$ are abstract, but this can be proved by fixpoint induction
on \hs{iterate}. First, we rewrite this function in terms of \hs{fix}:

\begin{code}
iterate = fix iter
iter i f x = x : i f (f x)
\end{code}

The predicate $P$ from fixpoint induction is $P(g) \w \Leftrightarrow
\w \faa{f}{x} \hs{map} \w f \w (g \w f \w x) \eq g \w f \w (f \w x) $. If we
prove the base case and step case we can then conclude
$P(\hs{fix iter})$, and that is by definition $P(\hs{iterate})$.

The base case is $P(\bot)$. Since \hs{map} is strict in its second
argument, it is both the left side and right side evaluate to $\bot$.
The for the step case we have to show
$P(\hs{i}) \rightarrow P(\hs{iter i})$. We start from the induction
hypothesis and work towards the goal as follows:

\begin{align*}
\w \faa{f}{x} \hs{map} \w f \w (\hs{i} \w f \w x) & \eq \hs{i} \w f \w (f \w x) \\
\descra{generalizing $x$ to $f \w x$} \\
\w \faa{f}{x} \hs{map} \w f \w (\hs{i} \w f \w (f \w x)) & \eq \hs{i} \w f \w (f \w (f \w x)) \\
\descra{substitutivity} \\
\w \faa{f}{x} f \w x \hs{:} \hs{map} \w f \w (\hs{i} \w f \w (f \w x)) & \eq f \w x \hs{:} \hs{i} \w f \w (f \w (f \w x)) \\
\desclra{\defof{\texttt{map}}} \\
\w \faa{f}{x} \hs{map} \w f \w (x \hs{:} \hs{i} \w f \w (f \w x)) & \eq f \w x \hs{:} \hs{i} \w f \w (f \w (f \w x)) \\
\desclra{\defof{\texttt{iter}}} \\
\w \faa{f}{x} \hs{map} \w f \w (\hs{iter} \w \hs{i} \w f \w x) & \eq \hs{iter} \w \hs{i} \w f \w (f \w x) \\
\end{align*}

As discussed earlier, the $P$ used is admissible since it is an
universally quantified equality. Hence, fixpoint induction gives us the
\hs{map}-\hs{iterate} property.


\begin{comment}
To illustrate why it is important that the property $P$ is admissible,
we shall consider an example
consider the predicate P to be “is not infinite”, and then you can
prove for a lot of functions that they return finite objects. For
instance, define this function:

CHANGE THIS

  Instead do the finite list predicate, and use $\neq
  \hs{False}$. This then servers for an example why inequality
  predicates are not admissible!

\begin{code}
listrec :: ([a] -> [a]) -> [a] -> [a]
listrec i [] = []
listrec i (x:xs) = x : i xs
\end{code}

Then define
$P(f) \Leftrightarrow \fa{x} ``f(x) \w \mathrm{is \w not \w infinite}"$,
and proceed to prove $P(\hs{fix listrec})$ by fixed point induction. The
base case $P(\bot)$ succeeds, since $\bot$ is not infinite, and if we
assume $P(\hs{i})$, we have no problem proving $P(\hs{listrec i})$.
Hence $P(\hs{fix listrec})$, and since \hs{fix listrec} is essentially
a linear identity function on lists, we have ``proved'' that all lists
are finite (but possibly partial).

The error is as promised that $P$ is not admissible: for the sequence
\begin{equation*}
\bot \sqsubseteq
\hs{0:}\bot \sqsubseteq
\hs{0:1:}\bot \sqsubseteq
\hs{0:1:2:}\bot \sqsubseteq
\cdots
\end{equation*}
$P$ holds for all elements but $P$ does not hold for its limit \hs{[0..]}.

\end{comment}

\begin{comment}
\subsection{Mutually Recursive Functions}

You can also mechanically transform mutually recursive functions to be
defined in terms of \hs{fix}. The functions \hs{even} and \hs{odd}
defined below, which determines if a \hs{Nat} is even, and odd,
respectively, are straightforwardly written by mutual recursion:

\begin{code}
even :: Nat -> Bool           odd :: Nat -> Bool
even Z     = True             odd Z     = False
even (S x) = odd x            odd (S x) = even x
\end{code}

To write these functions in terms of fix, as an additional argument,
the take a tuple of ``non-recursive'' copies of themselves.

\begin{code}
evenToFix :: (Nat -> Bool,Nat -> Bool) -> Nat -> Bool
evenToFix (evenUnFix,oddUnFix) Z     = True
evenToFix (evenUnFix,oddUnFix) (S x) = oddUnFix x

oddToFix :: (Nat -> Bool,Nat -> Bool) -> Nat -> Bool
oddToFix (evenUnFix,oddUnFix) Z     = True
oddToFix (evenUnFix,oddUnFix) (S x) = evenUnFix x
\end{code}

Here the prefix \hs{ToFix} means that it is a function subject to be
\hs{fix}-ed, and \hs{UnFix} means that it is the ``non-recursive''
function. The functions above can now be \hs{fix}-ed by giving the
tuple as an argument to both of them:

\begin{code}
even',odd' :: Nat -> Bool
(even',odd') = fix (\t -> (evenToFix t,oddToFix t))
\end{code}

This encoding makes \hs{even'} denotationally equal to \hs{even} and
the same relation hols for \hs{odd'} and \hs{odd}.
\end{comment}

\subsection{Simplification}

The mechanical translations introduced above for self-recursive
functions and mutually recursive functions makes a new function with
an additional argument, the ``non-recursive'' version of itself. By
the translation to FOL that is used, this would introduce a new
argument as a ``function pointer'' and introduce uses of $\appfn$,
which gives unnecessary overhead to the automated theorem
provers.

This is another approach. It avoids introducing these function
pointers and the additional argument to every function. Given a
function $f$ with arguments $\overline{x}$ defined as this:

\begin{equation*}
f \, \overline{x} = e(\overline{x},f)
\end{equation*}

Two new constants are introduced, $\tofix{f}$ and $\unfix{f}$
and this definition:

\begin{equation*}
\tofix{f} \, \overline{x} = e(\overline{x},\unfix{f})
\end{equation*}

\noindent
The empty circle $\unfix{}$ describes that this function is empty
(lacks implementation,) and the filled circle $\tofix{}$ means that
this function has an implementation.

Now we can get a simplified fixpoint schema:

\begin{mathpar}
  \inferrule*
     {
       P(\bot)
       \\
       P(\unfix{f}) \rightarrow P(\tofix{f})
       \\
       P \, \mathrm{admissible}
     }
     { P(f) }
\end{mathpar}

\noindent
The empty $\unfix{f}$ does not have any implementation. But it has
something much better, namely the induction hypothesis. The induction
conclusion is to prove the property for $\tofix{f}$, where the
recursive call to $f$ is replaced with $\unfix{f}$. We do this
simplification since it is better suited for the theorem provers.

\newpage
This also works for several functions at the same time, possibly
mutually recursive:

\begin{mathpar}
  \inferrule*
     {
       P(\bot,\bot)
       \\
       P(\unfix{f},\unfix{g}) \rightarrow P(\tofix{f},\tofix{g})
       \\
       P \, \mathrm{admissible}
     }
     { P(f,g) }
\end{mathpar}

This translation needs to be carried out with some care, since for $f
\, \overline{x} = e(\overline{x},f)$, it is also possible that $f$ is
called in bodies of other functions. These are of two kinds: either
this function is also called from $f$, making it recursive, or another
function which is not called from $f$, but makes use of $f$
anyway. The first example, with a recursive call, the body needs to be
edited so $f$ becomes translated (to $\bot$, $\unfix{f}$ or
$\tofix{f}$), and the second case should use the original $f$. The
transitive clousure of the call graph is calculated, and every
appropriate calls of $f$ are replaced.

\subsection{Erroneous Use of Fixed Point Induction}

The importance that the predicate $P$ is admissible is illustrated in
this example. The function \hs{finite} below returns \hs{True} on
finite lists and $\bot$ on partial lists.

\begin{code}
finite :: [a] -> Bool
finite []     = True
finite (x:xs) = finite xs
\end{code}

Is it possible to use \hs{finite} to prove that \emph{all} lists in
Haskell are either finite or partial? Let the predicate be
$P(f) \Leftrightarrow \fa{xs} f \w xs \neq \hs{False}$, and proceed by fixed
point induction. This is the proof plan with the predicate inlined:

\begin{mathpar}
  \inferrule*
     {
       \fa{xs} \bot \w xs \neq \hs{False}
       \\
       \fa{xs}             \unfix{\hs{finite}} \w xs \neq \hs{False}
               \rightarrow \tofix{\hs{finite}} \w xs \neq \hs{False}
     }
     { \fa{xs} \hs{finite} \w xs \neq \hs{False} }
\end{mathpar}

\noindent
The base case is trivial, as anything applied to bottom is $\bot$ and
$\bot \neq \hs{False}$. The step case also succeeds; trivially if $xs$
is the empty list, by the hypothesis if $xs$ is non-empty, and if $xs$
is bottom for similar reasons as in the base case.

The error is as promised that $P$ is not admissible. See the
$\sqsubseteq$-chain below:
\begin{equation*}
\bot \sqsubseteq
\hs{0:}\bot \sqsubseteq
\hs{0:1:}\bot \sqsubseteq
\hs{0:1:2:}\bot \sqsubseteq
\cdots
\end{equation*}
$P$ holds for all elements in the chain, but $P$ does not hold for its
limit \hs{[0..]}. This example also serves as a proof that inequality
in general is not admissible.


\subsection{Candidate Selection}

Faced with the following property saying that if you drop $n$ elements
from a list the length of this is the same as the length of the
original list minus $n$, which functions should one do fixed point
induction on?

\begin{verbatim}
prop_length_drop :: [a] -> Nat -> Prop Nat
prop_length_drop xs n = length (drop n xs) =:= length xs - n
\end{verbatim}

The answer here is to do fixed point induction on \hs{drop}, and on
\hs{-}. So far no better way to tackle this is used than to try fixed
point induction on all subsets of recursive functions mentioned in the
property.

\subsection{Future Work}

Just as with structural induction, it is also possible to use fixed
point in more than one ``depth'', giving for instance this inference
rule:

\begin{mathpar}
  \inferrule*
     {
       P(\bot)
       \\
       P(f \w \bot)
       \\
       P(x) \wedge P(f \w x) \rightarrow P(f \w (f \w x))
       \\
       P \w \mathrm{admissible}
     }
     { P(\fn{fix} f) }
\end{mathpar}

It is also possible to use such an encoding as in ``Automated depth''
in Section \ref{sec:futind} to let the theorem prover determine the
depth. As an example, the map-iterate property impossible to show with
\hs{map} redefined to \hs{doublemap}, defined below, with ordinary one
depth fixed point induction.

\begin{verbatim}
doublemap :: (a -> b) -> [a] -> [b]
doublemap f []       = []
doublemap f [x]      = [f x]
doublemap f (x:y:zs) = f x : f y : doublemap f zs
\end{verbatim}

\noindent
While \hs{doublemap} is behaviorally equivalent to \hs{map} on total
lists, it makes the induction hypothesis in fixed point induction too
weak.


An issue with the candidate selection is that is some selections are
quite stupid, for instance doing fixating functions on only one side
of the equality. A heuristic to find good candidates would be beneficial.

\bibliographystyle{apalikeurl}
\bibliography{masterbib}

\end{document}
\end{comment}

%\begin{comment}

\begin{document}

%% Title ----------------------------------------------------------------------
\title{Proving Equational Haskell Properties Using ATPs}
\author{Dan Rosén}
\maketitle

%% Abstract -------------------------------------------------------------------

\newpage
\abstract{
In this work we prove a lot of nice equational Haskell properties by
a first order logic translation and using automated theorem provers.
The results are astounding, writing correct Haskell is nowadays a
breeze.
}

%% Acknowledgements -----------------------------------------------------------

\newpage
\pagestyle{empty}
\section*{Acknowledgments}
\vspace{5mm}
ACK, ACK, ACK, ACK, ACK, ACK, NAK

%% Table of Contents ----------------------------------------------------------

\newpage
\tableofcontents
\addtocontents{toc}{\protect\thispagestyle{empty}}

\newpage
\setcounter{page}{1}

\chapter{Introduction}

Induction and Haskell

Infinite values in Haskell are commonplace, but why care about partial
values? They are still present because of non-terminating programs,
calls to the error function and pattern-match
failures \cite{chasingbot}.

\section{First Order Logic}

Connectives, quantifiers, predicates, functions and constants.
Derivation rules and models.

\section{FOL and Automated Theorem Provers}

Use derivations or models to deduce absurdity.

\section{Related Work}

\note{Related work in background or in discussion? Would be nice to
  have it after the technical part to be able to compare different
  approaches.}
Isabel, Sledgehammer, Rippling and other techniques
(Productive Use of Failure)

Other proof assistants as Coq

Agda and dependent types, Agda and ATPs by Peter, Bove, Sicard
\cite{agdaatp} and Dependently Typed Programming Based on ATPs

Zeno

Co-recursive techniques \cite{corecursive}, Hinze's papers, Generic
Approximation lemma, or have this in other chapters

\section{Outline}

Explain the outline of the report

%% Technical Part --------------------------------------------------------------

\chapter{Haskell to First Order Logic}
\label{ch:translation}

To enable automated theorem provers to do equational reasoning of
Haskell programs a translation to first order logic is needed. It is
here referred to as a translation, but it could also be regarded as a
compilation. The idea is to use constants and functions in first order
logic to correspond to constructors and functions, and arguments to
functions need to be universally quantified. We shall try to do a
na\"{\i}ve attempt of a translation with these ideas and see how far it
takes us.

\section{Na\"{\i}ve Translation}
\label{sec:treetrans}

We will use a data type of binary trees with an element at every
branch, and consider some examples of functions defined on it. The
definition of the data type is:

\begin{code}
data Tree a = Fork (Tree a) a (Tree a) | Leaf
\end{code}

\noindent
With the idea above, occurrences of the \hs{Fork} constructor in the
source code should be translated to a logic function $\fn{fork}$, and
similarly a constant for \hs{Leaf}. How should we then translate the
\hs{singleton} function, defined below?

\begin{code}
singleton :: a -> Tree a
singleton x = Fork Leaf x Leaf
\end{code}

\noindent
Following our intuition we make an universal quantification for
\hs{x}, and a new logic function for \hs{singleton}. The result is
this axiom\footnote{Haskell functions and constructors are written in
  \hs{monospace} and their counterpart in first order logic as
  $\fn{this}$. As customary when writing first order logic formulae,
  functions will be written in lowercase, and predicates with an
  initial capital. Hence constructors will be written as $\fn{fork}$
  rather than $\fn{Fork}$. }:
\begin{equation*}
\fa{x} \fn{singleton}(x) = \fn{fork}(\fn{leaf},x,\fn{leaf})
\end{equation*}

\noindent
To capture the intuition that values produced by different
constructors are indeed different, appropriate axioms needs to be
added. Without these, there will be models with only one element where
everything is identified. The axioms added expressing that values
created from different constructors are unequal will be called
\emph{disjoint constructor axioms}.

\pagebreak
\noindent
For the data type \hs{Tree}, one disjoint constructor axiom is
generated for its two constructors:

\begin{equation*}
\faaa{l}{x}{r} \fn{leaf} \neq \fn{fork}(l,x,r)
\end{equation*}

Constructors should also be injective to get regular models, and
expressing such axioms is straightforward. For \hs{Tree}, only \hs{Fork} has
arguments and this injectivity axiom is needed:
\begin{equation*}
\faaaaaa{l_0}{l_1}{x_0}{x_1}{r_0}{r_1} \fn{fork}(l_0,x_0,r_0) \eq
\fn{fork}(l_1,x_1,r_1) \rightarrow l_0 \eq l_1 \wedge x_0 \eq x_1 \wedge r_0 \eq r_1
\end{equation*}

In the next section, injectivity of constructors will a consequence of
another axiom.

To describe pattern matching, consider a \hs{mirror} function, which
recursively swaps the left sub tree with the right and vice-versa. We
follow our intuition to translate the pattern matching to these
two axioms\footnote {Axioms are enumerated by Roman numerals to tell
  them apart.}:

\begin{code}
mirror :: Tree a -> Tree a
mirror (Fork l x r) = Fork (mirror r) x (mirror l)
mirror Leaf         = Leaf
\end{code}
\begin{align*}
\rom{1} && \faaa{l}{x}{r} & \fn{mirror}(\fn{fork}(l,x,r)) \eq \fn{fork}(\fn{mirror}(r),x,\fn{mirror}(l)) \\
\rom{2} &&                & \fn{mirror}(\fn{leaf}) \eq \fn{leaf}
\end{align*}

\noindent
A problem with this translation is that there are no axioms for other
arguments of $\fn{mirror}$ than leafs and forks, and there are models
including other values than leafs and forks. Another problem is
encountered for \hs{singleton}'s left inverse \hs{top} defined below,
which returns the top element of a \hs{Tree}. This function is a partial
since the \hs{Leaf} pattern is omitted:

\begin{code}
top :: Tree a -> a
top (Fork l x r) = x
\end{code}

The translation must capture the pattern match failure that results
from trying to evaluate \hs{top} applied to a \hs{Leaf}. We conclude
that this na\"{\i}ve translation does not take us further, but we
shall see in the next section how to fix these problems.

\section{Bottom and Pattern Matching}

We borrow the concept bottom from domain theory. It is denoted $\bot$
and is the least defined value, and corresponds to pattern match
failures, use of \hs{error} and \hs{undefined} in the source code, and
also non-terminating functions. For \hs{top}, the idea is to add an
axiom so that $\fn{top}$ applied to anything that is not a \hs{Fork}
is bottom. This function is an example of such an
axiomatisation\footnote{This thesis uses the same convention for
  quantifiers as for lambda functions: they bind as far as possible.}:

\begin{align*}
\rom{1} \qquad & \faaa{l}{x}{r} \fn{top}(\fn{fork}(l,x,r)) \eq x \\
\rom{2} \qquad & \fa{t}         (\nexxx{l}{x}{r} \fn{fork}(l,x,r) \eq t) \rightarrow \fn{top}(t) \eq \bot
\end{align*}

Most theorem provers would as a preprocessing step \note{\qquad \qquad citation
  needed}skolemise the existential quantification in the second
axiom. A new unary function would be introduced for $l$, $x$ and $r$,
depending on $t$, an arbitrary choice of names are $\fn{top}$ prepended
to the original variable. The axiom then looks like
this:
\begin{align*}
\rom{2}' \qquad & \fa{t} \fn{fork}(\fn{topl}(t),\fn{topx}(t),\fn{topr}(t))) \neq t \rightarrow \fn{top}(t) \eq \bot
\end{align*}

For another function, like \hs{mirror} above, one of the skolemised
functions could be called $\fn{mirrorl}$. Since axioms of injective
constructors are also added, a theorem prover could, in some steps,
conclude that $\fn{mirrorl}(\fn{fork}(l,x,r)) \eq
\fn{topl}(\fn{fork}(l,x,r)) \eq l$. Instead such skolemised
\emph{selector functions} are introduced manually.  For the \hs{Fork}
constructor let us call them $\fn{fork_0}$, $\fn{fork_1}$ and
$\fn{fork_2}$, and their axioms are:
\begin{align*}
\rom{1} \qquad \faaa{l}{x}{r} \fn{fork_{0}}(\fn{fork}(l,x,r)) & \eq l \\
\rom{2} \qquad \faaa{l}{x}{r} \fn{fork_{1}}(\fn{fork}(l,x,r)) & \eq x \\
\rom{3} \qquad \faaa{l}{x}{r} \fn{fork_{2}}(\fn{fork}(l,x,r)) & \eq r
\end{align*}

\noindent
The translation of \hs{top} with these selector functions is:
\begin{align*}
\rom{1} \qquad & \faaa{l}{x}{r} \fn{top}(\fn{fork}(l,x,r)) \eq x \\
\rom{2} \qquad & \fa{t}         (\fn{fork}(\fn{fork_0}(t),\fn{fork_1}(t),\fn{fork_2}(t)) \neq t) \rightarrow \fn{top}(t) \eq \bot
\end{align*}

\noindent
As a nice side effect, injectivity of constructors is implied the
axioms of the skolemised selector functions. Assume we have
$\fn{fork}(l_0,x_0,r_0)=\fn{fork}(l_1,x_1,r_1)$ then the first
selector, $\fn{fork_0}$, gives $l_0=l_1$. Analogously the second and
the third give $x_0=x_1$ and $r_0=r_1$, respectively. Thus selector
axioms are added in place of the injectivity axioms.

With the bottom constant in the theory, the axioms disjointedness are
effected by this. It can be seen as an implicit constructor for every
data type. For the \hs{Tree} data type the axioms are:

\begin{align*}
\rom{1} \qquad & \faaa{l}{x}{r} \fn{fork}(l,x,r) \neq \fn{leaf} \\
\rom{2} \qquad & \faaa{l}{x}{r} \fn{fork}(l,x,r) \neq \bot      \\
\rom{3} \qquad & \bot \neq \fn{leaf}
\end{align*}

Now we have a good idea how to translate pattern matching, but
in Haskell we can pattern match almost everywhere! How would we
proceed to translate a function like this, taken from the
implementation of \hs{scanr} from the \hs{Prelude}?

\begin{code}
scanr             :: (a -> b -> b) -> b -> [a] -> [b]
scanr f q0 []     =  [q0]
scanr f q0 (x:xs) =  f x q : qs
                     where qs = scanr f q0 xs
                           q = case qs of
                                 q : _ -> q
\end{code}

\noindent
There is both pattern matching directly on the arguments, but also
pattern matching in a case statements in the where function
\hs{q}. There can also be pattern matching in lambdas. To help with
these difficulties, we define an intermediate language in the next
section.

\section{The Intermediate Language}

To address the difficulties of pattern matching elsewhere than in the
arguments of a function, a small intermediate language was designed
that can only do pattern matching at a very controlled location: in a
case statement that is the entire body of a function, and all arms are
simple expressions consisting of function and constructor applications
and variables. As a first step, Haskell is translated to this
language. This process includes several simplifications; pattern
matching at other locations are moved to new top level
definitions. Functions defined in let and where are raised to the
top level, with the necessary variables in scope as additional
arguments. The same is done for sections and lambda functions.

The BNF for the language is this:

\begin{equation*}
\begin{aligned}
\text{Variables} \quad & x \\
\text{Functions} \quad & f \\
\text{Constructors} \quad & C \\
\text{Type variables} \quad & \tau \\
\text{Type constructors} \quad & T \\
\defBNF{Declarations}{decl}{ f \; \overline{x} \; \hs{=} \; body}{function declaration} \\
    \defaltBNF{f \; :: \; t}{type signature} \\
    \defaltBNF{\hs{data} \; T \; \overline{\tau} \; \hs{=} \; \overline{C \; \overline{t}}}{data type declaration} \\
\defBNF{Function body}{body}{\hs{case} \; e \; \hs{of} \; \overline{alt}}{case body} \\
    \defaltBNF{e}{expression body} \\
\defBNF{Expressions}{e}{x}{variable} \\
    \defaltBNF{f \; \overline{e}}{function application} \\
    \defaltBNF{C \; \overline{e}}{constructor application} \\
\defBNF{Alternative}{alt}{pat \rightarrow e}{branch without guard} \\
    \defaltBNF{pat \; \hs{|} \; e \rightarrow e}{branch with guard} \\
\defBNF{Pattern}{p}{x}{pattern variable} \\
    \defaltBNF{C \; \overline{p}}{constructor pattern} \\
\defBNF{Types}{t}{\tau}{type variable} \\
    \defaltBNF{t \; \rightarrow \; t}{function type} \\
    \defaltBNF{T \; \overline{\tau}}{type constructor application} \\
\defBNF{Programs}{prog}{\overline{decl}}{} \\
\end{aligned}
\end{equation*}

This language is a strict subset of Haskell, and inherits its
semantics.  Repeated entities in the BNF are notated with an
$\overline{\text{overline}}$.  Data declarations are needed to
generate axioms of disjointedness and selectors. Type signatures are
ignored in the translation, but the proof techniques introduced later
use this information.

A function is just a function name with a number of variables, and
then a function body, which is either an expression of variables,
functions and constructors, or a case statement with an expression
scrutinee. Branches consists of a pattern, possibly with nested uses
of constructors, and an optional guard, and in the arm is an
expression. A notable exception from ordinary core languages is made
here: nested cases are not allowed. This restriction will aid the
translation. Nested cases will be lifted to top level definitions.

Now we need to distinguish between two translations: the intermediate
translation from Haskell to the intermediate language, and the logic
translation from this language to first order logic. The next section
explains the first part.

\section{The Intermediate Translation}

This section describes the transformation from Haskell to the
intermediate language. The main transformations are top level lifting
of lambdas, local definitions and restricting pattern matching only in
case statements.

\paragraph{Argument pattern matching} A function that does pattern matching will be translated to one that
takes in unmatched arguments and with a case in the body. The
\hs{mirror} function above is thus translated to this:

\begin{code}
mirror :: Tree a -> Tree a
mirror t = case t of
   Fork l x r -> Fork (mirror r) x (mirror l)
   Leaf       -> Leaf
\end{code}

\noindent
If you do pattern matching on several arguments, the scrutinee in the
case will be a tuple of all the arguments.

\paragraph{Local definitions} Where-clauses and let-expressions are
raised to the top level, with all necessary variables as
arguments. This example of an accumulator definition of multiplication
of Peano natural numbers needs such a rewrite:

\begin{code}
(*) :: Nat -> Nat -> Nat
x * y = go Zero x where go acc Zero    = acc
                        go acc (Suc n) = go (acc + y) n
\end{code}

\noindent
The \hs{go} function has the \hs{y} in scope but not as argument so it
is appended to the arguments to the top level lifted version of \hs{go}:

\begin{code}
go acc Zero    y = acc
go acc (Suc n) y = go (acc + y) n y

x * y = go Zero x y
\end{code}

\noindent
Finally the pattern matching in \hs{go} is translated to use a case expression:

\begin{code}
go acc x y = case x of
     Zero  -> acc
     Suc n -> go (acc + y) n y
\end{code}

A similar translation is done for let expressions.

\paragraph{Lambda functions} These are translated to top level
definitions. Take this example of defining \hs{fmap} in terms of the
functions from the \hs{Monad} type class as \hs{liftM}:

\begin{code}
liftM f m = m >>= \x -> return (f x)
\end{code}

\noindent
In the lambda, \hs{f} is a free variable so it becomes an argument to
the new top level function called \hs{lambda} below:

\begin{code}
lambda f x = return (f x)

liftM f m = m >>= lambda f
\end{code}

An analogous translation as is done for lambdas is done for operator sections.

This concludes the translation to the intermediate language, and the
rest of this chapter concentrates on the translation from it to first
order logic. Note that sometimes code will for clarity be written with pattern
matching on arguments directly, but it is implicitly assumed to be
translated to a pattern matching in a case statement.

\section{Pattern Matching Revisited}
\label{sec:patternsrevisited}

\paragraph{Overlapping patterns} These needs to be removed to prevent
generation of inconsistent theories. Example:

\begin{code}
overlap :: Bool -> Bool
overlap b = case b of
              True -> True
              True -> False
\end{code}

Certainly, this cannot be translated to:
\begin{align*}
\rom{1} \qquad & \fn{overlap}(\fn{true}) = \fn{true} \\
\rom{2} \qquad & \fn{overlap}(\fn{true}) = \fn{false} \\
\rom{3} \qquad & \fa{b} b \neq \fn{true} \rightarrow \fn{overlap}(b) = \bot
\end{align*}

Reflexivity gives $\fn{overlap}(\fn{true}) = \fn{overlap}(\fn{true})$,
transitivity of the equalities in the axioms $\romnodot{1}$ and
$\romnodot{2}$ gives that $\fn{true} = \fn{false}$. Together with the
axiom from disjoint constructors, $\fn{true} \neq \fn{false}$, we have
a contradiction.

In Haskell, pattern matching is done from top to bottom of the
definition, making the second match of \hs{True} to never occur. Thus,
the translation removes all patterns that are instances of a pattern
above.



\paragraph{Nested patterns and bottoms} The translation also handles
patterns in more than one depth. At every location in a pattern where
a constructor is matched against, a pattern with bottom at that spot
is also added, defined to bottom. This Haskell function \hs{even}
determines if a list is of even length:

\begin{code}
even :: List a -> Bool
even (Cons x (Cons y ys)) = even ys
even (Cons x xs)          = False
even Nil                  = True
\end{code}

\noindent
For the sake of readability we use the constructors \hs{Cons} and
\hs{Nil} for lists are used since the selectors $\fn{:_0}$ and
$\fn{:_1}$ for the normal cons are hard to read.

Here, \hs{even} should return $\bot$ when it is evaluated with an
argument constructed with neither \hs{Cons} nor \hs{Nil} (recall that
the logic is untyped.) This undefined value should also be returned if
applied to $\hs{Cons x\w}\bot$ for some \hs{x}, since the \hs{Cons}
constructor is matched again on depth two. So there are two different
situations at each depth. One is if there is a match any pattern (for
\hs{even}, it is the variable \hs{xs} in the second pattern), new
patterns are added that matches $\bot$. The other is if there is no
wild pattern, a new one is added that goes to $\bot$.

\begin{comment}
First, it needs to be determined if there is a match-anything branch or not.
For \hs{even} above, there is no match anything case, so a new one is added
that matches anything that is not

For each matched constructor, we need to add a new match to bottom,
which evaluates to bottom. Unnecessary bottoms can be carelessly added
since overlapping patterns are removed \emph{afterwards}. Furthermore,
a wild pattern is added at the end that goes to bottom in case there
are other constructors for the data type not mentioned in the
patterns.
\end{comment}


\pagebreak
No type information is needed to do this insertion, only inspection of
the patterns is required. Could the bottoms be seen in the definition
it would look like this:

\begin{code}[mathescape]
even :: List a -> Bool
even (Cons x (Cons y ys)) = even ys
even (Cons x $\bot$)            = $\bot$
even (Cons x xs)          = False
even Nil                  = True
even _                    = $\bot$
\end{code}

Haskell's behaviour of matching patterns from top to bottom is
justified with implications ensuring the \emph{upward agreement}. The
axioms for this definitions are:
\newcommand\uncons[1]{\cons{\fn{cons_0}(#1)}{\fn{cons_1}(#1)}}
\newcommand\even[1]{\fn{even}(#1)}
\newcommand\cons[2]{\fn{cons}(#1,#2)}
\begin{align*}
\rom{1} && \faaa{x}{y}{ys} & \even{\cons{x}{\cons{y}{ys}}} = \even{ys} \\
\rom{2} && \fa{x}          & \even{\cons{x}{\bot}}         = \bot      \\
\rom{3} && \faa{x}{xs}     & xs \neq \uncons{xs} \wedge xs \neq \bot \rightarrow \even{\cons{x}{xs}} = \fn{false}  \\
\rom{4} &&                 & \even{\fn{nil}} = \fn{true} \\
\rom{5} && \fa{xs}         & xs \neq \fn{nil} \wedge
                             xs \neq \uncons{xs}
                             \rightarrow \even{xs} = \bot
\end{align*}

\begin{comment}
Some room for improvement can be seen: the inserted
\hs{even }$\bot$\hs{ = }$\bot$ case is redundant as it is implied by
the wild pattern to $\bot$.
\end{comment}

The implications due to upward agreement are present in axioms
$\romnodot{3}$ and $\romnodot{5}$. This is needed for all wild
patterns.

\section{Functions as Arguments}

In Haskell, functions readily take other functions as arguments, and
functions can also be partially applied. To get the same behaviour in
logic, each function is assigned a \emph{function pointer}, and a new
binary function is added to the language, written infix as $\appfn$.
For instance if there is a binary function \hs{plus} then a constant
called $\fn{plus.ptr}$ is added to the theory and this axiom:

\begin{equation*}
\faa{x}{y}  \app{(\app{\fn{plus.ptr}}{x})}{y} = \fn{plus}(x,y)
\end{equation*}

When a function is only partially applied, or a function argument is
applied, $\appfn$ is used. Consider this Prelude function \hs{iterate}:

\begin{code}
iterate :: (a -> a) -> a -> [a]
iterate f x = x : iterate f (f x)
\end{code}

It is translated with $\appfn$ in the following way, with the cons
constructor \hs{:} written infix:

\begin{equation*}
\forall \, f \, x \, . \, \fn{iterate}(f,x) = x : \fn{iterate}(f,\app{f}{x})
\end{equation*}

Translaning all function applications could be done using
$\appfn$. However, this approach slows down the theorem provers
significantly, so an optimisation is crucial. Functions will be
regarded as having arity equal to the number of arguments on the left
hand side in their definition. Should a function not get all of its
arguments, appropriate use of $\, @ \, $ is added, as in this function
which increments all elements of the list by one using \hs{map}:

\begin{code}
incr = map (plus one)
\end{code}

As \hs{incr} is written $\eta$-reduced, both \hs{map} and \hs{plus}
are only partially applied. This is the translated axiom:

\begin{equation*}
\fn{incr} = \app{\fn{map.ptr}}{(\app{\fn{plus.ptr}}{\fn{one}})}
\end{equation*}

If \hs{incr} is applied to an argument $xs$, then \hs{incr} is applied
to more arguments than it takes, so we add $\appfn$ so the
corresponding formula becomes $\app{\fn{incr}}{xs}$. By substituting
the definition of $\fn{incr}$ we get
$\app{(\app{\fn{map.ptr}}{(\app{\fn{plus.ptr}}{\fn{one}})})}{xs}$ and
the axiom of $\fn{map.ptr}$ then equals this to
$\fn{map}(\app{\fn{plus.ptr}}{\fn{one}},xs)$.

\paragraph{Doing the impossible}
Although it is not possible to quantify over functions in first order
logic, this translation allows universal quantification of functions,
allowing a way to reason syntactically about partially applied
functions. On the model side, $\appfn$ gives a way to interpret
functions and universally quantify over them. If the function has a
pointer defined, it just constrains $\appfn$ on that pointer to do the
same as the function.

\section{Guards}

Guards are treated similar to pattern matching. If a guard expression
evaluates to \hs{True}, that branch is picked. The expression could
also evaluate to $\bot$, and then the result should be $\bot$. Let us
consider the \hs{filter} function:

\begin{code}
filter :: (a -> Bool) -> List a -> List a
filter p (Cons x xs) | p x = Cons x (filter p xs)
filter p (Cons x xs)       = filter p xs
filter p Nil               = Nil
\end{code}


To translate this to logic it is needed to ensure that if \hs{p x}
evaluates to $\bot$, then so should the function. The axioms look
like this:
\newcommand\filter[2]{\fn{filter}(#1,#2)}
\begin{align*}
\rom{1} && \faaa{p}{x}{xs} & (\app{p}{x}) = \fn{true}                                  \rightarrow \filter{p}{\cons{x}{xs}} = \cons{x}{\filter{p}{xs}} \\
\rom{2} && \faaa{p}{x}{xs} & (\app{p}{x}) = \bot                                       \rightarrow \filter{p}{\cons{x}{xs}} = \bot \\
\rom{3} && \faaa{p}{x}{xs} & (\app{p}{x}) \neq \fn{true} \wedge (\app{p}{x}) \neq \bot \rightarrow \filter{p}{\cons{x}{xs}} = \filter{p}{xs} \\
\rom{4} &&                 & \filter{p}{\fn{nil}} = \fn{nil} \\
\rom{5} && \fa{xs}         & xs \neq \fn{nil} \wedge xs \neq \uncons{xs} \rightarrow \filter{p}{xs} = \bot
\end{align*}

\section{Summary}

The translation of different Haskell concepts is summarised in the
table below:
% Table~\ref{tab:transtable}.

\begin{table}[h!]
  \centering
  \begin{tabular}{|l|l|}
    \hline
    Haskell                    & First Order Logic \\
    \hline
    function                   & function or constant \\
    constructor                & function or constant \\
    data type                  & disjoint constructors and selector axioms \\
    pattern matching           & overlap removal, bottoms insertion, upward agreement \\
    guards                     & equality to true and bottom and upward agreement \\
    partial application        & $\appfn$ on pointer constant \\
    partially applied function & pointer constant and $\appfn$ rule \\
    sections, lambdas, let     & new functions with variables in scope as arguments \\
    \hline
  \end{tabular}
%  \caption{Translation of different Haskell constructs
%    \label{tab:transtable}
%  }

\end{table}

%Equational reasaoning is traditional in proving corrected of Haskell
%programs, but it assumes that a simple denotational semantics exists,
%and there is not even a formal semantics for the language
%\cite{chasingbot}.
%

% Remove unnecessary definitions for a given proof


%% Domain theory

\section{Domain Theory}
\label{sec:domaintheory}

This section is stand alone, and could be skipped especially if you
already know the basics of Domain Theory: comp\-lete partial orders,
monotonicity and continuity.
It explains these concepts and discusses how it can be
used to verify the translation, furthermore it is used as a reference
in the future sections that rely on concepts from domain theory.
%The section explains these concepts and acts as a reference in future
%sections that rely on concepts from domain theory.

The values of every data type are ordered on how much ``information''
they contain. The least element bottom, denoted $\bot$, contains least
information. It corresponds to all kinds of crashes in Haskell; use of
\hs{undefined}, non-termination or non exhausite pattern matches.
Different constructors hold different information, so they are not
related by the ordering; this is a partial order, a relation that is
reflexive, transitive and antisymmetric. The ordering is usually
written $\sqsubseteq$ sometimes with a subscript indicating the type.

\begin{wrapfigure}{O}{0.4\textwidth} %\begin{figure}
\vspace{-7pt}
\centering \begin{tikzpicture}[
    level distance=-1.5cm,
    growth parent anchor=north,
    sibling distance=3cm
]
\node {$\bot$}
    child {
        node {$\hs{True}$}
    }
    child {
        node {$\hs{False}$}
    };
\end{tikzpicture}


%\end{document}
\vspace{-7pt}
\caption{
    The order of Bool values.
    \label{fig:boolcpo}
}
\end{wrapfigure}
For the \hs{Bool} data type the partial order can be drawn as a Hasse
Diagram and this can be shown in Figure \ref{fig:boolcpo}.  From the
picture it is understood that $\bot$ is the least element, and the
line from it to \hs{False} means that $\bot \sqsubseteq \hs{False}$,
since $\bot$ is below $\hs{False}$. Correspondingly for $\hs{True}$,
the diagram tells us that $\bot \sqsubseteq \hs{True}$. It can also
been seen that $\hs{True} \nsqsubseteq \hs{False}$; they are unrelated
since there is no line between them.

%\begin{figure}[h]
%\centering
%\begin{tikzpicture}[
    level distance=-1.5cm,
    growth parent anchor=north,
    sibling distance=3cm
]
\node {$\bot$}
    child {
        node {$\hs{True}$}
    }
    child {
        node {$\hs{False}$}
    };
\end{tikzpicture}


%\end{document}
%\caption{The partial order for \texttt{Bool} as a Hasse Diagram
%  \label{fig:boolcpo}
%}
%\end{figure}

%\begin{figure}[h]
%  \centering
%  \subfloat[\texttt{Bool}]{\label{fig:boolcpo}\begin{tikzpicture}[
    level distance=-1.5cm,
    growth parent anchor=north,
    sibling distance=3cm
]
\node {$\bot$}
    child {
        node {$\hs{True}$}
    }
    child {
        node {$\hs{False}$}
    };
\end{tikzpicture}


%\end{document}}
%  \hspace{20pt}
%  \subfloat[\texttt{(Bool,Bool)}]{\label{fig:boolboolcpo}%\documentclass[10pt]{article}
\newcommand{\myGlobalTransformation}[2]
{
    \pgftransformreset;
    \pgftransformcm{1.6}{0}{0.6}{0.5}{\pgfpoint{#1cm}{#2cm}}
}

\newcommand\tru{\hs{T}}
\newcommand\fal{\hs{F}}

\newcommand\ddraw[2]{
        \draw[-,line width=3pt,draw=white] (#1) -- (#2);
        \draw (#1) -- (#2);
}

%\begin{document}
%\pagestyle{empty}

\begin{tikzpicture}

    \begin{scope}
        \myGlobalTransformation{0}{0};
        \node (bottom) at (0,0) {$\bot$};

        \myGlobalTransformation{0}{1};
        \node (botbot) at (0,0) {$(\bot,\bot)$};

        \myGlobalTransformation{0}{3};
        \node (trubot) at (-1,0) {$(\tru,\bot)$};
        \node (bottru) at (0,1)  {$(\bot,\tru)$};
        \node (falbot) at (1,0)  {$(\fal,\bot)$};
        \node (botfal) at (0,-1) {$(\bot,\fal)$};

        \myGlobalTransformation{0}{5};
        \node (trutru) at (-0.7, 0.7) {$(\tru,\tru)$};
        \node (faltru) at ( 0.7, 0.7) {$(\fal,\tru)$};
        \node (falfal) at ( 0.7,-0.7) {$(\fal,\fal)$};
        \node (trufal) at (-0.7,-0.7) {$(\tru,\fal)$};

        \draw (bottom) -- (botbot);

        \draw (botbot) -- (trubot);
        \draw (botbot) -- (bottru);
        \draw (botbot) -- (falbot);
        \draw (botbot) -- (botfal);

        \ddraw{trubot}{trutru};
        \ddraw{bottru}{trutru};
        \ddraw{falbot}{faltru};
        \ddraw{bottru}{faltru};
        \ddraw{trubot}{trufal};
        \ddraw{botfal}{trufal};
        \ddraw{botfal}{falfal};
        \ddraw{falbot}{falfal};

    \end{scope}

\end{tikzpicture}

%\end{document}}
%  \caption{Two partial orders as Hasse Diagrams}
%  \label{fig:pos}
%\end{figure}

\begin{wrapfigure}[25]{r}{0.4\textwidth} %\begin{figure}\begin{figure}[h!]
\begin{center}
\vspace{-30pt}
%\documentclass[10pt]{article}
\newcommand{\myGlobalTransformation}[2]
{
    \pgftransformreset;
    \pgftransformcm{1.6}{0}{0.6}{0.5}{\pgfpoint{#1cm}{#2cm}}
}

\newcommand\tru{\hs{T}}
\newcommand\fal{\hs{F}}

\newcommand\ddraw[2]{
        \draw[-,line width=3pt,draw=white] (#1) -- (#2);
        \draw (#1) -- (#2);
}

%\begin{document}
%\pagestyle{empty}

\begin{tikzpicture}

    \begin{scope}
        \myGlobalTransformation{0}{0};
        \node (bottom) at (0,0) {$\bot$};

        \myGlobalTransformation{0}{1};
        \node (botbot) at (0,0) {$(\bot,\bot)$};

        \myGlobalTransformation{0}{3};
        \node (trubot) at (-1,0) {$(\tru,\bot)$};
        \node (bottru) at (0,1)  {$(\bot,\tru)$};
        \node (falbot) at (1,0)  {$(\fal,\bot)$};
        \node (botfal) at (0,-1) {$(\bot,\fal)$};

        \myGlobalTransformation{0}{5};
        \node (trutru) at (-0.7, 0.7) {$(\tru,\tru)$};
        \node (faltru) at ( 0.7, 0.7) {$(\fal,\tru)$};
        \node (falfal) at ( 0.7,-0.7) {$(\fal,\fal)$};
        \node (trufal) at (-0.7,-0.7) {$(\tru,\fal)$};

        \draw (bottom) -- (botbot);

        \draw (botbot) -- (trubot);
        \draw (botbot) -- (bottru);
        \draw (botbot) -- (falbot);
        \draw (botbot) -- (botfal);

        \ddraw{trubot}{trutru};
        \ddraw{bottru}{trutru};
        \ddraw{falbot}{faltru};
        \ddraw{bottru}{faltru};
        \ddraw{trubot}{trufal};
        \ddraw{botfal}{trufal};
        \ddraw{botfal}{falfal};
        \ddraw{falbot}{falfal};

    \end{scope}

\end{tikzpicture}

%\end{document}
\caption{
    \texttt{(Bool,Bool)} partial order.
    \label{fig:boolboolcpo}
}
\end{center}
\end{wrapfigure} %\end{figure}
For tuples and other constructors that take other data types as
parameters, the ordering is:
\begin{equation*}
\hstup{x_0}{y_0} \sqsubseteq_{(a,b)} \hstup{x_1}{y_1} \text{\quad iff \quad}
x_0 \sqsubseteq_a x_1 \text{\w and \w} y_0 \sqsubseteq_b y_1
\end{equation*}

The Hasse Diagram for the \hs{(Bool,Bool)} values can be seen in
Figure \ref{fig:boolboolcpo}. Here \hs{True} is abbreviated for \hs{T}
and similarly for \hs{False}. It is not flat as the one for \hs{Bool};
it can be seen as three dimensional. On the lowest layer the only
value is $\bot$, on the next layer $\hstup{\bot}{\bot}$. Above that
the tuples with one $\bot$, and finally the total values at the
top.

\vspace{55pt}

\subsection{Monotonicity}
 An important property all safe Haskell functions have is that they are
monotone with respect to this ordering.

\paragraph{Definition} A function $f$ is \emph{monotone} iff

\begin{equation*}
\faa{x}{y} \quad x \sqsubseteq y \quad \Rightarrow \quad f(x) \sqsubseteq f(y).
\end{equation*}

This can be understood in many ways. One way to see it is if you have
two inputs to a function, one containing \emph{less} information that
the other, i.e. more bottoms, it is impossible to return \emph{more}
information from the input with less information.

\newpage

One simple example of a consequence of this is the impossibility to
make a function \hs{isBottom :: a -> Bool}, returning \hs{True} if the
argument is bottom, and \hs{False} otherwise:

\note{rewrite with code}
\begin{align*}
& \hs{isBottom} \w :: \hs{a} \rightarrow \hs{Bool} \\
& \hs{isBottom} \w \bot = \hs{True} \\
& \hs{isBottom} \w x \, = \hs{False}, \qquad x \neq \bot
\end{align*}

\noindent
Since $\bot \sqsubseteq x$ for any $x$, then by monotonicity we must
necessarily have
$$\hs{isBottom} \w \bot \sqsubseteq \hs{isBottom} \w x.$$
Take any non-bottom $x$, and this equation gives
$\hs{True} \sqsubseteq \hs{False}$, which is false. Hence
\hs{isBottom} is not monotone.

\subsection{Continuity}
Another domain theoretic property that Haskell functions have is that
they are continuous. This is a property that gives us insight in how
functions behave on infinite input.  To describe this, we need to
consider the partial order of a data type with infinite values. The
prime candidate \hs{data Nat = Zero | Succ Nat} is used and Hasse
Diagram can be seen in Figure \ref{fig:natcpo}.

\begin{figure}[h]
\centering
\usetikzlibrary{positioning,shadows,arrows}

\def\adots{\mathinner{\mkern2mu\raise\hbox{.}
\mkern2mu\raise4\hbox{.}\mkern1mu
\raise8\vbox{\kern7\hbox{.}}\mkern1mu}}


%\begin{tikzpicture}[scale=10]
%
%  \node (bottom)                          {$\bot$};
%  \node (zero)        [above=of bottom]   {$Zero$};
%  \node (suc bot)     [right=of zero]     {$Suc \, \bot$};
%  \node (suc zero)    [above=of suc bot]  {$Suc \, Zero$};
%  \node (suc suc bot) [right=of suc zero] {$Suc \, (Suc \, \bot)$};
%
%  \draw [-] (bottom) -- (zero);
%  \draw [-] (bottom) -- (suc bot);
%  \draw [-] (suc bot) -- (suc zero);
%  \draw [-] (suc bot) -- (suc suc bot);
%
%\end{tikzpicture}
%
\begin{tikzpicture}[grow'=up,sibling distance=2cm]
\node {$\bot$}
    child {
        node {$\hs{Zero}$}
    }
    child {
        node {$\hs{Succ} \, \bot$}
        child {
            node {$\hs{Succ} \, \hs{Zero}$}
        }
        child {
            node {$\hs{Succ} \, (\hs{Succ} \, \bot)$}
            child {
                node {$\hs{Succ} \, (\hs{Succ} \, \hs{Zero})$}
            }
            child {
              node {$ ^{ ^{\adots}}$}
              child [edge from parent/.style={draw=white}] { }
              child {
                node {$\hs{inf}$}
              }
            }
        }
    }

\end{tikzpicture}


\caption{
    The (complete) partial order for \texttt{Nat}, with \hs{inf = Succ inf.}
    \label{fig:natcpo}
}
\end{figure}

At the top we have the infinite value \hs{inf}, defined in Haskell as
\hs{inf = Succ inf}. Here \hs{inf} is the \emph{limit} of an
$\omega$-chain, i.e a chain with the same number of elements as
$\omega$, the natural numbers. The chain is:

\begin{equation*}
\bot \sqsubseteq
\hs{Succ} \, \bot \sqsubseteq
\hs{Succ} \, (\hs{Succ} \, \bot) \sqsubseteq
\hs{Succ} \, (\hs{Succ} \, (\hs{Succ} \, \bot)) \sqsubseteq
\cdots
\end{equation*}

This chain could succinctly be written $\langle \hs{Succ}^n \, \bot
\rangle_{n \in \omega}$.  Here $\hs{Succ}^n$ means $n$ applications of
the \hs{Succ} constructor. The limit is written $\lub{n \in
  \omega}(\hs{Succ}^n \, \bot)$ and is equal to \hs{inf}, where
$\lub{}$ is the least upper bound. All elements in the chain satisfy
the property of being less than or equal to the limit: $\hs{Succ}^n \,
\bot \sqsubseteq \hs{inf}$.

A partial order is a complete partial order iff there is a limit for
every $\omega$ chain. All data types in Haskell are complete partial
orders\footnote{Notice that the data type \hs{data StrictNat = Zero |
    Succ !StrictNat} is flat and therefore complete.}. Now we can
define continuity.

\paragraph{Definition} A function $f$ is \emph{continuous} iff it is
monotone and preserves the $\lub{ }$ of all $\omega$-chains: i.e.
assume any chain $\langle x_n \rangle_{n \in \omega}$, then:

\begin{equation*}
\lub{n \in \omega} \, (f \, x_n) \eq f \, (\lub{n \in \omega} \, x_n)
\end{equation*}

Just as with monotonicity, there are several ways to interpret
this. One way is to say that what a function does on a chain, it must
also do on the chain's limit, as with \hs{map} on increasingly longer
lists. Another is to say that a function cannot produce finite output by
inspecting infinite input: there is no function
\hs{isFinite :: [a] -> Bool} returning \hs{True} on finite lists and
\hs{False} on infinite lists. On the increasing chain
$$ \bot \sqsubseteq x_0 \hs{:} \bot \sqsubseteq x_0 \hs{:} x_1 \hs{:} \bot
\sqsubseteq \cdots$$
the function \hs{isFinite} returns \hs{True} (or $\bot$), but the
limit should return \hs{False}, so this is not a continuous function.

An interesting formulation of Church's Thesis in terms of continuity
is given by Plotkin \cite{domains}:

\begin{center}
\emph{A function is continuous iff it is physically feasible.}
\end{center}

This means that all computable functions are contiuous, and the other
way around. The conclusion for us is that all Haskell functions are
continuous.

\subsection{Unsafe Haskell}
In GHC, you can use \hs{unsafePerformIO} and \hs{catch} from
\hs{Control.Exception} and other tricks to unsafely catch errors
(bottoms). With this machinery it is possible to write a function
\hs{isBottom :: a -> Bool} to catch calls to \hs{undefined}, pattern
match failures, etcetera. In addition, some non-termination some can
also be catched in Haskell because of the \emph{blackhole} run time
object that replaces a \emph{thunk} that is being currently
evaluated. It does not and indeed cannot cover all non terminating
functions because of the undecidability of the Halting problem.

The domain theoretic results remain; one can see $\bot$ as another,
albeit inconveniently inspected, constructor to every data type. All
patterns are exhaustive: every function has an implicit match any
pattern to $\bot$.  Then we add a \emph{true} bottom to the domain
denotes the uncatchable bottoms; undeterminable non termination. With
this setting all Haskell functions are continuous with respect to the
\emph{true} bottoms. But for the rest of this thesis, we shall only
consider pure and safe Haskell functions.

\subsection{Monotonicity as Verification}

Continuity is a concept that is hard to express in first order logic:
it in not able to express countability. We can come close with an
axiomatization of set theory, but we leave that issue and focus on
monotonicity. A way to verify the translation is to add axioms to the
generated theory describing the $\sqsubseteq$ relation, and axioms
that asserts that each function is monotone. An automated theorem
prover could not easily show that it is a satisfiable theory since it
will normally only have infinite models. However, a long run without
any counter model could be seen as a witness for a successful
translation in this respect.


\section{Future Work}

Haskell is a big language, and translating it all in one go is a
daunting task. Therefore, some restrictions were settled to be able to
focus on proving rather than translating.  The goal was to add enough
of the Haskell language to enable to prove interesting properties, but
much of the widely available sugar in Haskell was omitted since it
does not add extra expressibility. This means that list comprehensions
and are not supported but can be added by their respective rewriting
rules. \hs{Type} definitions should be unrolled, so they could be
used in the signature for properties. Type classes is probably the
most interesting thing to add, and is discussed in Section
\ref{sec:typeclasses} in future work.

Another interesting but omitted feature are the built-in types
\hs{Int}, \hs{Integer}, \hs{Double}, \hs{Char}, etc. For \hs{Integer}
appropriate axioms could be added that hold for $\mathbb{Z}$, the
canonical infinite discretely ordered commutative ring.  The other
data types do not enjoy such well behaved properties because of
different bit sizes and overflow and precision errors.

Syntactic features for controlling lazy and strict evaluation namely
irrefutable patterns, \hs{seq} and bang patterns, and richer pattern
matching in form of pattern bindings are discussed below.

\begin{comment}
 It should be
noted that it is already possible to prove a lot of interesting
Haskell properties, it is far from able to prove things about bigger
Haskell projects which usually use a richer part of the language.
\end{comment}

\subsection{Irrefutable Patterns and Pattern Bindings}

Irrefutable patterns can be defined in terms of projections, examples
are \hs{fst}, \hs{snd}, \hs{head}, \hs{fromJust} defined in the
standard library. Each irrefutable pattern is translated to a
constant, and when you use the variables in the pattern, you translate
it to appropriate use of projections. One example is the translation
of the \hs{uncurry} function:

\begin{code}[mathescape]
uncurry f ~(x,y) = f x y        $\Rightarrow$      uncurry f t = f (fst t) (snd t)
\end{code}

\noindent
The irrefutable pattern \verb:~(x,y): is replaced with the new constant
\hs{t}, and in the body of the function, \hs{x} is replaced with the
strict projection \hs{fst t}, and similarly for \hs{y}.

Top level patterns, also called pattern bindings, can
also be written in terms of such projections. The whole pattern
is replaced with a constant, and when the variables from the pattern
are used, you again replace it with projections. This is how it
could look for a simple \hs{fromJust . lookup} implementation:

\begin{code}[mathescape]
unsafeLookup x xs = v           $\Rightarrow$      unsafeLookup x xs = fromJust t
  where Just v = lookup x xs            where t = lookup x xs
\end{code}

The strict projections would not rely on the user having \hs{fst} or
\hs{fromJust} in scope, they can automatically be generated by
inspection of the data type definition.

\subsection{Bang Patterns and \texttt{seq}}

The translations for bang patterns and \hs{seq} are also
straightforward. \hs{seq} defined by bang patterns is:

\begin{code}
seq :: a -> b -> b
seq !x y = y
\end{code}

The axioms for a translation of \hs{seq} needs to ensure that if
\hs{x} evaluates to $\bot$, then \hs{seq x} also evaluates to
$\bot$. The two axioms for this functions are:
\begin{align*}
\rom{1} \qquad & \fa{y}    seq(\bot,y) \eq \bot \\
\rom{2} \qquad & \faa{x}{y} x \neq \bot \rightarrow seq(x,y) \eq y
\end{align*}

Either you implement bang patterns in this fashion, or you do the same
translation as GHC for bang patterns: for each strict variable, you
add a \hs{seq} for that variable for the expression of that pattern,
and you simply add the axioms for \hs{seq} to the theory if the
program uses it or bang patterns. For data types with strictness
fields one proceeds by adding \hs{seq} when constructing elements.

\subsection{Pattern Guards}

Patterns guards is a GHC specific extension to Haskell which allows
arbitrary pattern matching on the result from an expression in a
guard. An example is this elaboration of the \hs{lookup} function from
the \hs{Prelude}, which applies a function to the element, if found:

\begin{code}
transformLookup :: Eq k => k -> [(k,v)] -> (k -> v -> b) -> Maybe b
transformLookup k xs f | Just v <- lookup k xs = Just (f k v)
                       | otherwise             = Nothing
\end{code}

\noindent
If the look up returns \hs{Just}, \hs{v} is already bound and can be
used in the expression of the right hand side. This is very similar to
normal guards, as they are a special case of pattern guards: the guard
\hs{f x | p x} is expressed as \hs{f x | True <- p x}. The current
translation of guards checks if \hs{p x} is \hs{True}, and then
``picks'' this branch, or is $\bot$. This could be done for
constructors, bottoms would need to be added in the guard branches as
is currently done for ordinary patterns.



\chapter{Proof Techniques}

To prove things using this technique, properties are entered in the
Haskell source code. A small prelude called \hs{AutoPrelude} needs to
be imported that gives access to the relevant functions. One example
is the associativity of list concatenation:

\begin{code}
import AutoPrelude

prop_app_assoc :: [a] -> [a] -> [a] -> Prop [a]
prop_app_assoc xs ys zs = xs ++ (ys ++ zs) =:= (xs ++ ys) ++ zs
\end{code}

The infix function \hs{=:=} comes from the import, as well as the type
constructor \hs{Prop}. The type signature cannot be omitted as this is
used for some proof techinques. For induction, you need to see which
kind of induction we will use.

Running the program is simple. Just save the file as for instance
\hs{ListProps.hs} and run

\begin{code}
autospec ListProps.hs
\end{code}

\noindent
and the program will report if it was provable or not, and which
techniques succeeded. Equational properties written like this are also
testable with QuickCheck, so you can run the normal \hs{quickCheck}
function on them, given that there are appropriate \hs{Eq} and
\hs{Arbitrary} instances provided.

The rest of this chapter explains the different proof methods
supported in this tool: definitional equality (Section
\ref{sec:equality}), structural induction (Section
\ref{sec:induction}), fixed point induction (Section
\ref{sec:fixpoint}) and approximation lemma (Section
\ref{sec:approx}).
% Definitional Equality -------------------------------------------------------

\section{Definitional Equality}
\label{sec:equality}

Some properties cannot or need not use induction or some more
sophisticated technique, since they are true by definition. Examples
are properties for fully polymorphic functions such as the definition
of \hs{id} in the SK-calculus, here

\begin{code}
s f g x = f x (g x)
k x y = x
id x = x

prop_skk_id :: Prop (a -> a)
prop_skk_id = s k k =:= id
\end{code}

Then, the generated conjecture is simply

\begin{equation*}
\app{ (\app {\ptr{s}} {\ptr{k}} )
    }{\ptr{k}} = \ptr{id}
\end{equation*}

Another example where this is useful is to prove functor and monad
laws for the environment monad.

\subsection{Extensional Equality and seq}

To prove the previous property we also need to have extensional
equality, postulated with the following axiom

\begin{equation*}
\faa{f}{g} (\fa{x} \app{f}{x} = \app{g}{x}) \rightarrow f = g
\end{equation*}

\noindent
which identifies function pointers and functions composed with $@$.
One problem with extensional equality in Haskell, is that the presence
of \hs{seq} breaks it. \hs{seq} is a built in function with the
following behavior:

\begin{code}[mathescape]
seq :: a -> b -> b
seq $\bot$ y = $\bot$
seq x y = y$, \qquad x \neq
\end{code}

It forces the first argument to weak head normal form and returns the
second. For our purposes, it is only important if the first argument
is $\bot$, then the function also returns this as it is strict in its
first argument. With \hs{seq} it is possibly to distinguish between
these two functions, which otherwise are observationally equal:

\begin{code}[mathescape]
f = $\bot$
g = \x -> $\bot$
\end{code}

Because \hs{seq f ()} evaluates to $\bot$, and \hs{seq g ()} evaluates
to \hs{()}, but on any argument \hs{f} and \hs{g} gets, they both
return $\bot$. Here we also need an extra axiom, which says that
anything applied to $\bot$ is $\bot$:

\begin{equation*}
\fa{x} \app{\bot}{x} = \bot
\end{equation*}

However, \hs{seq} is the only function that can tell such functions
apart, so we will ignore its presence in Haskell.  In the future,
there could be added as a switch \hs{--enable-seq}, which weakens
extensional equality appropriately.

Furthermore, if extensional equality is assumed we also have
the property that \hs{Prop (a -> b)} is equivalent to
\hs{a -> Prop b}, by letting the property have an extra argument that
is applied to the left and right hand side of the equality. This has
two benefits. Firstly, it can trigger other proof methods should \hs{a}
or \hs{b} be concrete types (the former for induction and the latter
for approximation lemma). Secondly, it does not need to use the
extensionality axiom introduced above which introduces extra steps in
the proof search.

\subsection{Future Work: Concrete Concerns}

This only works on non-concrete types because of the way bottoms are
added. One example when this is a problem with is this plausible
definition of boolean or, and the property of right identity:

\begin{code}
False || a = a
True  || _ = True

prop_or_right_identity :: Bool -> Prop Bool
prop_or_right_identity x = x || False =:= x
\end{code}

The translation of \hs{||} makes any element in the domain that is not
the introduced constants $\fn{false}$ or $\fn{true}$ for \hs{Bool}'s
constructors, equal $\bot$. Now consider the translation of the
property:

\begin{equation*}
\fa{x} x \, \fn{||} \, \fn{false} = x
\end{equation*}

Now this is false: take a model with another element $\star$ in the
domain:

$$\star \, \fn{||} \, \fn{false} = \bot$$

The consequence of this is that the proof principle of definitional
equality is only used on abstract types, rather than polymorphic, as
they cannot be strict. Do not fear: the property above is trivially
proved by induction. For the \hs{Bool} type and other non recursive
data types, induction degenerate into mere case enumeration.




% Structual Induction ---------------------------------------------------------

\section{Structural induction}

\subsection{Induction}

Some background of induction and how it is present in well-known
theories like PA, ZFC, MLTT and CoC.

PA which only concerns natural numbers has a small vocabulary
consisting only of the constant $0$, the unary successor function $s$,
and binary plus and multiplication.
Here the induction schema from looks like this:

\note{One could also be explicit about the free variables in $P$}
\begin{mathpar}
  \inferrule*
     {
       \overbrace{P(0)}^{\mathrm{base}}
       \\
       \overbrace{
           \fa{x}
                 \underbrace{P(x)}_{\mathrm{hypothesis}}
              \rightarrow
                 \underbrace{P(s(x))}_{\mathrm{conclusion}}
       }^{\mathrm{step}}
     }
     { \fa{x} P(x) }
\end{mathpar}

This is a axioms schema since it is not possible to quantify over the
predicate $P$ in FOL. Rather, it is a infinite set of axioms, one for
each (well-formed) formula instantiated in place for $P$. Generally,
ATPs do not instantiate schemas themselves but it has to be done
manually, with an appropriate formula for $P$.

Any non-recursive, or more importantly recursive data type gives rise
to induction schemata.

\footnote{Haskell's natural numbers are of course also cluttered with
  elements that are not natural numbers, such as $\bot$, but also the
  infinite ``natural number'' defined by \hs{infinite = Succ infinite}}
, defined the usual way in Haskell by \hs{data Nat = Zero | Succ Nat}
yields this induction axiom schema:

\begin{mathpar}
  \inferrule*
     {
       P(\fn{Zero})
       \\
       \fa{x} P(x) \rightarrow P(\fn{Succ}(x))
     }
     { \fa{x} P(x) }
\end{mathpar}


% Fixpoint Induction ----------------------------------------------------------

\section{Fixed point induction}
\label{sec:fixpoint}

Induction is applicable when arguments are of a concrete type, such as
lists or natural numbers. There are also properties where all
arguments are of abstract types. The canonical example is the
map-iterate property:

\begin{equation*}
\faa{(f : a \rightarrow a)}{(x : a)} \hs{map} \w f \w (\hs{iterate} \w f \w x) \eq
           \hs{iterate} \w f \w (f \w x)
\end{equation*}

Here $f$ is an abstract function $a \rightarrow a$, and $x$ is
something of type $a$. This example is further investigated in Section
\ref{sec:mapiter} below, but it is already clear that we cannot proceed to
prove this with structural induction.

Enter fixed point induction. It allows a way of performing induction
on the recursive structure of the program. In short, if the property
regards a function $f$, the hypothesis is that the property holds for all
the recursive calls in the definiton of $f$, and the goal is to prove
that it holds for $f$.

\begin{comment}
It
is an early example of a technique from domain theory, attributed to
Scott and de Bakker,
\note{Citation needed: there is a book called
  Mathematical Theory of Program Correctness by Jaco de Bakker that
  could be appropriate if found}
and sometimes called Scott induction
or computational induction.  \cite{domains}
\end{comment}

The least fixed point for a function can be found in Haskell with the
function \hs{fix}, which can simply be defined as:

\begin{code}
fix :: (a -> a) -> a
fix f = f (fix f)
\end{code}

This function solves the equation $x = f \w x$, since substituting $x$
for $\hs{fix} \w x$: the left side evaluates to $f \w (\hs{fix} \w f)$
in one step, which is then equal to the right side. This is the origin
of the name of the combinator \hs{fix}: this is a fixed point of the
equation.  Any self-recursive function can be rewritten in terms of
\hs{fix}. Recall the \hs{Prelude} function \hs{repeat}, which makes an
infinite list of the same element \hs{repeat x = x : repeat}. Its
transformation to \hs{fix}-style is this:

\begin{code}
repeat x = fix r
  where r i = x : i
\end{code}

Computing \hs{repeat x}, we get the following unfolds:
\begin{equation*}
  \hs{repeat x}
= \hs{fix r}
= \hs{x:fix r}
= \hs{x:x:fix r}
= \hs{x:x:x:fix r}
  \cdots
\end{equation*}
So \hs{fix (x:)} is the infinite list of \hs{x}. The translation of a
self-recursive function to be defined in terms of \hs{fix} is
mechanical. Assume $f$ is defined with arguments $\overline{x}$ and
has a body $e$ that uses both itself and its arguments, let us write
this as $e(f,\overline{x})$. Then the translation is this:

\begin{equation*}
f \w \overline{x} \eq e(f,\overline{x})
\w \Leftrightarrow \w
f \eq \hs{fix} \w (\lambda \w f' \w \overline{x} \w \rightarrow \w e(f',\overline{x}))
\end{equation*}

Another more domain theoretic approach is to say that
$\hs{fix} \w f \eq \lub{n}(f^n \bot)$, where $f^n \bot$ is $n$ applications of $f$:
\begin{equation*}
f^n \bot \eq \underbrace{f (f (\cdots (f}_{n \w \mathrm{copies \w of} \w f}} \bot) \cdots))
\end{equation*}
This corresponds to a potentially infinite, countable unrolling of $f$.
It is easy to verify that $\langle f^n \bot\rangle_{n\in\omega}$ is a
$\sqsubseteq$-chain by induction on $n$, and that this is the least
pre-fixed point of $f$ is also showed by induction: Assume there
is another pre-fixed point $\theta$, thus satisfying
$\theta \eq f \w \theta$. The base case is
$\bot \eq f^0 \bot \sqsubseteq \theta$, trivially satisified since
$\bot$ is the least element. For the step case, assume that
$f^n \bot \sqsubseteq \theta$, and we get the conclusion
$f^{n+1} \bot = f (f^n \bot) \sqsubseteq f \w \theta = \theta$ as desired.
Fixpoint induction proves properties about a function written in terms
of \hs{fix}, and its inference rule is this:

\begin{mathpar}
  \inferrule*
     {
       P(\bot)
       \\
       P(x) \rightarrow P(f \w x)
       \\
       P \w \mathrm{admissible}
     }
     { P(\fn{fix} f) }
\end{mathpar}


Here it is important that $P$ is \emph{admissible} , meaning that for
all $\sqsubseteq$-chains of length $\omega$, if the property holds for
all elements in the chain it must necessary hold for its limit, futher
described in Section \ref{sec:admissible}, where a proof is given that
the properties we are interested in, universally quantified equalities
of continuous functions, are admissible predicates.

An interesting property of fixed point induction is that it does not
care about types: indeed, it works in an untyped setting. In addition,
it can exploit strange recursive structures of the function. A caveat
is that it can only prove properties that must hold for infinite and
partial values.

The proof that fixed point induction relies on the fact that
$\lub{n}(f^n \w \bot) \eq \hs{fix} \, f$, where $f^n$ is $n$
self-applications of $f$. This is true since \hs{fix} is defined as
$f$ self-applied to it self. Apart from this, the proof only uses
induction over natural numbers and that $f^0 \w \bot \eq \bot$, and
it is of course important that $P$ is admissible. See proof below:

\begin{align*}
P(\bot) & \wedge \fa{x} P(x) \rightarrow P(f x) \\
\desclra{$f^0 \w \bot \eq \bot$} \\
P(f^0 \w \bot) & \wedge \fa{x} P(x) \rightarrow P(f x) \\
\descra{quantifying} \\
P(f^0 \w \bot) & \wedge \fa{n} P(f^n \w \bot) \rightarrow P(f^{n+1} \w \bot) \\
\desclra{induction} \\
\fa{n} & P(f^n \w \bot) \\
\desclra{\textit{P} \w admissible} \\
& P(\lub{n}(f^n \w \bot)) \\
\desclra{definition \w of \w \hs{fix}} \\
& P(\hs{fix} \w f) \\
\end{align*}

One reason to introduce fixed point induction is to avoid the natural
numbers in $\fa{n} P(f^n \bot)$  to prove $P(\hs{fix} \w f)$.

\subsection{Example: map-iterate}
\label{sec:mapiter}

For properties that do not have any arguments with a concrete type,
structural induction is not applicable. The Haskell function
\hs{iterate} is a that makes an infinite list from a seed, by repeated
application of a function, i.e \hs{iterate f x} is the list
 \hs{x:f x:f (f x):}$\cdots$. It is related to Haskell function
 \hs{map} in the map-iterate property, stated as follows:

\begin{equation*}
\faa{f}{x} \hs{map} \w f \w (\hs{iterate} \w f \w x) \eq
           \hs{iterate} \w f \w (f \w x)
\end{equation*}

\noindent
With their standard definitions given in \ref{code:mapiterate} below.

\begin{figure}[h!]
\centering
\begin{minipage}[b]{6cm}
\begin{code}
map :: (a -> b) -> [a] -> [b]
map f (x:xs) = f x : map f xs
map f [] = []
\end{code}
\end{minipage}
\hspace{10pt}
\begin{minipage}[b]{6cm}
\begin{code}[mathescape]
iterate :: (a -> a) -> a -> [a]
iterate f x = x : iterate f (f x)
$\newline$
\end{code}
\end{minipage}
\caption{Definition of \texttt{map} and \texttt{iterate}
\label{code:mapiterate}
}
\end{figure}

The behavior of \hs{map} is to apply a function to every element of a
list. We see that we cannot use structural induction here, since both
$f$ and $x$ are abstract, but this can be proved by fixpoint induction
on \hs{iterate}. First, we rewrite this function in terms of \hs{fix}:

\begin{code}
iterate = fix iter
iter i f x = x : i f (f x)
\end{code}

The predicate $P$ from fixpoint induction is $P(g) \w \Leftrightarrow
\w \faa{f}{x} \hs{map} \w f \w (g \w f \w x) \eq g \w f \w (f \w x) $. If we
prove the base case and step case we can then conclude
$P(\hs{fix iter})$, and that is by definition $P(\hs{iterate})$.

The base case is $P(\bot)$. Since \hs{map} is strict in its second
argument, it is both the left side and right side evaluate to $\bot$.
The for the step case we have to show
$P(\hs{i}) \rightarrow P(\hs{iter i})$. We start from the induction
hypothesis and work towards the goal as follows:

\begin{align*}
\w \faa{f}{x} \hs{map} \w f \w (\hs{i} \w f \w x) & \eq \hs{i} \w f \w (f \w x) \\
\descra{generalizing $x$ to $f \w x$} \\
\w \faa{f}{x} \hs{map} \w f \w (\hs{i} \w f \w (f \w x)) & \eq \hs{i} \w f \w (f \w (f \w x)) \\
\descra{substitutivity} \\
\w \faa{f}{x} f \w x \hs{:} \hs{map} \w f \w (\hs{i} \w f \w (f \w x)) & \eq f \w x \hs{:} \hs{i} \w f \w (f \w (f \w x)) \\
\desclra{\defof{\texttt{map}}} \\
\w \faa{f}{x} \hs{map} \w f \w (x \hs{:} \hs{i} \w f \w (f \w x)) & \eq f \w x \hs{:} \hs{i} \w f \w (f \w (f \w x)) \\
\desclra{\defof{\texttt{iter}}} \\
\w \faa{f}{x} \hs{map} \w f \w (\hs{iter} \w \hs{i} \w f \w x) & \eq \hs{iter} \w \hs{i} \w f \w (f \w x) \\
\end{align*}

As discussed earlier, the $P$ used is admissible since it is an
universally quantified equality. Hence, fixpoint induction gives us the
\hs{map}-\hs{iterate} property.


\begin{comment}
To illustrate why it is important that the property $P$ is admissible,
we shall consider an example
consider the predicate P to be “is not infinite”, and then you can
prove for a lot of functions that they return finite objects. For
instance, define this function:

CHANGE THIS

  Instead do the finite list predicate, and use $\neq
  \hs{False}$. This then servers for an example why inequality
  predicates are not admissible!

\begin{code}
listrec :: ([a] -> [a]) -> [a] -> [a]
listrec i [] = []
listrec i (x:xs) = x : i xs
\end{code}

Then define
$P(f) \Leftrightarrow \fa{x} ``f(x) \w \mathrm{is \w not \w infinite}"$,
and proceed to prove $P(\hs{fix listrec})$ by fixed point induction. The
base case $P(\bot)$ succeeds, since $\bot$ is not infinite, and if we
assume $P(\hs{i})$, we have no problem proving $P(\hs{listrec i})$.
Hence $P(\hs{fix listrec})$, and since \hs{fix listrec} is essentially
a linear identity function on lists, we have ``proved'' that all lists
are finite (but possibly partial).

The error is as promised that $P$ is not admissible: for the sequence
\begin{equation*}
\bot \sqsubseteq
\hs{0:}\bot \sqsubseteq
\hs{0:1:}\bot \sqsubseteq
\hs{0:1:2:}\bot \sqsubseteq
\cdots
\end{equation*}
$P$ holds for all elements but $P$ does not hold for its limit \hs{[0..]}.

\end{comment}

\begin{comment}
\subsection{Mutually Recursive Functions}

You can also mechanically transform mutually recursive functions to be
defined in terms of \hs{fix}. The functions \hs{even} and \hs{odd}
defined below, which determines if a \hs{Nat} is even, and odd,
respectively, are straightforwardly written by mutual recursion:

\begin{code}
even :: Nat -> Bool           odd :: Nat -> Bool
even Z     = True             odd Z     = False
even (S x) = odd x            odd (S x) = even x
\end{code}

To write these functions in terms of fix, as an additional argument,
the take a tuple of ``non-recursive'' copies of themselves.

\begin{code}
evenToFix :: (Nat -> Bool,Nat -> Bool) -> Nat -> Bool
evenToFix (evenUnFix,oddUnFix) Z     = True
evenToFix (evenUnFix,oddUnFix) (S x) = oddUnFix x

oddToFix :: (Nat -> Bool,Nat -> Bool) -> Nat -> Bool
oddToFix (evenUnFix,oddUnFix) Z     = True
oddToFix (evenUnFix,oddUnFix) (S x) = evenUnFix x
\end{code}

Here the prefix \hs{ToFix} means that it is a function subject to be
\hs{fix}-ed, and \hs{UnFix} means that it is the ``non-recursive''
function. The functions above can now be \hs{fix}-ed by giving the
tuple as an argument to both of them:

\begin{code}
even',odd' :: Nat -> Bool
(even',odd') = fix (\t -> (evenToFix t,oddToFix t))
\end{code}

This encoding makes \hs{even'} denotationally equal to \hs{even} and
the same relation hols for \hs{odd'} and \hs{odd}.
\end{comment}

\subsection{Simplification}

The mechanical translations introduced above for self-recursive
functions and mutually recursive functions makes a new function with
an additional argument, the ``non-recursive'' version of itself. By
the translation to FOL that is used, this would introduce a new
argument as a ``function pointer'' and introduce uses of $\appfn$,
which gives unnecessary overhead to the automated theorem
provers.

This is another approach. It avoids introducing these function
pointers and the additional argument to every function. Given a
function $f$ with arguments $\overline{x}$ defined as this:

\begin{equation*}
f \, \overline{x} = e(\overline{x},f)
\end{equation*}

Two new constants are introduced, $\tofix{f}$ and $\unfix{f}$
and this definition:

\begin{equation*}
\tofix{f} \, \overline{x} = e(\overline{x},\unfix{f})
\end{equation*}

\noindent
The empty circle $\unfix{}$ describes that this function is empty
(lacks implementation,) and the filled circle $\tofix{}$ means that
this function has an implementation.

Now we can get a simplified fixpoint schema:

\begin{mathpar}
  \inferrule*
     {
       P(\bot)
       \\
       P(\unfix{f}) \rightarrow P(\tofix{f})
       \\
       P \, \mathrm{admissible}
     }
     { P(f) }
\end{mathpar}

\noindent
The empty $\unfix{f}$ does not have any implementation. But it has
something much better, namely the induction hypothesis. The induction
conclusion is to prove the property for $\tofix{f}$, where the
recursive call to $f$ is replaced with $\unfix{f}$. We do this
simplification since it is better suited for the theorem provers.

\newpage
This also works for several functions at the same time, possibly
mutually recursive:

\begin{mathpar}
  \inferrule*
     {
       P(\bot,\bot)
       \\
       P(\unfix{f},\unfix{g}) \rightarrow P(\tofix{f},\tofix{g})
       \\
       P \, \mathrm{admissible}
     }
     { P(f,g) }
\end{mathpar}

This translation needs to be carried out with some care, since for $f
\, \overline{x} = e(\overline{x},f)$, it is also possible that $f$ is
called in bodies of other functions. These are of two kinds: either
this function is also called from $f$, making it recursive, or another
function which is not called from $f$, but makes use of $f$
anyway. The first example, with a recursive call, the body needs to be
edited so $f$ becomes translated (to $\bot$, $\unfix{f}$ or
$\tofix{f}$), and the second case should use the original $f$. The
transitive clousure of the call graph is calculated, and every
appropriate calls of $f$ are replaced.

\subsection{Erroneous Use of Fixed Point Induction}

The importance that the predicate $P$ is admissible is illustrated in
this example. The function \hs{finite} below returns \hs{True} on
finite lists and $\bot$ on partial lists.

\begin{code}
finite :: [a] -> Bool
finite []     = True
finite (x:xs) = finite xs
\end{code}

Is it possible to use \hs{finite} to prove that \emph{all} lists in
Haskell are either finite or partial? Let the predicate be
$P(f) \Leftrightarrow \fa{xs} f \w xs \neq \hs{False}$, and proceed by fixed
point induction. This is the proof plan with the predicate inlined:

\begin{mathpar}
  \inferrule*
     {
       \fa{xs} \bot \w xs \neq \hs{False}
       \\
       \fa{xs}             \unfix{\hs{finite}} \w xs \neq \hs{False}
               \rightarrow \tofix{\hs{finite}} \w xs \neq \hs{False}
     }
     { \fa{xs} \hs{finite} \w xs \neq \hs{False} }
\end{mathpar}

\noindent
The base case is trivial, as anything applied to bottom is $\bot$ and
$\bot \neq \hs{False}$. The step case also succeeds; trivially if $xs$
is the empty list, by the hypothesis if $xs$ is non-empty, and if $xs$
is bottom for similar reasons as in the base case.

The error is as promised that $P$ is not admissible. See the
$\sqsubseteq$-chain below:
\begin{equation*}
\bot \sqsubseteq
\hs{0:}\bot \sqsubseteq
\hs{0:1:}\bot \sqsubseteq
\hs{0:1:2:}\bot \sqsubseteq
\cdots
\end{equation*}
$P$ holds for all elements in the chain, but $P$ does not hold for its
limit \hs{[0..]}. This example also serves as a proof that inequality
in general is not admissible.


\subsection{Candidate Selection}

Faced with the following property saying that if you drop $n$ elements
from a list the length of this is the same as the length of the
original list minus $n$, which functions should one do fixed point
induction on?

\begin{verbatim}
prop_length_drop :: [a] -> Nat -> Prop Nat
prop_length_drop xs n = length (drop n xs) =:= length xs - n
\end{verbatim}

The answer here is to do fixed point induction on \hs{drop}, and on
\hs{-}. So far no better way to tackle this is used than to try fixed
point induction on all subsets of recursive functions mentioned in the
property.

\subsection{Future Work}

Just as with structural induction, it is also possible to use fixed
point in more than one ``depth'', giving for instance this inference
rule:

\begin{mathpar}
  \inferrule*
     {
       P(\bot)
       \\
       P(f \w \bot)
       \\
       P(x) \wedge P(f \w x) \rightarrow P(f \w (f \w x))
       \\
       P \w \mathrm{admissible}
     }
     { P(\fn{fix} f) }
\end{mathpar}

It is also possible to use such an encoding as in ``Automated depth''
in Section \ref{sec:futind} to let the theorem prover determine the
depth. As an example, the map-iterate property impossible to show with
\hs{map} redefined to \hs{doublemap}, defined below, with ordinary one
depth fixed point induction.

\begin{verbatim}
doublemap :: (a -> b) -> [a] -> [b]
doublemap f []       = []
doublemap f [x]      = [f x]
doublemap f (x:y:zs) = f x : f y : doublemap f zs
\end{verbatim}

\noindent
While \hs{doublemap} is behaviorally equivalent to \hs{map} on total
lists, it makes the induction hypothesis in fixed point induction too
weak.


An issue with the candidate selection is that is some selections are
quite stupid, for instance doing fixating functions on only one side
of the equality. A heuristic to find good candidates would be beneficial.

% Approximation Lemma ---------------------------------------------------------

\section{Approximation Lemma}

The approximation lemma must be considered a standard technique for
proving properties about corecursive programs. Just like fixed point
induction it can be used with functions that are produce infinite
values, like \hs{repeat} and \hs{iterate}, out of abstract values
making structural induction impossible \cite{corecursive}.  The
approximataion lemma supersedes the classical take lemma
\cite{introfp} by being easier to apply and generalize: unlike the
take lemma, it can be applied to equalities of any polynomial
data type \cite{genapprox}. The definitions of \hs{take} and
\hs{approx}: % for lists can be viewed in figure \ref{fig:takeapprox}.

%% This is weird!

%\floatstyle{ruled}
%\newfloat{program}{thp}{lop}
%\floatname{program}{Program}

\note{how to put/float source code side by side?}
%\begin{program}
\begin{verbatim}
take :: Nat -> [a] -> [a]             approx :: Nat -> [a] -> [a]
take Zero    _      = []              approx (Suc n) []     = []
take (Suc n) []     = []              approx (Suc n) (x:xs) = x : approx n xs
take (Suc n) (x:xs) = x : take n xs
\end{verbatim}
%\caption{Definition of \hs{take} and \hs{approx} on lists, with Peano-\hs{Nat}s.}
%\label{fig:takeapprox}
%\end{program}
\note{write this as Haskell code? (those \hs{Nat} are actually $\mathbb{N}$)}

Whereas \hs{take} approximates a list and ends it with \hs{[]},
\hs{approx} ends it with $\bot$ since the \hs{Zero} case is
omitted. The idea of these techniques is then to show that show that
two lists are equal by showing that their prefix or approximation
coincides for all natural numbers. The approximation lemma is given thusly

\begin{equation}
\label{eq:approxeq}
xs \, = \, ys \quad \Leftrightarrow \quad \fa{n \in \mathbb{N}} \hs{approx} \, n \, xs = \hs{approx} \, n \, ys
\end{equation}

Equation \ref{eq:approxeq} quantifies over the \emph{true} natural
numbers, rather than the \emph{polluted} Haskell naturals. Showing an
equality then amounts to a proof by induction over natural numbers,
and the base case for $0$ is always true by reflexivity, as the
approximation is $\bot$.
The right to left implication is (trivially) true by substitution, and
the other direction hinges on the lemma that better and better
approximations form a chain with limit \hs{id}, as illustrated in
Equation \ref{eq:approxchain}.

\begin{equation}
\label{eq:approxchain}
\hs{approx} \, 0 \,
   \sqsubseteq \,
\hs{approx} \, 1 \,
   \sqsubseteq \,
\cdots
   \sqsubseteq \,
\hs{approx} \, n \,
   \sqsubseteq \,
\hs{approx} \, (\hs{Suc} \, n) \,
   \sqsubseteq \,
\cdots
   \sqsubseteq \,
\hs{id}
\end{equation}

The inclusions in \ref{eq:approxchain} are easily given by induction
on natural numbers and the limit by structural induction on lists.
For other polynomial data types, this lemma is established by
the structural induction induced on that data type.
The desired implication is then readily deduced:

\newcommand{\xsys}[2]{#1 \, xs \, #2 & = #1 \, ys #2}
\newcommand{\desca}[1]{  & \hspace{44.5mm}                              \{ \mathrm{#1} \}}
\newcommand{\descra}[1]{ & \hspace{35mm} \Rightarrow     \hspace{4mm} \{ \mathrm{#1} \}}
\newcommand{\descla}[1]{ & \hspace{35mm} \Leftarrow      \hspace{4mm} \{ \mathrm{#1} \}}
\newcommand{\desclra}[1]{& \hspace{35mm} \Leftrightarrow \hspace{4mm} \{ \mathrm{#1} \}}
\begin{align*}
\xsys{\fa{n} \hs{approx} \, n}{}            \\
\descra{limits}                             \\
\xsys{\lub{n} \, (\hs{approx} \, n}{)}      \\
\desclra{continuity \, of \, application}   \\
\xsys{\lub{n} \, (\hs{approx} \, n)}{}      \\
\desclra{Equation \, \ref{eq:approxchain}} \\
\xsys{\hs{id}}{}                            \\
\desclra{definition \, of \, \hs{id}}       \\
\xsys{}{}                                   \\
\end{align*}

\subsection{Example: Mirroring an Expression}

Consider these definitions of a modest but prototypical expression
data type, and its mirroring function:

\begin{verbatim}
data Expr = Add Expr Expr | Value Nat

mirror :: Expr -> Expr
mirror (Add e1 e2) = Add (mirror e2) (mirror e1)
mirror (Value n)   = Value n

prop_mirror_involutive :: Expr -> Prop Expr
prop_mirror_involutive e = e =:= mirror (mirror e)
\end{verbatim}

Two things to notice here is that the type \hs{Expr} does not have a
nullary constructor. Then the \hs{take} lemma would not be usable as
there is no such function over these expressions: the list version
returns the empty list \hs{[]} for the zero case, but there is no such
alternative for \hs{Expr} above. It is important that the limit of
approximations is the identity, and we cannot get this property when
trying to generalise the take lemma.

Furthermore, fixed point induction fails on this property: choosing
either or both occurrences of \hs{mirror} on the right side is
constant bottom for the base case, and the left side is the identity.


We shall now proceed to prove that \hs{mirror} is involutive by the
approximation lemma. The approximation function for \hs{Expr} is
automatically generated, by approximating each self-recursive
constructor, and hence \hs{Value}'s \hs{Nat} is not further approximated:

\begin{verbatim}
approx :: Nat -> Expr -> Expr
approx (Suc n) (Add e1 e2) = Add (approx n e1) (approx n e2)
approx (Suc n) (Value n)   = Value n
\end{verbatim}

\note{Use $\bot$ or bottom in this text?}
Indeed, we also get a third case for $\bot$ which states that the
approximation of $\bot$ is, quite unsurprisingly, $\bot$.
As always in proofs by approximation lemma, we proceed by induction
over natural numbers, and the base case is always trivial: true by
reflexivity as both sides are $\bot$. The step case - which indeed is
the only proof obligation in any proof of this kind - is to prove
this:

\begin{equation*}
\fa{e}  approx \, (\hs{Suc} \, n) \, e = approx \, (\hs{Suc} \, n) \, (mirror \, (mirror \, e))
\end{equation*}

An important property of the induction hypothesis is the universal
quantification of the expression $e$, unlike the fixed natural number
$n$:

\begin{equation*}
\fa{e}  approx \, n \, e = approx \, n \, (mirror \, (mirror \, e))
\end{equation*}

The proof is by case exhaustion. The case for \hs{Value} and $\bot$
are trivial: \hs{mirror} is strict in its first argument, and
mirroring \hs{Value} twice is the identity, so these cases are both
true by reflexivity. The \hs{Add} case is ever so slightly more
elaborate, and with names shortened to \hs{app} and \hs{mir} the
reasoning is as follows:

\note{$(\hs{Suc} \, n)$ or $(n + 1)$?}
\newcommand{\Adds}[2]{\hs{Add} \, #1 e_1 #2 \, #1 e_2 #2}
\newcommand{\Approxn}[0]{\hs{app} \, n \,}
\newcommand{\ApproxSucn}[0]{\hs{app} \, (\hs{Suc} \, n) \,}
\newcommand{\mirmir}[0]{\hs{mir} \, (\hs{mir} \, }
\begin{align*}
\faa{e_1}{e_2}&  \ApproxSucn (\Adds{}{})  = \ApproxSucn (\mirmir \Adds{}{} ))                                                                   \\
                                                                                 \desclra{\defof{\hs{mirror}}}                                   \\
\faa{e_1}{e_2}&  \ApproxSucn (\Adds{}{})  = \ApproxSucn (\Adds{(\mirmir}{))})                                                                    \\
                                                                                \desclra{\defof{\hs{approx}}}                                    \\
\faa{e_1}{e_2}&  \Adds{(\Approxn}{)}      = \Adds{(\Approxn(\mirmir}{)))}                                                                        \\
                                                                                \desclra{induction \, hypothesis \, twice \, (e_1 \, and \, e_2)} \\
\faa{e_1}{e_2}&  \Adds{(\Approxn}{)}      = \Adds{(\Approxn}{)}                                                                                  \\
                                                                                \desca{reflexivity}                                              \\
\end{align*}

\subsection{Approximation Lemma is Fixpoint Induction}

This technique is already simple and widely applicable, however it can
further be simplified. Implementing it in this form relies on the
auxiliary structure of Peano natural numbers which also needs to be
added to the theory. This can be removed by the observation that is
\note{What is a better name for these functions than \hs{id}?}
expressed as fixed point induction over a recursive form of \hs{id}.

\begin{verbatim}
id :: [a] -> [a]              id :: Expr -> Expr
id []     = []                id (Add e1 e2) = Add (id e1) (id e2)
id (x:xs) = x : id xs         id (Value n)   = Value n
\end{verbatim}

Each \hs{id} function constructed in this way is indeed an identity
function, equivalent to the implementation \hs{id x = x} if we
disregard time and space complexity. Now, to prove

\begin{equation*}
e_1 \, = \, e_2
\end{equation*}

we simply use fixed point induction to prove

\begin{equation*}
\hs{id} \, e_1 \, = \, \hs{id} \, e_2
\end{equation*}

where \hs{id} is a such a specialized recursive identity function over
the data type of the equality. With the same translation of recursive
functions as in the fixed point section, the axioms for \hs{id} for, say
lists becomes:

\begin{align*}
            & \tofix{id}(\hs{[]}) \,   = \, \hs{[]}                                                           \\
\faa{x}{xs} & \tofix{id}(x\hs{:}xs) \, = x\hs{:}\unfix{id}(xs)                                                \\
\fa{xs}     & \tofix{id}(xs)           = \bot  \leftarrow    xs \neq [] \wedge xs \neq head(xs)\hs{:}tail(xs) \\
\end{align*}

The step case in induction $P(\unfix{id}) \rightarrow P(\tofix{id})$
is then exactly the same strength as the approximation lemma with
natural numbers.

\subsection{Implementation} The implementation of approximation lemma
was the simplest to implement, after definitional equality. First,
from the type signature from the property, such a recursive identity
function as above is generated for the data type of the equality. Then
the lemma $P(\unfix{id})$ is added to the theory, with $P$
instantiated to the universally quantified equality, and the
conjecture is $P(\tofix{id})$. The base case need not be proven,
$P(\bot)$ is always true since it evaluates to $\bot=\bot$.

\subsection{Future Work: Total Approximation Lemma}

It is also possible to adjust the approximation lemma to prove
properties that are true for total and potentially infinite objects,
but false for objects with partial values. One such property is the
idempotence of \hs{nub}. Here is a version of \hs{nub} on booleans,
and the said property about it:

\begin{verbatim}
nub :: [Bool] -> [Bool]
nub (True :True :xs) = nub (True:xs)
nub (False:False:xs) = nub (False:xs)
nub (x:xs)           = x:nub xs
nub _                = []

prop_nub_idem :: [Bool] -> Prop [Bool]
prop_nub_idem xs = nub (nub xs) =:= nub xs
\end{verbatim}

Consider the list \hs{True:False:}$\bot$. One application of \hs{nub}
gives \hs{True:}$\bot$, and two gives $\bot$ immediately. In spite of
this, the property is a truism for finite as well as infinite lists,
provided that there are not bottoms.

The way to enable the approximation lemma to prove such properties is
to add predicates of totality, and add axioms like the following to
the theory:

\note{How to make $Total$ look like one word? The kerning for $T$ in
  $Total$ looks incorrect}
\begin{align*}
            & \neg \, Total(\bot) \\
            & Total(\hs{[]}) \\
\faa{x}{xs} & Total(x) \wedge Total(xs) \rightarrow Total(x\hs{:}xs) \\
\fa{xs}     & Total(xs) \rightarrow Total(\hs{nub} \, xs) \\
\end{align*}

To use fixed point induction \hs{Total} needs to be an admissible
predicate. But this is indeed so, for each model with infinite lists,
we have that all such lists are total.  However, that a total argument
to \hs{nub} yields a total value also needs to be proven. Furthermore,
the totality property for the cons case of lists is debatable: should
the consed element also be total?

The proof obligation is then:

\begin{mathpar}
  \inferrule*
     {
        \fa{xs} Total(xs) \Rightarrow \unfix{id}(\hs{nub} (\hs{nub} \, xs)) = \unfix{id}(\hs{nub} \, xs)
     }
     {
        \fa{xs} Total(xs) \Rightarrow \tofix{id}(\hs{nub} (\hs{nub} \, xs)) = \tofix{id}(\hs{nub} \, xs)
     }
\end{mathpar}

It seems like you have to add
$Total(x) \rightarrow Total(\tofix{id}(x))$ and
$Total(x) \rightarrow Total(\unfix{id}(x))$ to the theory, too.

\subsection{Uncategorized / old}

After fixed point induction since it is an easy consequence of fixed
point induction, and how we removed the auxiliary structure of natural
numbers. This makes it equivalent to fixed point induction of id on
both sides (or does it?)

Approximation lemma is a generalization of the take lemma, and its
first form is used for properties about infinite and partial lists,
but it is easily generalized to other recursive data types.
In particular, all polynomial data types — for example, any
sum-of-products data type — can be defined in this way.
This result generalizes
to mutually recursive, parameterised, exponential and nested datatypes, but
for simplicity we only consider polynomial datatypes in this article.
\cite{genapprox}

How to use approximation lemma on exponential, mutually recursive and
nested data types? This would be nice.

%% End of Technical Part ------------------------------------------------------


\chapter{Discussion}

\section{Results}

Test suite

\section{Future work}

QuickSpec (IsaCosy)

Finite/Total domains and what is a terminating function, anyway?
Predicates, functions or fast and loose reasoning \cite{fastandloose}

Lemmas

Implications

Finite fixed point induction on terminating functions

Fix-point induction with id on variables

Recursion-induction

Fix-point induction (and approximation lemma) induction depth machines

Cover a larger part of Haskell (type classes, do-notation, list
comprehensions, records, pattern bindings, irrefutable patterns, seq
and bang patterns)

Min-predicate for finite models and quicker proof searches

Since we will need to prove termination for functions, how would one
prove non-termination (for some inputs)? Then we could get more
completeness: this function with these inputs would be $\bot$.

\section{Conclusion}

%% References -----------------------------------------------------------------

\bibliographystyle{apalikeurl}
\bibliography{masterbib}
\end{document}

%\end{comment}