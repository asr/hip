\documentclass{report}
\usepackage[utf8]{inputenc}
\usepackage{amssymb}
\usepackage{amsmath}
\usepackage{graphicx}
\usepackage[margin=1.5in]{geometry}
\usepackage{parsetree}
\usepackage{verbatim}
\usepackage{subfig}
\usepackage{wrapfig}
\usepackage{tipa}
\usepackage{textcomp}
%\usepackage{courier}

\begin{document}

\newcommand\note[1]{\mbox{}\marginpar{\footnotesize\raggedright\hspace{0pt}\emph{#1}}}

\newcommand\hs[1]{\verb~#1~}
\newcommand\fn[1]{\mathrm{#1}}

%% Title ----------------------------------------------------------------------
\title{Proving Equational Haskell Properties Using ATPs}
\author{Dan Rosén}
\maketitle

%% Abstract -------------------------------------------------------------------

\newpage
\abstract{
In this work we prove a lot of nice equational Haskell properties by
a first order logic translation and using automated theorem provers.
The results are astounding, writing correct Haskell is nowadays a
breeze.
}

%% Acknowledgements -----------------------------------------------------------

\newpage
\pagestyle{empty}
\section*{Acknowledgments}
\vspace{5mm}
ACK, ACK, ACK, ACK, ACK, ACK, NAK

%% Table of Contents ----------------------------------------------------------

\newpage
\tableofcontents
\addtocontents{toc}{\protect\thispagestyle{empty}}

\newpage
\setcounter{page}{1}

\chapter{Introduction}

Induction and Haskell

\section{Outline}

Explain the outline of the report

\chapter{Background}

This work's main areas, Haskell, FOL, ATPs and some basic DT.

\section{Haskell}

Describe some of the features of Haskell, types, lazy evaluation,
bottoms, pattern-matching, data types, type classes.

Haskell is declarative and side-effect free which makes it easy
to model in logic in comparison to imperative effectful programming languages.

\section{First Order Logic}

Connectives, quantifiers, predicates, functions and constants.
Derivation rules and models.

\section{FOL and Automated Theorem Provers}

Use derivations or models to deduce absurdity.

\section{Induction}

Some background of induction and how it is present in well-known
theories like PA, ZFC, MLTT and CoC.

\section{Domain Theory}

CPOs, monotonicity, continuity and admissible predicates.

\section{Related Work}

Isabel, Sledgehammer, Rippling and other techniques
(Productive Use of Failure)

Other proof assistants as Coq

Agda and dependent types, Agda and ATPs by Peter, Bove, Sicard
\cite{agdaatp} and Dependently Typed Programming Based on ATPs

Zeno

Co-recursive techniques \cite{corecursive}, Hinze's papers, Generic
Approximation lemma, or have this in other chapters

%% Techical Part --------------------------------------------------------------

\chapter{Technical Part}

Implementation technicalities.

\section{Translation to FOL}

The goal of this translation is to enable equational reasoning of
Haskell programs, in FOL. It is here referred to as a translation, but
it could also in be regarded as compilation.

Equational reasoning is traditional in proving corrected of Haskell
programs, but it assumes that a simple denotational semantics exists,
and there is not even a formal semantics for the language
\cite{chasingbot}.

% Whoa, much in this section
\subsection{Data types}

Haskell's data types and its consequence pattern matching is
translated by giving each constructor a function or constant symbol.

Consider this examples of binary trees in Haskell style:

\begin{verbatim}
data Tree a = Empty | Branch (Tree a) a (Tree a)
\end{verbatim}

We get one constant $\fn{Empty}$ for \hs{Empty}, and one trinary
function $\fn{Branch}$ for \hs{Branch}. Normally, it is custom
to write functions in logic with lowercase letters, but this
convention is disregarded here. The values of these constructors are
distinct, so the following disjointedness axioms are added to the
theory of a Haskell program with this data type.

\begin{align*}
\fn{\bot} &  \neq \fn{Empty}\\
 \forall \, l \,  x \,  r \,  . \,  \fn{\bot} &  \neq \fn{Branch}(l,x,r)\\
 \forall \, l \,  x \,  r \,  . \,  \fn{Empty} &  \neq \fn{Branch}(l,x,r)\\
\end{align*}

Notice that in Haskell, each data type also has an extra value, bottom, which
can come from the functions \hs{undefined} or \hs{error}, as well as
irrefutable pattern matches but also non-terminating programs. The constant
$\bot$ captures this notion and is naturally also distinct from the values
of \hs{Empty} and \hs{Branch}.

\subsection{Injective constructors and projections}

We also want injectivity of constructors, for example if we have the
cons constructor, \hs{:} in Haskell, and \hs{x:xs = y:ys} then
\hs{x = y} and \hs{xs = ys}. As we will see later, we also want the
projections of the left and right sub tree, and the value in a Branch.
It turns out that the projections imply the injectivity. For the Branch
constructor of the Tree example, we get the following projections:

\begin{align*}
 \forall \, l \, x \, r \, . \,  \fn{Branch_{0}}(\fn{Branch}(l,x,r)) &  = l\\
 \forall \, l \, x \, r \, . \,  \fn{Branch_{1}}(\fn{Branch}(l,x,r)) &  = x\\
 \forall \, l \, x \, r \, . \,  \fn{Branch_{2}}(\fn{Branch}(l,x,r)) &  = r\\
\end{align*}

These ternary functions are bluntly named by indexing on the projected
coordinate, but could just as well have more descriptive names as
$\fn{left}$, $\fn{value}$ and $\fn{right}$.

Now, these projections imply injective constructors. Assume we have
$\fn{Branch}(l,x,r) = \fn{Branch}(l',x',r')$ then the first
projection, $\fn{Branch_0}$, gives us that $l=l'$. Analogously,
and the second and the third give $x=x'$ and $r=r'$, respectively.

\subsection{Translation of functions}

Translating functions that do not use pattern matching is
straightforward. Consider this Haskell definition of a singleton tree:

\begin{verbatim}
leaf :: a -> Tree a
leaf x = Branch Empty x Empty
\end{verbatim}

We simply introduce a new function symbol, $\fn{leaf}$, and due
to referential transparency we can turn the definition into an
equality with quantified variables:

\begin{align*}
 \forall \, x \, . \, \fn{leaf}(x) &  = \fn{Branch}(\fn{Empty},x,\fn{Empty})\\
\end{align*}

\subsection{Pattern matching}

Let's now consider \hs{leaf}'s partial inverse, \hs{top} that
yields the top element of the tree if there is one, or is undefined
otherwise:

\begin{verbatim}
top :: Tree a -> a
top (Branch _ x _) = x
\end{verbatim}

The function call of \hs{top Empty} would yield a run time error
since this pattern is not covered, equivalent to an undefined
value. Indeed, an equivalent formulation would be to cover the
\hs{Empty} case by \hs{undefined} or a helpful message from
\hs{error}, but the run time error is still remains, and this
behavior is modeled by the $\bot$ value in the theory. Thus the
translation to an equality is as follows: if the argument is
constructed with \hs{Branch}, it is equivalent to the top value,
otherwise, it is $\bot$:

\begin{align*}
 \forall \, l \, x \, r \, & . \, \fn{top}(\fn{Branch}(l,x,r) = x\\
 \forall \, t \, & . \, t \neq
p \fn{Branch}(\fn{Branch_{0}}(t),\fn{Branch_{1}}(t),\fn{Branch_{2}}(t))
 \rightarrow \, \fn{top}(t)  = \fn{\bot}\\
\end{align*}

Here the projections functions come in handy. Indeed, an equivalent
but skolemized formulations of the second formulas are:

\begin{align*}
 \forall \, t \, & . \neg (\exists \, l \, x \, r . \, t =
 \fn{Branch}(l,x,r))
 \rightarrow \, \fn{top}(t) = \bot \\
 \forall \, t \, & . (\forall \, l \, x \, r . \, t \neq
 \fn{Branch}(l,x,r))
 \rightarrow \, \fn{top}(t) = \bot \\
\end{align*}

But as we saw earlier, projections also imply injectivity so this is
the approach used here but the choice is of little importance (or is
it? benchmark!!)

\subsection{Overlapping patterns}

Overlapping patterns need to be removed, otherwise we could easily get
an inconsistent theory, consider

\begin{verbatim}
overlap :: Bool -> Bool
overlap True = True
overlap True = False
\end{verbatim}

Certainly, we cannot translate this to
\begin{align*}
\fn{overlap}(\fn{True}) & = \fn{True} \\
\fn{overlap}(\fn{True}) & = \fn{False} \\
\forall \, b \, . \, b \neq True & \rightarrow \fn{overlap}(b) = \bot \\
\end{align*}

Transitivity of equality then yields $\fn{True} = \fn{False}$,
and this together with the axioms of disjoint constructors gives a
contradiction.

In Haskell, pattern matching is done from top to bottom of the
definition, making the second match of True to never occur. Thus, the
translation to FOL also removes all subsequent patterns that are
instances of any pattern above.

\subsection{Nested patterns and bottoms}

The translation also handles patterns in more than one depth. At every
location in a pattern where a constructor is matched against, a
pattern with bottom at that spot is also added, defined to
bottom. This definition that balances a tree to the left is defined
with pattern matching on depth two:

\begin{verbatim}
unbalance :: Tree a -> Tree a
unbalance (Branch (Branch l x r) y r') = unbalance (Branch l x (Branch r y r'))
unbalance (Branch l x r)               = Branch l x (unbalance r)
unbalance Empty                        = Empty
\end{verbatim}

If we could see the bottoms in Haskell, the definition would look like this:

\begin{verbatim}
unbalance :: Tree a -> Tree a
unbalance (Branch (Branch l x r) y r') = unbalance (Branch l x (Branch r y r'))
unbalance (Branch Bottom _ _)          = Bottom
unbalance (Branch l x r)               = Branch l x (unbalance r)
unbalance Empty                        = Empty
unbalance Bottom                       = Bottom
\end{verbatim}

And such an addition of bottoms is made by the translation.
\note{This writing actually discovered a bug. Report the solution and how it was tested}

\subsection{Guards}

Guards are not much of a complication. Either the guard expression is
\hs{True}, then that branch is picked. If the expression returns
bottom, then for this argument, the function is bottom. Care needs to
be taken when looking ``upwards'' the branches, to not collide with
the guards.
\note{Add example}

\subsection{Functions as arguments}

In Haskell, functions readily take other functions as arguments, and
functions can also be partially applied. To get the same behavior in
logic, each function gets a \emph{function pointer}, and a new binary
function is added to the language, written infix with $\fn{@}$.
\note{Should $\fn{@}$ be written infix? $\fn{app}$ prefix is also a viable option}
For instance, the if there is a binary function plus then a constant
called plus-ptr is added to the theory and this axiom:

\begin{equation*}
\forall \, x \, y \, . \, \app{(\app{\fn{plus.ptr}}{x})}{y} = \fn{plus}(x,y)
\end{equation*}

When a function is only partially applied, or a function argument is
applied, $\, @ \, $ is used. Consider the Prelude function \hs{iterate}

\begin{verbatim}
iterate :: (a -> a) -> a -> [a]
iterate f x = x : iterate f (f x)
\end{verbatim}

It is translated with $\, @ \, $ in the following way, with \hs{:} written infix:

\begin{equation*}
\forall \, f \, x \, . \, \fn{iterate}(f,x) = x : \fn{iterate}(f,\app{f}{x})
\end{equation*}

Should a function not get all its arguments, appropriate use of $\, @ \, $ is
added, as in this function which increments all elements of the list
by one using \hs{map}:

\begin{verbatim}
incr = map (plus one)
\end{verbatim}

As \hs{incr} is also written point-free or eta-reduced, \hs{map} is
only partially applied. This is the translated axiom:

\begin{equation*}
\fn{incr} = \app{\fn{map.ptr}}{(\app{\fn{plus.ptr}}{\fn{one}})}}
\end{equation*}

If \hs{incr} is applied to an argument $xs$, then \hs{incr} is applied
to more arguments then it takes, so we add $\, @ \,$ so the
corresponding formula becomes $\app{\fn{incr}}{xs}$, and by equational
substitution from the definition of $\fn{incr}$ we get
$\app{(\app{\fn{map.ptr}}{(\app{\fn{plus.ptr}}{\fn{one}})})}{xs}$ and
the axiom of $\fn{map.ptr}$ then equals this to
$\fn{map}(\app{\fn{plus.ptr}}{\fn{one}},xs)$.

\subsection{The intermediate language}

Everything becomes top level definitions, \hs{let} and \hs{where} are
floated to the top, with the free variables added as extra parameters,
as well as lambda functions, the same approach was taken for extra or
nested case-expressions. Extra care needs to be taken for mutually
recursive \hs{let} and \hs{where} functions, which potentially need's
each other's free variables as arguments.
\note{Incomplete, add examples}

Functions do not do any pattern matching on their arguments directly,
but in a case statement that is the entire body of the function. The
branch expressions are just constructor or function application.
\note{Add BNF}

Type-signatures and data types are also supported since this
information is needed for different proof techniques.

\subsection{Haskell coverage}

Haskell is a big language, and translating it all in one go is a
daunting task. Therefore, some restrictions were settled to be able to
focus on proving rather than translating.
\note{Maybe move this to background?}
The goal was to add enough of the Haskell language to enable to prove
interesting properties, but much of the widely available sugar in
Haskell was omitted since it does not add extra
expressibility. Therefore there is no support for
list comprehensions, do-notation, pattern bindings and type classes .

A more serious omitted feature is the lack of built-in types like \hs{Int},
\hs{Integer}, \hs{Double}, \hs{Char}, etc.

(Higher-kinded type variables are currently not supported)

\subsection{Uncategorized}

\begin{itemize}

\item $\checkmark$ Describe (and motivate here?) the intermediate language

\item $\checkmark$ Pattern-matching and bottoms

\item $\checkmark$ Higher order functions and function pointers

\item $\checkmark$ Axioms of disjointedness

\item $\checkmark$ Axioms of projections and injectivity of constructors

\item $\checkmark$ Extensional equality and application of bottom

\item Remove unnecessary definitions for a given proof

\end{itemize}

\section{Proof techniques}

\note{Need to decide some running example}

To prove things using this technique, properties are entered in the
Haskell source code. A small prelude called \hs{AutoPrelude} needs to
be imported that gives access to the relevant functions. One example
is the associativity of list concatenation:

\begin{verbatim}
import AutoPrelude

prop_app_assoc :: [a] -> [a] -> [a] -> Prop [a]
prop_app_assoc xs ys zs = xs ++ (ys ++ zs) =:= (xs ++ ys) ++ zs
\end{verbatim}

The infix function \hs{=:=} comes from the imported library, as well
as the type constructor \hs{Prop}. The type signature cannot be
omitted as this is used for instance to see which kind of induction we
will use.

Running the program is simple. Just save the file as for instance
\hs{ListProps.hs} and run

\begin{verbatim}
autospec ListProps.hs
\end{verbatim}

and the program will report if it was provable or not, and which
techniques succeeded.

These properties are also QuickCheck-testable, so you can run the
normal \hs{quickCheck} function on them, given that there are
relevant \hs{Eq} and \hs{Arbitrary} instances provided.

Let's now take a look what different proof methods that are supported.
\note{This is too informal}

\subsection{Definitional Equality}

Some properties cannot or need not use induction or some more
sophisticated technique, since they are true by definition. Examples
are properties for fully polymorphic functions such as the definition
of \hs{id} in the SK-calculus, here

\begin{verbatim}
s f g x = f x (g x)
k x y = x
id x = x

prop_skk_id :: Prop (a -> a)
prop_skk_id = s k k =:= id
\end{verbatim}

Then, the generated conjecture is simply

\begin{equation*}
\app{ (\app {\ptr{s}} {\ptr{k}} )
    }{\ptr{k}}} = \ptr{id}
\end{equation*}

Another example where this is useful is to prove functor and monad
laws for the environment monad.

\subsubsection{Extensional Equality and seq}

To prove the previous property we also need to have extensional
equality, postulated with the following axiom

\begin{equation*}
\faa{f}{g} (\fa{x} \app{f}{x} = \app{g}{x}) \rightarrow f = g
\end{equation*}

which identifies function pointers and functions composed with $@$.
One problem with extensional equality in Haskell, is that the presence
of \hs{seq} breaks it. \hs{seq} is a built in function with the
following behavior:

\begin{verbatim}
seq :: a -> b -> b
seq bottom b = bottom
seq a      b = b
\end{verbatim}

Where \hs{a} is any other value than $\bot$. With \hs{seq} it is
possibly to distinguish between these two functions, which otherwise
are observationally equal:

\begin{verbatim}
f = bottom
g = \x -> bottom
\end{verbatim}

Because \hs{seq f ()} evaluates to $\bot$, and \hs{seq g ()} evaluates
to \hs{()}, but on any argument \hs{f} and \hs{g} gets, they both
return $\bot$. Here we also need an extra axiom, which says that
anything applied to $\bot$ is $\bot$:

\begin{equation*}
\fa{x} \app{\bot}{x} = \bot
\end{equation*}

However, \hs{seq} is the only function that can do this, so we will
ignore its presence in Haskell.
\note{This could be added as a switch \hs{--enable-seq}, which removes
  extensional equality}

Furthermore, if we assume we have extensional equality we also have
the property that \hs{Prop (a -> b)} is equivalent to
\hs{a -> Prop b}, by letting the property have an extra argument that
is applied to the left and right hand side of the equality. This has
two benefits, firstly it can trigger other proof methods should \hs{a}
or \hs{b} be concrete types (the former for induction and the latter
for approximation lemma), and secondly it does not need to use the
extensionality axiom introduced above which is costly and confusing
for the theorem provers tested.
\note{Add some support for this claim. For instance, SPASS seems to be
  utterly horrible at this, but Eprover and Vampire are OK}

\subsubsection{Concrete Concerns}

This only works on non-concrete types because of the way bottoms are
added. One example when this is a problem with is this plausible
definition of boolean or

\begin{verbatim}
False || a = a
True  || _ = True
\end{verbatim}

Then an extra branch is added that matches everything else that goes
to $\bot$. This is of efficiency reasons: imagine if we had only used
just a few constructors of a data type with hundreds of constructors,
then we do not want to write a new line with all those constructors to
$\bot$. A property that should indeed hold (regardless of presence of
partial values or not) is \hs{x || False == x} for all
\hs{x}. However, for models with another point, say $\top$, we get
that $\top \, \fn{||} \, \fn{False} = \bot$, and the conjecture is
counter satisfiable.

\note{This blends a bit with future work. One solution is just to add
  a predicate describing the elements of the type. Hopefully I will have time
  to implement this so it need not be in future work. Further, it is
  pretty sloppily written}

\subsection{Structural induction}

Simple induction to more complex structural induction and its
implementation.

\subsection{Fixed point induction}

Describe how and why it works

Fixed point induction on several functions and subsets of the existing
functions

\subsection{Approximation lemma}

After fixed point induction since it is an easy consequence of fixed
point induction, and how we removed the auxiliary structure of natural
numbers. This makes it equivalent to fixed point induction of id on
both sides (or does it?)


%% End of Technical Part ------------------------------------------------------



\chapter{Discussion}

\section{Results}

Test suite

\section{Future work}

QuickSpec (IsaCosy)

Finite/Total domains and what is a terminating function, anyway?
Predicates, functions or fast and loose reasoning \cite{fastandloose}

Lemmas

Implications

Finite fixed point induction on terminating functions

Fix-point induction with id on variables

Recursion-induction

Fix-point induction (and approximation lemma) induction depth machines

Cover a larger part of Haskell (type classes, do-notation, list
comprehensions, records, pattern bindings)

Min-predicate for finite models and quicker proof searches

\section{Conclusion}

%% References -----------------------------------------------------------------

\bibliographystyle{apalikeurl}
\bibliography{masterbib}
\end{document}
