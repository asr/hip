\chapter{Future work}
\label{ch:future}

The chapter describes the limitations of our approach and ideas how to
get around them.

\section{Lemmas}

For many proofs, it is essential to use some lemma to reach the
goal. For commutativity of plus, we already touched on this aspect, it
needs the left identity as well as the $s(n) + m \eq n + s(m)$
lemma. Furthermore, to prove associativity of multiplication you need
at least associativity of addition.

Implementing addition of lemmas for properties that always hold, i.e,
for also partial and infinite values is straightforward; you can just
add the universally quantified property to the theory. The problem is
when it only holds for finite or total values: you will need FOL
predicates or functions to describe exactly when the property
holds. Research has been made how to use such predicates
\cite{sortMonotonicity}, \cite{polyMonotonicity}, but it is not
decided yet on which way to encode types. Furthermore, you will need
to prove that functions return finite or total values on such
input. For instance, append for lists gives a total list if both input
lists are total, but on the other hand, if the result list is total
but potentially infinite, we can only draw the conclusion that the
first list is total, since that might be the infinite one. So, to
prove properties that does not always hold you will need to prove a
lot more properties about your functions.

Another way to reason about this is to assume that the undefined value
does not exist, and this could be deemed as morally correct
\cite{fastandloose}. Then you cannot use fixed point induction but you
can still use structural induction.

\note{Discuss this with Moa and get references}
Another aspect of this is where to get the lemmas from. In the Isabel
induction and rippling community, there are concepts such as critics,
which makes lemmas and generalizations when the term rewriting
heuristic rippling fails, and lemma calculation, which tries to
syntactically analyze the problem before proving in order to figure
out suitable lemmas. Another approach is to do property speculation,
implemented for Isabell \cite{isacosy} and Haskell \cite{quickspec},
where you simply tell the QuickSpec program all functions in your
program, and it tries to find equality properties by creating small syntax
trees of the functions. It would be very interesting to see how useful
such properties are as lemmas.

The third aspect is how should the user inform the program which
lemmas are needed? One solution is to allow the user to annotate the
source code where the properties are entered which lemmas might be
appropriate to use. Another is to first try to prove all properties,
and for those which did not turn out to be a theorem, add all the
succeeded properties as a lemma and iterate until you reach a fixed
point. This could probably be optimized: if a lemma only concerns
functions that are not relevant for a given property, it is
unnecessary to add it to the theory.

%\section{Material Implication}
%
%Zeno has support for properties with implications,

%\section{Proving non termination}
%
%Since we will need to prove termination for functions, how would one
%prove non-termination (for some inputs)? Then we could get more
%completeness: this function with these inputs would be $\bot$.

\section{Other Techniques}

Each of the proof methods have their own future work chapter, but it
would be nice to have some finite fixed point induction on terminating
functions and a finite approximation lemma. You can also do as with
the approximation lemma which became just a fixpoint induction over a
structural identity function, but instead of putting one of each side
of the equality, you can do it on the individual variables or sub
trees of an expression: this would be like structural induction
encoded as approximation lemma.

Another technique is recursion induction \cite{recind}, where you
prove that two functions are equal by asserting that one of them
fulfills the same equations as the other.

\section{Faster Proof Searches via Predicates}

It is possible to add annotations in the equations generated for
functions to make the theorem prover not unroll unnecessary
definitions and regard these equalities more like definitions, so they
do not rewrite from the wrong direction. One way is to add a
min-predicate (named so for historical reasons), that can also
sometimes lead to finite models.

