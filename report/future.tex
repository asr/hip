\chapter{Future work}
\label{ch:future}

The chapter addresses some limitations of our approach and ideas how
to get around them.

\section{Pattern Matching Re-Revisited}
\label{sec:rerevisited}

The current translation of pattern matching is described in Section
\ref{sec:patternsrevisited}. One of the strengths of it is the
translation of wild patterns. Consider this function for equality of a
data type with three elements:

\begin{code}
data Tri = A | B | C

equal :: Tri -> Tri -> Bool
equal A A = True
equal B B = True
equal C C = True
equal _ _ = False
\end{code}

\noindent
The translation adds two more cases that go to bottom, one where the
first argument is bottom, and one where the second one is bottom. All
other values that are not the same \hs{Tri} and not bottom go to
\hs{False}.

If there for models of this function which includes
another value $\star$, other values than those of this data type and
bottom, must have $\fn{equal}(\star,x) = \fn{false}$, for all non
bottom $x$, including $\star$. This is not a problem per se, since the
Haskell source is well typed, and \hs{equal} will not be applied to
anything but \hs{Tri}s and bottoms.

One weakness of that approach is that two functions that are
extensionally equivalent in Haskell are not in the generated theory,
when applied to non sense arguments. An example of this behaviour is
these two implementations of the boolean or function in Figure
\ref{code:or}.

\begin{figure}[h!]
\centering
\begin{minipage}[b]{6cm}
\begin{code}
or :: Bool -> Bool -> Bool
or False b = b
or True  _ = True
\end{code}
\end{minipage}
\hspace{10pt}
\begin{minipage}[b]{6cm}
\begin{code}
or' :: Bool -> Bool -> Bool
or' False b = b
or' _     _ = True
\end{code}
\end{minipage}
\caption{Two definitions of boolean or, \texttt{or} and \texttt{or'}
\label{code:or}
}
\end{figure}

Let us analyse the models of this program in Figure \ref{code:or} with
an additional value $\star$ in the domain. The translation makes
$\fn{or}(\star,\fn{true}) = \bot$, since there is a wild pattern that
goes to $\bot$. For the other function we have
$\fn{or'}(\star,\fn{true}) = \fn{false}$, as the wild pattern already
goes to \fn{false}. Because of this differences, the approximation
lemma cannot prove this property. While this is a simple example, it
is easy to imagine more complex cases where different pattern matching
techniques makes no difference to the Haskell program, but to this
translation.

This suggests that wild patterns should be expanded to pattern for all
their constructors, and add a match any pattern that goes to
bottom. This make the the meta theorem
$\faa{x}{y} \fn{or}(x,y) = \fn{or'}(x,y)$ consistent with rest of the
theory.  Further, it would be provable by the approximation lemma, and
in more complex cases with recursive functions, by fixed point
induction. This translation is also assumed to be a little easier to
implement. The down side is that functions such as \hs{equal} above
would generate $O(n^2)$ axioms, where $n$ is the number of constructors for
the data type.

\section{Lemmas}

For many properties, especially more advanced ones, it is crucial to
be able to use lemmas to obtain a proof. The introduced proof
techniques in this theses are just not strong enough. For structural
induction, it is possible to do induction in more than one depth. This
sometimes make the hypotheses equally strong as the required lemmas,
as in the example of plus commutativity of natural numbers in Section
\ref{sec:genind}. But this approach fails for properties about
multiplication: it is essential to use lemmas about addition.

The concept of adding lemmas is of course quite simple. Assume your
program has two properties \hs{prop\_a} and \hs{prop\_b}, and the
second needs the first as a lemma. If \hs{prop\_a} succeeds, then the
should program just add that property to the theory generated for
\hs{prop\_b}, and try it again. Unfortunately, adding a property to a
theory needs to be carried out with care. For properties that hold for
infinite and partial values it is actually a bit simpler than for
those that are only true for finite values. These settings are
addressed one by one below.

\subsection{Lemmas from Theorems}

One example of a Theorem, a property that holds for partial and
infinite values, is the right identity of \hs{||}, boolean or:

$$\fa{x} x \w \hs{||} \w \fn{false} = x$$

\noindent
Adding this meta theorem to the theory would make it
inconsistent. Again, in a model with an extra value $\star$, we have
that $\star \w \hs{||} \w \fn{false} = \bot$. However, we can create a
function that forces a value to be a \hs{Bool} or $\bot$ as this:

\begin{align*}
\rom{1} \qquad &&        & \fn{force}(\fn{true})  && = \fn{true} \\
\rom{2} \qquad &&        & \fn{force}(\fn{false}) && = \fn{false} \\
\rom{3} \qquad && \fa{x} & \fn{force}(x)          && = \bot \vee x = \fn{false} \vee x = \fn{true}
\end{align*}

The theory would be consistent with this reformulation of the meta theorem:

$$\fa{x} \fn{force}(x) \w \hs{||} \w \fn{false} = \fn{force}(x)$$

It is easy to generalise $\fn{force}$ beyond booleans. Make it a
binary function with first argument a description of the type. Each
simply kinded type would be given a constant, and higher-kinded types
functions, an example would be $\fn{list}(\alpha)$ for lists of $\alpha$:

\begin{align*}
\rom{1} \qquad && \fa{\alpha}          & \fn{force}(\fn{list}(\alpha),\fn{nil})        && = \fn{nil} \\
\rom{2} \qquad && \faaa{\alpha}{x}{xs} & \fn{force}(\fn{list}(\alpha),\fn{cons}(x,xs)) && = \fn{cons}(\fn{force}(\alpha,x),\fn{force}(\fn{list}(\alpha),xs) \\
\rom{3} \qquad && \faa{\alpha}{xs}     & \fn{force}(\fn{list}(\alpha),xs)              && = \bot  \\
               &&                      & \multicolumn{3}{l}{\vee xs = \fn{nil} \vee xs = \fn{cons}(\fn{cons_0}(xs),\fn{cons_1}(xs)}
\end{align*}

Using functions and predicates to witness type information has been
studied by \cite{sortMonotonicity} and by \cite{polyMonotonicity},
with ideas how to encode type tags efficiently.

Proofs by definitional equality would also benefit from such tags, see
the current limitations in section
\ref{sec:concreteconcerns}. Furthermore, the extensional equality of
the different implementations of the or function in Figure
\ref{code:or} is not provable with approximation lemma with the
current translation, as described in Section
\ref{sec:rerevisited}. However, stated with $\fn{force}$-tags it
should be possible. This could give us the best of two worlds: minimal
translation of pattern matching and universal applicability of the
approximation lemma and fixed point induction.

\subsection{Lemmas from Finite Theorems}

\begin{comment}

REWRITE - Fin predicate...

Fin(take n xs) <=> Fin(n) \/ Fin(xs)
Fin(xs ++ ys) <=> Fin(xs) /\ Fin(ys)p

Implementing addition of lemmas for properties that always hold, i.e,
for also partial and infinite values is straightforward; you can just
add the universally quantified property to the theory. The problem is
when it only holds for finite or total values: you will need FOL
predicates or functions to describe exactly when the property
holds. Research has been made how to use such predicates
\cite{sortMonotonicity}, \cite{polyMonotonicity}, but it is not
decided yet on which way to encode types. Furthermore, you will need
to prove that functions return finite or total values on such
input. For instance, append for lists gives a total list if both input
lists are total, but on the other hand, if the result list is total
but potentially infinite, we can only draw the conclusion that the
first list is total, since that might be the infinite one. So, to
prove properties that does not always hold you will need to prove a
lot more properties about your functions.

\end{comment}

\subsection{Speculating Lemmas}

The previous sections assumed that the necessary lemmas are present as
properties in the source file. This puts a lot of burden on the
programmer to both figure out required lemmas and to state them, it is
not always clear which lemmas are required to prove a given
property. This section discusses some ideas to generate candidate
lemmas.

\note{Discuss this with Moa and get references}

In the Isabelle induction and rippling community, there are concepts
such as critics, which makes lemmas and generalizations when the term
rewriting heuristic rippling fails, and lemma calculation, which tries
to syntactically analyze the problem before proving in order to figure
out suitable lemmas.

Another approach is to do property speculation, implemented for
Isabell \cite{isacosy} and Haskell \cite{quickspec}, where you simply
tell the QuickSpec program all functions in your program, and it tries
to find equality properties by creating small syntax trees of the
functions. It would be very interesting to see how useful such
properties are as lemmas.



%\section{Material Implication}
%
%Zeno has support for properties with implications,

%\section{Proving non termination}
%
%Since we will need to prove termination for functions, how would one
%prove non-termination (for some inputs)? Then we could get more
%completeness: this function with these inputs would be $\bot$.

\section{Type classes}
\label{sec:typeclasses}

There are some obstacles to support type classes. One is introduced by
the technicalities of type inference to decide instances in the
program. Another is how to express type classes in logic. An approach
would be to use dictionary passing, inlined for concrete types.

A third obstacle is to decide which axioms to use for the functions
from a type class. Let us take Monoids as an example. Its signature
consists of a binary operator and an element from the type. The
required laws are associativity of the operator and that the element
is the identity element for this operator. Lists are an instance of
Monoid, and the associativity law holds for finite and partial lists,
but the right identity does not hold: \hs{$\bot$ ++ []} is not equal
to \hs{[]}, as it is equal to $\bot$. This example suggests that it
could be appropriate to assume the laws only for total values, and
that it would be sensible to automatically check these laws for all
instances.

We know something else about the functions from a type class: they are
all continuous. This is also true for functions which are quantified
over. This is currently not enforced in the generated theories, and an
open question is how this effects the results.

\section{Other Techniques}

Each of the proof methods have their own future work chapter, but it
would be nice to have some finite fixed point induction on terminating
functions and a finite approximation lemma. You can also do as with
the approximation lemma which became just a fixpoint induction over a
structural identity function, but instead of putting one of each side
of the equality, you can do it on the individual variables or sub
trees of an expression: this would be like structural induction
encoded as approximation lemma.

Another technique is recursion induction \cite{recind}, where you
prove that two functions are equal by asserting that one of them
fulfills the same equations as the other.

\section{Faster Proof Searches via Predicates}

It is possible to add annotations in the equations generated for
functions to make the theorem prover not unroll unnecessary
definitions and regard these equalities more like definitions, so they
do not rewrite from the wrong direction. One way is to add a
min-predicate (named so for historical reasons), that can also
sometimes lead to finite models.

