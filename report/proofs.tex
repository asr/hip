\chapter{Proof Techniques}
\label{ch:proofs}

The proof developed in this thesis is called \hs{hip}, the Haskell
Inductive Prover. To use it, properties are inserted to the source
code where the definitions of the relevant functions are. As an
example, this is how the associativity of list concatenation can be
entered:

\begin{code}
import Prelude ()
import AutoPrelude

(++) :: [a] -> [a] -> [a]
[]   ++ ys = ys
x:xs ++ ys = x:(xs ++ ys)

prop_app_assoc :: [a] -> [a] -> [a] -> Prop [a]
prop_app_assoc xs ys zs = xs ++ (ys ++ zs) =:= (xs ++ ys) ++ zs
\end{code}

\noindent
There is no module system implemented, so all definitions must be
present in one source file. The usual imports from \ts{Prelude} are
hidden. The arguments are universally quantified, so this property
means:

\newcommand\append{+\!+}
\begin{equation*}
  \faaa{xs}{ys}{zs} xs \append (ys \append zs) = (xs \append ys) \append zs
\end{equation*}

\noindent
The equality is interpreted as equality on the constructor level: two
values are identified if they are constructed with the same
constructor and their arguments are also equal.  The function \hs{=:=}
and the type constructor \hs{Prop} come from the imported module
\ts{AutoPrelude}.  The type argument to \hs{Prop} is the type of the
equality, so the type of \hs{=:=} is:

\begin{code}
(=:=) :: a -> a -> Prop a
\end{code}

The type signature of properties cannot be omitted as this is used for
some proof techniques. Using \hs{hip} is then a matter of saving the
file, for instance to \hs{ListProps.hs}, and executing this statement
in your favourite terminal:

\begin{code}
hip ListProps.hs
\end{code}

The program will report to you which proof methods succeeded on proving
this property, if any. In this case, it is provable with all three
inductive techniques.

By importing \hs{AutoPrelude} the properties in the file are also
testable with QuickCheck, given that there are appropriate \hs{Eq} and
\hs{Arbitrary} instances provided.

The rest of this chapter explains the different proof methods
supported in this tool. Some properties are a direct consequence from
the definitions in your file. How to prove such properties is
described in Section \ref{sec:equality} about definitional
equality. The three other techniques uses induction in different
ways. Structural Induction is explained in Section
\ref{sec:induction}, which uses the structure of the data types a
property quantifies over. Another method which does induction on the
recursive structure of the program is Fixed Point Induction,
introduced in Section \ref{sec:fixpoint}. A more subtle way of
induction is used in the Approximation Lemma, Section
\ref{sec:approx}, where the structure of the data type of the equality
is approximated.

% Definitional Equality -------------------------------------------------------

\section{Definitional Equality}
\label{sec:equality}

Some properties cannot or need not use induction or some more
sophisticated technique, since they are true by definition. Examples
are properties for fully polymorphic functions such as the definition
of \hs{id} in the SK-calculus, here

\begin{code}
s f g x = f x (g x)
k x y = x
id x = x

prop_skk_id :: Prop (a -> a)
prop_skk_id = s k k =:= id
\end{code}

Then, the generated conjecture is simply

\begin{equation*}
\app{ (\app {\ptr{s}} {\ptr{k}} )
    }{\ptr{k}} = \ptr{id}
\end{equation*}

Another example where this is useful is to prove functor and monad
laws for the environment monad.

\subsection{Extensional Equality and seq}

To prove the previous property we also need to have extensional
equality, postulated with the following axiom

\begin{equation*}
\faa{f}{g} (\fa{x} \app{f}{x} = \app{g}{x}) \rightarrow f = g
\end{equation*}

\noindent
which identifies function pointers and functions composed with $@$.
One problem with extensional equality in Haskell, is that the presence
of \hs{seq} breaks it. \hs{seq} is a built in function with the
following behavior:

\begin{code}[mathescape]
seq :: a -> b -> b
seq $\bot$ y = $\bot$
seq x y = y$, \qquad x \neq
\end{code}

It forces the first argument to weak head normal form and returns the
second. For our purposes, it is only important if the first argument
is $\bot$, then the function also returns this as it is strict in its
first argument. With \hs{seq} it is possibly to distinguish between
these two functions, which otherwise are observationally equal:

\begin{code}[mathescape]
f = $\bot$
g = \x -> $\bot$
\end{code}

Because \hs{seq f ()} evaluates to $\bot$, and \hs{seq g ()} evaluates
to \hs{()}, but on any argument \hs{f} and \hs{g} gets, they both
return $\bot$. Here we also need an extra axiom, which says that
anything applied to $\bot$ is $\bot$:

\begin{equation*}
\fa{x} \app{\bot}{x} = \bot
\end{equation*}

However, \hs{seq} is the only function that can tell such functions
apart, so we will ignore its presence in Haskell.  In the future,
there could be added as a switch \hs{--enable-seq}, which weakens
extensional equality appropriately.

Furthermore, if extensional equality is assumed we also have
the property that \hs{Prop (a -> b)} is equivalent to
\hs{a -> Prop b}, by letting the property have an extra argument that
is applied to the left and right hand side of the equality. This has
two benefits. Firstly, it can trigger other proof methods should \hs{a}
or \hs{b} be concrete types (the former for induction and the latter
for approximation lemma). Secondly, it does not need to use the
extensionality axiom introduced above which introduces extra steps in
the proof search.

\subsection{Future Work: Concrete Concerns}

This only works on non-concrete types because of the way bottoms are
added. One example when this is a problem with is this plausible
definition of boolean or, and the property of right identity:

\begin{code}
False || a = a
True  || _ = True

prop_or_right_identity :: Bool -> Prop Bool
prop_or_right_identity x = x || False =:= x
\end{code}

The translation of \hs{||} makes any element in the domain that is not
the introduced constants $\fn{false}$ or $\fn{true}$ for \hs{Bool}'s
constructors, equal $\bot$. Now consider the translation of the
property:

\begin{equation*}
\fa{x} x \, \fn{||} \, \fn{false} = x
\end{equation*}

Now this is false: take a model with another element $\star$ in the
domain:

$$\star \, \fn{||} \, \fn{false} = \bot$$

The consequence of this is that the proof principle of definitional
equality is only used on abstract types, rather than polymorphic, as
they cannot be strict. Do not fear: the property above is trivially
proved by induction. For the \hs{Bool} type and other non recursive
data types, induction degenerate into mere case enumeration.




% Structual Induction ---------------------------------------------------------

\section{Structural induction}

\subsection{Induction}

Some background of induction and how it is present in well-known
theories like PA, ZFC, MLTT and CoC.

PA which only concerns natural numbers has a small vocabulary
consisting only of the constant $0$, the unary successor function $s$,
and binary plus and multiplication.
Here the induction schema from looks like this:

\note{One could also be explicit about the free variables in $P$}
\begin{mathpar}
  \inferrule*
     {
       \overbrace{P(0)}^{\mathrm{base}}
       \\
       \overbrace{
           \fa{x}
                 \underbrace{P(x)}_{\mathrm{hypothesis}}
              \rightarrow
                 \underbrace{P(s(x))}_{\mathrm{conclusion}}
       }^{\mathrm{step}}
     }
     { \fa{x} P(x) }
\end{mathpar}

This is a axioms schema since it is not possible to quantify over the
predicate $P$ in FOL. Rather, it is a infinite set of axioms, one for
each (well-formed) formula instantiated in place for $P$. Generally,
ATPs do not instantiate schemas themselves but it has to be done
manually, with an appropriate formula for $P$.

Any non-recursive, or more importantly recursive data type gives rise
to induction schemata.

\footnote{Haskell's natural numbers are of course also cluttered with
  elements that are not natural numbers, such as $\bot$, but also the
  infinite ``natural number'' defined by \hs{infinite = Succ infinite}}
, defined the usual way in Haskell by \hs{data Nat = Zero | Succ Nat}
yields this induction axiom schema:

\begin{mathpar}
  \inferrule*
     {
       P(\fn{Zero})
       \\
       \fa{x} P(x) \rightarrow P(\fn{Succ}(x))
     }
     { \fa{x} P(x) }
\end{mathpar}


% Fixpoint Induction ----------------------------------------------------------

\section{Fixed point induction}
\label{sec:fixpoint}

Induction is applicable when arguments are of a concrete type, such as
lists or natural numbers. There are also properties where all
arguments are of abstract types. The canonical example is the
map-iterate property:

\begin{equation*}
\faa{(f : a \rightarrow a)}{(x : a)} \hs{map} \w f \w (\hs{iterate} \w f \w x) \eq
           \hs{iterate} \w f \w (f \w x)
\end{equation*}

Here $f$ is an abstract function $a \rightarrow a$, and $x$ is
something of type $a$. This example is further investigated in Section
\ref{sec:mapiter} below, but it is already clear that we cannot proceed to
prove this with structural induction.

Enter fixed point induction. It allows a way of performing induction
on the recursive structure of the program. In short, if the property
regards a function $f$, the hypothesis is that the property holds for all
the recursive calls in the definiton of $f$, and the goal is to prove
that it holds for $f$.

\begin{comment}
It
is an early example of a technique from domain theory, attributed to
Scott and de Bakker,
\note{Citation needed: there is a book called
  Mathematical Theory of Program Correctness by Jaco de Bakker that
  could be appropriate if found}
and sometimes called Scott induction
or computational induction.  \cite{domains}
\end{comment}

The least fixed point for a function can be found in Haskell with the
function \hs{fix}, which can simply be defined as:

\begin{code}
fix :: (a -> a) -> a
fix f = f (fix f)
\end{code}

This function solves the equation $x = f \w x$, since substituting $x$
for $\hs{fix} \w x$: the left side evaluates to $f \w (\hs{fix} \w f)$
in one step, which is then equal to the right side. This is the origin
of the name of the combinator \hs{fix}: this is a fixed point of the
equation.  Any self-recursive function can be rewritten in terms of
\hs{fix}. Recall the \hs{Prelude} function \hs{repeat}, which makes an
infinite list of the same element \hs{repeat x = x : repeat}. Its
transformation to \hs{fix}-style is this:

\begin{code}
repeat x = fix r
  where r i = x : i
\end{code}

Computing \hs{repeat x}, we get the following unfolds:
\begin{equation*}
  \hs{repeat x}
= \hs{fix r}
= \hs{x:fix r}
= \hs{x:x:fix r}
= \hs{x:x:x:fix r}
  \cdots
\end{equation*}
So \hs{fix (x:)} is the infinite list of \hs{x}. The translation of a
self-recursive function to be defined in terms of \hs{fix} is
mechanical. Assume $f$ is defined with arguments $\overline{x}$ and
has a body $e$ that uses both itself and its arguments, let us write
this as $e(f,\overline{x})$. Then the translation is this:

\begin{equation*}
f \w \overline{x} \eq e(f,\overline{x})
\w \Leftrightarrow \w
f \eq \hs{fix} \w (\lambda \w f' \w \overline{x} \w \rightarrow \w e(f',\overline{x}))
\end{equation*}

Another more domain theoretic approach is to say that
$\hs{fix} \w f \eq \lub{n}(f^n \bot)$, where $f^n \bot$ is $n$ applications of $f$:
\begin{equation*}
f^n \bot \eq \underbrace{f (f (\cdots (f}_{n \w \mathrm{copies \w of} \w f}} \bot) \cdots))
\end{equation*}
This corresponds to a potentially infinite, countable unrolling of $f$.
It is easy to verify that $\langle f^n \bot\rangle_{n\in\omega}$ is a
$\sqsubseteq$-chain by induction on $n$, and that this is the least
pre-fixed point of $f$ is also showed by induction: Assume there
is another pre-fixed point $\theta$, thus satisfying
$\theta \eq f \w \theta$. The base case is
$\bot \eq f^0 \bot \sqsubseteq \theta$, trivially satisified since
$\bot$ is the least element. For the step case, assume that
$f^n \bot \sqsubseteq \theta$, and we get the conclusion
$f^{n+1} \bot = f (f^n \bot) \sqsubseteq f \w \theta = \theta$ as desired.
Fixpoint induction proves properties about a function written in terms
of \hs{fix}, and its inference rule is this:

\begin{mathpar}
  \inferrule*
     {
       P(\bot)
       \\
       P(x) \rightarrow P(f \w x)
       \\
       P \w \mathrm{admissible}
     }
     { P(\fn{fix} f) }
\end{mathpar}


Here it is important that $P$ is \emph{admissible} , meaning that for
all $\sqsubseteq$-chains of length $\omega$, if the property holds for
all elements in the chain it must necessary hold for its limit, futher
described in Section \ref{sec:admissible}, where a proof is given that
the properties we are interested in, universally quantified equalities
of continuous functions, are admissible predicates.

An interesting property of fixed point induction is that it does not
care about types: indeed, it works in an untyped setting. In addition,
it can exploit strange recursive structures of the function. A caveat
is that it can only prove properties that must hold for infinite and
partial values.

The proof that fixed point induction relies on the fact that
$\lub{n}(f^n \w \bot) \eq \hs{fix} \, f$, where $f^n$ is $n$
self-applications of $f$. This is true since \hs{fix} is defined as
$f$ self-applied to it self. Apart from this, the proof only uses
induction over natural numbers and that $f^0 \w \bot \eq \bot$, and
it is of course important that $P$ is admissible. See proof below:

\begin{align*}
P(\bot) & \wedge \fa{x} P(x) \rightarrow P(f x) \\
\desclra{$f^0 \w \bot \eq \bot$} \\
P(f^0 \w \bot) & \wedge \fa{x} P(x) \rightarrow P(f x) \\
\descra{quantifying} \\
P(f^0 \w \bot) & \wedge \fa{n} P(f^n \w \bot) \rightarrow P(f^{n+1} \w \bot) \\
\desclra{induction} \\
\fa{n} & P(f^n \w \bot) \\
\desclra{\textit{P} \w admissible} \\
& P(\lub{n}(f^n \w \bot)) \\
\desclra{definition \w of \w \hs{fix}} \\
& P(\hs{fix} \w f) \\
\end{align*}

One reason to introduce fixed point induction is to avoid the natural
numbers in $\fa{n} P(f^n \bot)$  to prove $P(\hs{fix} \w f)$.

\subsection{Example: map-iterate}
\label{sec:mapiter}

For properties that do not have any arguments with a concrete type,
structural induction is not applicable. The Haskell function
\hs{iterate} is a that makes an infinite list from a seed, by repeated
application of a function, i.e \hs{iterate f x} is the list
 \hs{x:f x:f (f x):}$\cdots$. It is related to Haskell function
 \hs{map} in the map-iterate property, stated as follows:

\begin{equation*}
\faa{f}{x} \hs{map} \w f \w (\hs{iterate} \w f \w x) \eq
           \hs{iterate} \w f \w (f \w x)
\end{equation*}

\noindent
With their standard definitions given in \ref{code:mapiterate} below.

\begin{figure}[h!]
\centering
\begin{minipage}[b]{6cm}
\begin{code}
map :: (a -> b) -> [a] -> [b]
map f (x:xs) = f x : map f xs
map f [] = []
\end{code}
\end{minipage}
\hspace{10pt}
\begin{minipage}[b]{6cm}
\begin{code}[mathescape]
iterate :: (a -> a) -> a -> [a]
iterate f x = x : iterate f (f x)
$\newline$
\end{code}
\end{minipage}
\caption{Definition of \texttt{map} and \texttt{iterate}
\label{code:mapiterate}
}
\end{figure}

The behavior of \hs{map} is to apply a function to every element of a
list. We see that we cannot use structural induction here, since both
$f$ and $x$ are abstract, but this can be proved by fixpoint induction
on \hs{iterate}. First, we rewrite this function in terms of \hs{fix}:

\begin{code}
iterate = fix iter
iter i f x = x : i f (f x)
\end{code}

The predicate $P$ from fixpoint induction is $P(g) \w \Leftrightarrow
\w \faa{f}{x} \hs{map} \w f \w (g \w f \w x) \eq g \w f \w (f \w x) $. If we
prove the base case and step case we can then conclude
$P(\hs{fix iter})$, and that is by definition $P(\hs{iterate})$.

The base case is $P(\bot)$. Since \hs{map} is strict in its second
argument, it is both the left side and right side evaluate to $\bot$.
The for the step case we have to show
$P(\hs{i}) \rightarrow P(\hs{iter i})$. We start from the induction
hypothesis and work towards the goal as follows:

\begin{align*}
\w \faa{f}{x} \hs{map} \w f \w (\hs{i} \w f \w x) & \eq \hs{i} \w f \w (f \w x) \\
\descra{generalizing $x$ to $f \w x$} \\
\w \faa{f}{x} \hs{map} \w f \w (\hs{i} \w f \w (f \w x)) & \eq \hs{i} \w f \w (f \w (f \w x)) \\
\descra{substitutivity} \\
\w \faa{f}{x} f \w x \hs{:} \hs{map} \w f \w (\hs{i} \w f \w (f \w x)) & \eq f \w x \hs{:} \hs{i} \w f \w (f \w (f \w x)) \\
\desclra{\defof{\texttt{map}}} \\
\w \faa{f}{x} \hs{map} \w f \w (x \hs{:} \hs{i} \w f \w (f \w x)) & \eq f \w x \hs{:} \hs{i} \w f \w (f \w (f \w x)) \\
\desclra{\defof{\texttt{iter}}} \\
\w \faa{f}{x} \hs{map} \w f \w (\hs{iter} \w \hs{i} \w f \w x) & \eq \hs{iter} \w \hs{i} \w f \w (f \w x) \\
\end{align*}

As discussed earlier, the $P$ used is admissible since it is an
universally quantified equality. Hence, fixpoint induction gives us the
\hs{map}-\hs{iterate} property.


\begin{comment}
To illustrate why it is important that the property $P$ is admissible,
we shall consider an example
consider the predicate P to be “is not infinite”, and then you can
prove for a lot of functions that they return finite objects. For
instance, define this function:

CHANGE THIS

  Instead do the finite list predicate, and use $\neq
  \hs{False}$. This then servers for an example why inequality
  predicates are not admissible!

\begin{code}
listrec :: ([a] -> [a]) -> [a] -> [a]
listrec i [] = []
listrec i (x:xs) = x : i xs
\end{code}

Then define
$P(f) \Leftrightarrow \fa{x} ``f(x) \w \mathrm{is \w not \w infinite}"$,
and proceed to prove $P(\hs{fix listrec})$ by fixed point induction. The
base case $P(\bot)$ succeeds, since $\bot$ is not infinite, and if we
assume $P(\hs{i})$, we have no problem proving $P(\hs{listrec i})$.
Hence $P(\hs{fix listrec})$, and since \hs{fix listrec} is essentially
a linear identity function on lists, we have ``proved'' that all lists
are finite (but possibly partial).

The error is as promised that $P$ is not admissible: for the sequence
\begin{equation*}
\bot \sqsubseteq
\hs{0:}\bot \sqsubseteq
\hs{0:1:}\bot \sqsubseteq
\hs{0:1:2:}\bot \sqsubseteq
\cdots
\end{equation*}
$P$ holds for all elements but $P$ does not hold for its limit \hs{[0..]}.

\end{comment}

\begin{comment}
\subsection{Mutually Recursive Functions}

You can also mechanically transform mutually recursive functions to be
defined in terms of \hs{fix}. The functions \hs{even} and \hs{odd}
defined below, which determines if a \hs{Nat} is even, and odd,
respectively, are straightforwardly written by mutual recursion:

\begin{code}
even :: Nat -> Bool           odd :: Nat -> Bool
even Z     = True             odd Z     = False
even (S x) = odd x            odd (S x) = even x
\end{code}

To write these functions in terms of fix, as an additional argument,
the take a tuple of ``non-recursive'' copies of themselves.

\begin{code}
evenToFix :: (Nat -> Bool,Nat -> Bool) -> Nat -> Bool
evenToFix (evenUnFix,oddUnFix) Z     = True
evenToFix (evenUnFix,oddUnFix) (S x) = oddUnFix x

oddToFix :: (Nat -> Bool,Nat -> Bool) -> Nat -> Bool
oddToFix (evenUnFix,oddUnFix) Z     = True
oddToFix (evenUnFix,oddUnFix) (S x) = evenUnFix x
\end{code}

Here the prefix \hs{ToFix} means that it is a function subject to be
\hs{fix}-ed, and \hs{UnFix} means that it is the ``non-recursive''
function. The functions above can now be \hs{fix}-ed by giving the
tuple as an argument to both of them:

\begin{code}
even',odd' :: Nat -> Bool
(even',odd') = fix (\t -> (evenToFix t,oddToFix t))
\end{code}

This encoding makes \hs{even'} denotationally equal to \hs{even} and
the same relation hols for \hs{odd'} and \hs{odd}.
\end{comment}

\subsection{Simplification}

The mechanical translations introduced above for self-recursive
functions and mutually recursive functions makes a new function with
an additional argument, the ``non-recursive'' version of itself. By
the translation to FOL that is used, this would introduce a new
argument as a ``function pointer'' and introduce uses of $\appfn$,
which gives unnecessary overhead to the automated theorem
provers.

This is another approach. It avoids introducing these function
pointers and the additional argument to every function. Given a
function $f$ with arguments $\overline{x}$ defined as this:

\begin{equation*}
f \, \overline{x} = e(\overline{x},f)
\end{equation*}

Two new constants are introduced, $\tofix{f}$ and $\unfix{f}$
and this definition:

\begin{equation*}
\tofix{f} \, \overline{x} = e(\overline{x},\unfix{f})
\end{equation*}

\noindent
The empty circle $\unfix{}$ describes that this function is empty
(lacks implementation,) and the filled circle $\tofix{}$ means that
this function has an implementation.

Now we can get a simplified fixpoint schema:

\begin{mathpar}
  \inferrule*
     {
       P(\bot)
       \\
       P(\unfix{f}) \rightarrow P(\tofix{f})
       \\
       P \, \mathrm{admissible}
     }
     { P(f) }
\end{mathpar}

\noindent
The empty $\unfix{f}$ does not have any implementation. But it has
something much better, namely the induction hypothesis. The induction
conclusion is to prove the property for $\tofix{f}$, where the
recursive call to $f$ is replaced with $\unfix{f}$. We do this
simplification since it is better suited for the theorem provers.

\newpage
This also works for several functions at the same time, possibly
mutually recursive:

\begin{mathpar}
  \inferrule*
     {
       P(\bot,\bot)
       \\
       P(\unfix{f},\unfix{g}) \rightarrow P(\tofix{f},\tofix{g})
       \\
       P \, \mathrm{admissible}
     }
     { P(f,g) }
\end{mathpar}

This translation needs to be carried out with some care, since for $f
\, \overline{x} = e(\overline{x},f)$, it is also possible that $f$ is
called in bodies of other functions. These are of two kinds: either
this function is also called from $f$, making it recursive, or another
function which is not called from $f$, but makes use of $f$
anyway. The first example, with a recursive call, the body needs to be
edited so $f$ becomes translated (to $\bot$, $\unfix{f}$ or
$\tofix{f}$), and the second case should use the original $f$. The
transitive clousure of the call graph is calculated, and every
appropriate calls of $f$ are replaced.

\subsection{Erroneous Use of Fixed Point Induction}

The importance that the predicate $P$ is admissible is illustrated in
this example. The function \hs{finite} below returns \hs{True} on
finite lists and $\bot$ on partial lists.

\begin{code}
finite :: [a] -> Bool
finite []     = True
finite (x:xs) = finite xs
\end{code}

Is it possible to use \hs{finite} to prove that \emph{all} lists in
Haskell are either finite or partial? Let the predicate be
$P(f) \Leftrightarrow \fa{xs} f \w xs \neq \hs{False}$, and proceed by fixed
point induction. This is the proof plan with the predicate inlined:

\begin{mathpar}
  \inferrule*
     {
       \fa{xs} \bot \w xs \neq \hs{False}
       \\
       \fa{xs}             \unfix{\hs{finite}} \w xs \neq \hs{False}
               \rightarrow \tofix{\hs{finite}} \w xs \neq \hs{False}
     }
     { \fa{xs} \hs{finite} \w xs \neq \hs{False} }
\end{mathpar}

\noindent
The base case is trivial, as anything applied to bottom is $\bot$ and
$\bot \neq \hs{False}$. The step case also succeeds; trivially if $xs$
is the empty list, by the hypothesis if $xs$ is non-empty, and if $xs$
is bottom for similar reasons as in the base case.

The error is as promised that $P$ is not admissible. See the
$\sqsubseteq$-chain below:
\begin{equation*}
\bot \sqsubseteq
\hs{0:}\bot \sqsubseteq
\hs{0:1:}\bot \sqsubseteq
\hs{0:1:2:}\bot \sqsubseteq
\cdots
\end{equation*}
$P$ holds for all elements in the chain, but $P$ does not hold for its
limit \hs{[0..]}. This example also serves as a proof that inequality
in general is not admissible.


\subsection{Candidate Selection}

Faced with the following property saying that if you drop $n$ elements
from a list the length of this is the same as the length of the
original list minus $n$, which functions should one do fixed point
induction on?

\begin{verbatim}
prop_length_drop :: [a] -> Nat -> Prop Nat
prop_length_drop xs n = length (drop n xs) =:= length xs - n
\end{verbatim}

The answer here is to do fixed point induction on \hs{drop}, and on
\hs{-}. So far no better way to tackle this is used than to try fixed
point induction on all subsets of recursive functions mentioned in the
property.

\subsection{Future Work}

Just as with structural induction, it is also possible to use fixed
point in more than one ``depth'', giving for instance this inference
rule:

\begin{mathpar}
  \inferrule*
     {
       P(\bot)
       \\
       P(f \w \bot)
       \\
       P(x) \wedge P(f \w x) \rightarrow P(f \w (f \w x))
       \\
       P \w \mathrm{admissible}
     }
     { P(\fn{fix} f) }
\end{mathpar}

It is also possible to use such an encoding as in ``Automated depth''
in Section \ref{sec:futind} to let the theorem prover determine the
depth. As an example, the map-iterate property impossible to show with
\hs{map} redefined to \hs{doublemap}, defined below, with ordinary one
depth fixed point induction.

\begin{verbatim}
doublemap :: (a -> b) -> [a] -> [b]
doublemap f []       = []
doublemap f [x]      = [f x]
doublemap f (x:y:zs) = f x : f y : doublemap f zs
\end{verbatim}

\noindent
While \hs{doublemap} is behaviorally equivalent to \hs{map} on total
lists, it makes the induction hypothesis in fixed point induction too
weak.


An issue with the candidate selection is that is some selections are
quite stupid, for instance doing fixating functions on only one side
of the equality. A heuristic to find good candidates would be beneficial.

% Approximation Lemma ---------------------------------------------------------

\section{Approximation Lemma}

The approximation lemma must be considered a standard technique for
proving properties about corecursive programs. Just like fixed point
induction it can be used with functions that are produce infinite
values, like \hs{repeat} and \hs{iterate}, out of abstract values
making structural induction impossible \cite{corecursive}.  The
approximataion lemma supersedes the classical take lemma
\cite{introfp} by being easier to apply and generalize: unlike the
take lemma, it can be applied to equalities of any polynomial
data type \cite{genapprox}. The definitions of \hs{take} and
\hs{approx}: % for lists can be viewed in figure \ref{fig:takeapprox}.

%% This is weird!

%\floatstyle{ruled}
%\newfloat{program}{thp}{lop}
%\floatname{program}{Program}

\note{how to put/float source code side by side?}
%\begin{program}
\begin{verbatim}
take :: Nat -> [a] -> [a]             approx :: Nat -> [a] -> [a]
take Zero    _      = []              approx (Suc n) []     = []
take (Suc n) []     = []              approx (Suc n) (x:xs) = x : approx n xs
take (Suc n) (x:xs) = x : take n xs
\end{verbatim}
%\caption{Definition of \hs{take} and \hs{approx} on lists, with Peano-\hs{Nat}s.}
%\label{fig:takeapprox}
%\end{program}
\note{write this as Haskell code? (those \hs{Nat} are actually $\mathbb{N}$)}

Whereas \hs{take} approximates a list and ends it with \hs{[]},
\hs{approx} ends it with $\bot$ since the \hs{Zero} case is
omitted. The idea of these techniques is then to show that show that
two lists are equal by showing that their prefix or approximation
coincides for all natural numbers. The approximation lemma is given thusly

\begin{equation}
\label{eq:approxeq}
xs \, = \, ys \quad \Leftrightarrow \quad \fa{n \in \mathbb{N}} \hs{approx} \, n \, xs = \hs{approx} \, n \, ys
\end{equation}

Equation \ref{eq:approxeq} quantifies over the \emph{true} natural
numbers, rather than the \emph{polluted} Haskell naturals. Showing an
equality then amounts to a proof by induction over natural numbers,
and the base case for $0$ is always true by reflexivity, as the
approximation is $\bot$.
The right to left implication is (trivially) true by substitution, and
the other direction hinges on the lemma that better and better
approximations form a chain with limit \hs{id}, as illustrated in
Equation \ref{eq:approxchain}.

\begin{equation}
\label{eq:approxchain}
\hs{approx} \, 0 \,
   \sqsubseteq \,
\hs{approx} \, 1 \,
   \sqsubseteq \,
\cdots
   \sqsubseteq \,
\hs{approx} \, n \,
   \sqsubseteq \,
\hs{approx} \, (\hs{Suc} \, n) \,
   \sqsubseteq \,
\cdots
   \sqsubseteq \,
\hs{id}
\end{equation}

The inclusions in \ref{eq:approxchain} are easily given by induction
on natural numbers and the limit by structural induction on lists.
For other polynomial data types, this lemma is established by
the structural induction induced on that data type.
The desired implication is then readily deduced:

\newcommand{\xsys}[2]{#1 \, xs \, #2 & = #1 \, ys #2}
\newcommand{\desca}[1]{  & \hspace{44.5mm}                              \{ \mathrm{#1} \}}
\newcommand{\descra}[1]{ & \hspace{35mm} \Rightarrow     \hspace{4mm} \{ \mathrm{#1} \}}
\newcommand{\descla}[1]{ & \hspace{35mm} \Leftarrow      \hspace{4mm} \{ \mathrm{#1} \}}
\newcommand{\desclra}[1]{& \hspace{35mm} \Leftrightarrow \hspace{4mm} \{ \mathrm{#1} \}}
\begin{align*}
\xsys{\fa{n} \hs{approx} \, n}{}            \\
\descra{limits}                             \\
\xsys{\lub{n} \, (\hs{approx} \, n}{)}      \\
\desclra{continuity \, of \, application}   \\
\xsys{\lub{n} \, (\hs{approx} \, n)}{}      \\
\desclra{Equation \, \ref{eq:approxchain}} \\
\xsys{\hs{id}}{}                            \\
\desclra{definition \, of \, \hs{id}}       \\
\xsys{}{}                                   \\
\end{align*}

\subsection{Example: Mirroring an Expression}

Consider these definitions of a modest but prototypical expression
data type, and its mirroring function:

\begin{verbatim}
data Expr = Add Expr Expr | Value Nat

mirror :: Expr -> Expr
mirror (Add e1 e2) = Add (mirror e2) (mirror e1)
mirror (Value n)   = Value n

prop_mirror_involutive :: Expr -> Prop Expr
prop_mirror_involutive e = e =:= mirror (mirror e)
\end{verbatim}

Two things to notice here is that the type \hs{Expr} does not have a
nullary constructor. Then the \hs{take} lemma would not be usable as
there is no such function over these expressions: the list version
returns the empty list \hs{[]} for the zero case, but there is no such
alternative for \hs{Expr} above. It is important that the limit of
approximations is the identity, and we cannot get this property when
trying to generalise the take lemma.

Furthermore, fixed point induction fails on this property: choosing
either or both occurrences of \hs{mirror} on the right side is
constant bottom for the base case, and the left side is the identity.


We shall now proceed to prove that \hs{mirror} is involutive by the
approximation lemma. The approximation function for \hs{Expr} is
automatically generated, by approximating each self-recursive
constructor, and hence \hs{Value}'s \hs{Nat} is not further approximated:

\begin{verbatim}
approx :: Nat -> Expr -> Expr
approx (Suc n) (Add e1 e2) = Add (approx n e1) (approx n e2)
approx (Suc n) (Value n)   = Value n
\end{verbatim}

\note{Use $\bot$ or bottom in this text?}
Indeed, we also get a third case for $\bot$ which states that the
approximation of $\bot$ is, quite unsurprisingly, $\bot$.
As always in proofs by approximation lemma, we proceed by induction
over natural numbers, and the base case is always trivial: true by
reflexivity as both sides are $\bot$. The step case - which indeed is
the only proof obligation in any proof of this kind - is to prove
this:

\begin{equation*}
\fa{e}  approx \, (\hs{Suc} \, n) \, e = approx \, (\hs{Suc} \, n) \, (mirror \, (mirror \, e))
\end{equation*}

An important property of the induction hypothesis is the universal
quantification of the expression $e$, unlike the fixed natural number
$n$:

\begin{equation*}
\fa{e}  approx \, n \, e = approx \, n \, (mirror \, (mirror \, e))
\end{equation*}

The proof is by case exhaustion. The case for \hs{Value} and $\bot$
are trivial: \hs{mirror} is strict in its first argument, and
mirroring \hs{Value} twice is the identity, so these cases are both
true by reflexivity. The \hs{Add} case is ever so slightly more
elaborate, and with names shortened to \hs{app} and \hs{mir} the
reasoning is as follows:

\note{$(\hs{Suc} \, n)$ or $(n + 1)$?}
\newcommand{\Adds}[2]{\hs{Add} \, #1 e_1 #2 \, #1 e_2 #2}
\newcommand{\Approxn}[0]{\hs{app} \, n \,}
\newcommand{\ApproxSucn}[0]{\hs{app} \, (\hs{Suc} \, n) \,}
\newcommand{\mirmir}[0]{\hs{mir} \, (\hs{mir} \, }
\begin{align*}
\faa{e_1}{e_2}&  \ApproxSucn (\Adds{}{})  = \ApproxSucn (\mirmir \Adds{}{} ))                                                                   \\
                                                                                 \desclra{\defof{\hs{mirror}}}                                   \\
\faa{e_1}{e_2}&  \ApproxSucn (\Adds{}{})  = \ApproxSucn (\Adds{(\mirmir}{))})                                                                    \\
                                                                                \desclra{\defof{\hs{approx}}}                                    \\
\faa{e_1}{e_2}&  \Adds{(\Approxn}{)}      = \Adds{(\Approxn(\mirmir}{)))}                                                                        \\
                                                                                \desclra{induction \, hypothesis \, twice \, (e_1 \, and \, e_2)} \\
\faa{e_1}{e_2}&  \Adds{(\Approxn}{)}      = \Adds{(\Approxn}{)}                                                                                  \\
                                                                                \desca{reflexivity}                                              \\
\end{align*}

\subsection{Approximation Lemma is Fixpoint Induction}

This technique is already simple and widely applicable, however it can
further be simplified. Implementing it in this form relies on the
auxiliary structure of Peano natural numbers which also needs to be
added to the theory. This can be removed by the observation that is
\note{What is a better name for these functions than \hs{id}?}
expressed as fixed point induction over a recursive form of \hs{id}.

\begin{verbatim}
id :: [a] -> [a]              id :: Expr -> Expr
id []     = []                id (Add e1 e2) = Add (id e1) (id e2)
id (x:xs) = x : id xs         id (Value n)   = Value n
\end{verbatim}

Each \hs{id} function constructed in this way is indeed an identity
function, equivalent to the implementation \hs{id x = x} if we
disregard time and space complexity. Now, to prove

\begin{equation*}
e_1 \, = \, e_2
\end{equation*}

we simply use fixed point induction to prove

\begin{equation*}
\hs{id} \, e_1 \, = \, \hs{id} \, e_2
\end{equation*}

where \hs{id} is a such a specialized recursive identity function over
the data type of the equality. With the same translation of recursive
functions as in the fixed point section, the axioms for \hs{id} for, say
lists becomes:

\begin{align*}
            & \tofix{id}(\hs{[]}) \,   = \, \hs{[]}                                                           \\
\faa{x}{xs} & \tofix{id}(x\hs{:}xs) \, = x\hs{:}\unfix{id}(xs)                                                \\
\fa{xs}     & \tofix{id}(xs)           = \bot  \leftarrow    xs \neq [] \wedge xs \neq head(xs)\hs{:}tail(xs) \\
\end{align*}

The step case in induction $P(\unfix{id}) \rightarrow P(\tofix{id})$
is then exactly the same strength as the approximation lemma with
natural numbers.

\subsection{Implementation} The implementation of approximation lemma
was the simplest to implement, after definitional equality. First,
from the type signature from the property, such a recursive identity
function as above is generated for the data type of the equality. Then
the lemma $P(\unfix{id})$ is added to the theory, with $P$
instantiated to the universally quantified equality, and the
conjecture is $P(\tofix{id})$. The base case need not be proven,
$P(\bot)$ is always true since it evaluates to $\bot=\bot$.

\subsection{Future Work: Total Approximation Lemma}

It is also possible to adjust the approximation lemma to prove
properties that are true for total and potentially infinite objects,
but false for objects with partial values. One such property is the
idempotence of \hs{nub}. Here is a version of \hs{nub} on booleans,
and the said property about it:

\begin{verbatim}
nub :: [Bool] -> [Bool]
nub (True :True :xs) = nub (True:xs)
nub (False:False:xs) = nub (False:xs)
nub (x:xs)           = x:nub xs
nub _                = []

prop_nub_idem :: [Bool] -> Prop [Bool]
prop_nub_idem xs = nub (nub xs) =:= nub xs
\end{verbatim}

Consider the list \hs{True:False:}$\bot$. One application of \hs{nub}
gives \hs{True:}$\bot$, and two gives $\bot$ immediately. In spite of
this, the property is a truism for finite as well as infinite lists,
provided that there are not bottoms.

The way to enable the approximation lemma to prove such properties is
to add predicates of totality, and add axioms like the following to
the theory:

\note{How to make $Total$ look like one word? The kerning for $T$ in
  $Total$ looks incorrect}
\begin{align*}
            & \neg \, Total(\bot) \\
            & Total(\hs{[]}) \\
\faa{x}{xs} & Total(x) \wedge Total(xs) \rightarrow Total(x\hs{:}xs) \\
\fa{xs}     & Total(xs) \rightarrow Total(\hs{nub} \, xs) \\
\end{align*}

To use fixed point induction \hs{Total} needs to be an admissible
predicate. But this is indeed so, for each model with infinite lists,
we have that all such lists are total.  However, that a total argument
to \hs{nub} yields a total value also needs to be proven. Furthermore,
the totality property for the cons case of lists is debatable: should
the consed element also be total?

The proof obligation is then:

\begin{mathpar}
  \inferrule*
     {
        \fa{xs} Total(xs) \Rightarrow \unfix{id}(\hs{nub} (\hs{nub} \, xs)) = \unfix{id}(\hs{nub} \, xs)
     }
     {
        \fa{xs} Total(xs) \Rightarrow \tofix{id}(\hs{nub} (\hs{nub} \, xs)) = \tofix{id}(\hs{nub} \, xs)
     }
\end{mathpar}

It seems like you have to add
$Total(x) \rightarrow Total(\tofix{id}(x))$ and
$Total(x) \rightarrow Total(\unfix{id}(x))$ to the theory, too.

\subsection{Uncategorized / old}

After fixed point induction since it is an easy consequence of fixed
point induction, and how we removed the auxiliary structure of natural
numbers. This makes it equivalent to fixed point induction of id on
both sides (or does it?)

Approximation lemma is a generalization of the take lemma, and its
first form is used for properties about infinite and partial lists,
but it is easily generalized to other recursive data types.
In particular, all polynomial data types — for example, any
sum-of-products data type — can be defined in this way.
This result generalizes
to mutually recursive, parameterised, exponential and nested datatypes, but
for simplicity we only consider polynomial datatypes in this article.
\cite{genapprox}

How to use approximation lemma on exponential, mutually recursive and
nested data types? This would be nice.
