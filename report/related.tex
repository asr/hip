\section{Related Work}

Zeno \citep{zeno} is probably the most similar work to this and
proving equational properties for Haskell programs. Zeno supports type
classes, and implications with equational antecedents, and can
generalise properties, it makes no attempt to tackle infinite and
partial values, a goal of our work.

%Indeed, the authors even ask themselves if it is not really a ML theorem prover.

%\note{Discuss this with Moa and get references}
Zeno translates its proofs to Isabelle, a framework for different
logics. Its most prominent use is as a theorem prover for Higher Order
Logic, commonly abbreviated Isabelle/HOL \citep{isabelbook}. Different
automated proving techniques has been implemented, most notably the
heuristic Rippling \citep{dixonphd}, found in IsaPlanner
\citep{isaplanner}, which also relies on critics
\citep{productiveuse}. Isabelle/HOL can also call theorem provers with
Sledgehammer \citep{hammering}, and a comparison of different theorem
provers in Sledgehammer has been done by \cite{judgement}.

The proof assistant and programming language Coq resembles the
function package in Isabelle in the sense that each data type
declaration generates an induction principle and the user can write
own induction procedures should it be too weak. Proofs can be assisted
by tactics to automatically resolve some goals, and has an own
language Ltac to write such tactics. A tactic for Rippling has been
implemented for Coq \citep{inductioncoq}.

Coq has some support for corecursion, and so has the dependently
language Agda \citep{norell:thesis}. Its strong type system based on
Martin-Löf type theory \citep{mltt} with inductive families makes it a
proof assistant thanks to the Curry-Howard correspondence. The
support for corecursion in Agda has been investigated by \cite{productivity}.

In the work by \cite{agdaatp} Agda is used as a fronted to ATPs
enabling classical reasoning about functional programs.  Reconstruction
of proofs from the ATP Waldmeister to dependently typed code is
investigated by \cite{dtatp}. The source language is Mella,
syntactically similar to Agda but based on the Calculus of
Constructions. A similar approach was taken by \cite{agsy} does a
search akin to Prolog, and can also handle induction. A contemporary
implementation of this is the Agsy proof search assistant for Agda.

% Effort has been put into automating the proof generation in Agda,
%notably by \cite{agdaatp}, which is quite alike this work in its
%translation from Agda to first order logic. One must however give
%quite much information to the theorem prover, like how to instantiate
%the induction hypotheses. %\note{Peter Dybjer maybe said that it could be seen as a very nice way to interact with a theorem prover, thanks to Agda.}

These automated approaches to Agda do not handle infinite
values. Techniques used in this thesis about corecursive functional
programs are given for Haskell in \cite{corecursive}, and for Isabelle
\cite{mechanizingcoinduction}, and in general by
\cite{coindtutorial}. New approaches to stream processing can be
found in the work by \cite{streams}.

A radically different way to reason about Haskell programs than this
thesis does is investigated by \cite{fastandloose}. The idea is to do
reasoning in a language assumed to only be total, and under many
circumstances the proofs also holds for partial values. Another
approach based on operational semantics is given by
\cite{totalcorrectness}. The proofs from this technique resembles
those from fixed point induction, and the same techniques can also be
used to derive locally improved functions.
