\documentclass[serif,professionalfont]{beamer}
\def\maindocument{} % To tell tikz images that they are not stand alone
\usepackage{tikz}
\usetikzlibrary{shapes,arrows,calc}
\usepackage[utf8]{inputenc}
\usepackage[T1]{fontenc}
\usepackage{fixltx2e}
\usepackage{graphicx}
\usepackage{longtable}
\usepackage{float}
\usepackage{wrapfig}
\usepackage{soul}
\usepackage{textcomp}
\usepackage{marvosym}
\usepackage{wasysym}
\usepackage{latexsym}
\usepackage{amssymb}
\usepackage{hyperref}
\usepackage{mathpartir}
\usepackage{color}
\tolerance=1000
\usepackage{palatino,inconsolata,amsmath,array}
\providecommand{\alert}[1]{\textbf{#1}}

\definecolor{Purpleee}{RGB}{140,20,140}
\setbeamercolor{title}{fg=Purpleee}
\setbeamercolor{frametitle}{fg=Purpleee}
\setbeamercolor{structure}{fg=Purpleee}

\usepackage{listings}

\lstnewenvironment{code}[1][]%
  {
   \noindent
   \minipage{\linewidth}
   \vspace{0.2\baselineskip}
%   \vspace{-0.4\baselineskip}
   \lstset{basicstyle=\ttfamily,
%           frame=single,
           language=Haskell,
           keywordstyle=\color{black},
           #1}}
  {%\vspace{-0.8\baselineskip}
   \endminipage}

\title{HipSpec}
\author{Dan Rosén}
\date{\today}

\begin{document}

\maketitle


\makeatletter
\newcommand*{\rom}[1]{\text{\footnotesize\expandafter\@slowromancap\romannumeral #1@.}}
\newcommand*{\romnodot}[1]{\text{\footnotesize\expandafter\@slowromancap\romannumeral #1@}}
\makeatother

%\newcommand\note[1]{\mbox{}\marginpar{\footnotesize\raggedright\hspace{0pt}\emph{#1}}}
%\newcommand\note[1]{}
\newcommand\PA{\mathcal{P\!A}}
\newcommand\hs[1]{\verb~#1~}
\newcommand\ts[1]{\verb~#1~}
\newcommand\fn[1]{\mathrm{#1}}
%\newcommand\fn[1]{#1}
\newcommand\ptr[1]{\fn{\operatorname{#1-ptr}}}
\newcommand\appfn{@}
\newcommand\app[2]{#1 \, \appfn \, #2}
\newcommand\appp[2]{(#1) \, \appfn \, #2}
\newcommand\ex[1]{\exists \, #1 \, . \,}
\newcommand\nexxx[3]{\nexists \, #1 , #2 , #3 . \,}
\newcommand\fa[1]{} % \forall \, #1 . \,}
\newcommand\faa[2]{} % \forall \, #1 , #2 . \,}
\newcommand\faaa[3]{} % \forall \, #1 , #2 , #3 . \,}
\newcommand\faaaaaa[6]{} % \forall \, #1 , #2 , #3 , #4 , #5 , #6 . \,}

\newcommand{\HRule}{\rule{\linewidth}{0.5mm}}%\usetikzlibrary {\trees,positioning,arrows}

\newcommand\fixb[0]{ ^{\bullet}}
\newcommand\fixw[0]{ ^{\circ}}

\newcommand\tofix[1]{#1^{\bullet}}
\newcommand\unfix[1]{#1^{\circ}}

\newcommand\append[0]{\texttt{\small{++}}}

\newcommand{\xsys}[2]{#1 \, xs \, #2 & = #1 \, ys #2}
\newcommand{\desca}[1]{  & \hspace{44.5mm}                            \{ \text{#1} \}}
\newcommand{\descra}[1]{ & \hspace{35mm} \Rightarrow     \hspace{4mm} \{ \text{#1} \}}
\newcommand{\descla}[1]{ & \hspace{35mm} \Leftarrow      \hspace{4mm} \{ \text{#1} \}}
\newcommand{\desclra}[1]{& \hspace{35mm} \Leftrightarrow \hspace{4mm} \{ \text{#1} \}}

\newcommand\lub[1]{\sqcup_{#1}}

\newcommand\defof[1]{definition of #1}

\newcommand\w[0]{\,\,}
\newcommand\eq[0]{ = }

\newcommand{\defBNF}[4] {\text{#1}\quad&#2&::=&\;#3&\text{#4}}
\newcommand{\defaltBNF}[2] {&&|&\;#1&\text{#2}}

\newcommand{\hstup}[2]{\hs{(} #1 \hs{,} #2 \hs{)}}

\newcommand{\nsqsubseteq}{\,\,\, /\!\!\!\!\!\!\sqsubseteq}

\newcommand{\bindname}{>\!\!>\!\!=}
\newcommand{\bind}[2]{#1 \bindname #2}
\newcommand{\bindp}[3]{\fn{bind'}(#1,#2,#3)}
\newcommand{\fork}[2]{\fn{fork}(#1,#2)}
\newcommand{\forkr}[1]{\fn{right}(#1)}
\newcommand{\forkl}[1]{\fn{left}(#1)}
\newcommand{\leaf}[1]{\fn{leaf}(#1)}
\newcommand{\unleaf}[1]{\fn{unleaf}(#1)}
\newcommand{\return}[1]{\fn{return}(#1)}

\newcommand{\bindb}[2]{#1 \tofix{\bindname} #2}
\newcommand{\bindw}[2]{#1 \unfix{\bindname} #2}


\newcommand\Inf{\fn{Inf}}
\newcommand\Total{\fn{Total}}
\newcommand\Fin{\fn{Fin}}

\begin{frame}
  \frametitle{HipSpec is based on...}

  \begin{center}
  {\Large Hip: the Haskell Inductive Prover}

  \vspace{2\baselineskip}

  Master's Thesis

  \vspace{2\baselineskip}

  Supervisor: Koen Claessen

  \end{center}
\end{frame}

\begin{frame}
  \frametitle{HipSpec}

  \begin{center}
  {\Large HipSpec}

  \vspace{2\baselineskip}

  Our new tool

  Combines Hip with QuickSpec to speculate lemmas

  \vspace{2\baselineskip}

  Koen Claessen, Moa Johansson,

  Nick Smallbone, \textbf{Dan Ros\'en}

  \end{center}
\end{frame}

\begin{frame}[fragile]
\frametitle{HipSpec overview}

% A little hack so this can be compiled stand alone
\ifx\maindocument\undefined
    \documentclass{article}
    \usepackage[latin1]{inputenc}
    \usepackage{tikz}
    \usetikzlibrary{shapes,arrows,calc}

    \begin{document}
    \pagestyle{empty}
    \def\donext{\end{document}}
\else
    \def\donext{}
\fi

% Define block styles
\tikzstyle{bigblock}   = [rectangle, draw=Purpleee, thick,
                          text width=4.75em, text centered,
                          minimum height=10em,
                          rounded corners=.8ex]

\tikzstyle{smallblock} = [rectangle,
                          text width=6em, text centered,
                          minimum height=3em,thick,
                          draw=Purpleee,rounded corners=.8ex]

\tikzstyle{arr} = [->,>=stealth',semithick]

\makebox[\textwidth][c]{
\begin{tikzpicture}

    % Place nodes
    \node at (-0.75,0)     [bigblock] (src)  {Haskell Source};
    \node at (4,1.5)  [smallblock] (conj) {Conjectures};
    \node at (4,-1.5) [smallblock] (thy)  {First-Order Theory};
    \node at (8.5,0)     [smallblock] (atp)  {Theorem Prover};
   % Draw edges

    \node at (4.5,0) (text) {Induction (Hip)};

    \draw [arr] (src)  to [bend left=15] node [above] {QuickSpec} (conj);
    \draw [arr] (src)  to [bend left=-15] node [below,text centered,text width=5em] {Translation (Hip)} (thy);

    \draw [arr] (conj) to [bend right=17]  (atp);
    \draw [arr] (thy)  to [bend right=-17] (atp);

    \draw [arr] (atp)  to [bend left=-15]
        node [right,near start] {\texttt{ Timeout}}
        node [above right,very near end] {Open conjecture} (conj);
    \draw [arr] (atp)  to [bend left=15]
        node [right,near start] {\texttt{ Theorem}}
        node [below right,very near end] {Extend theory} (thy);

\end{tikzpicture}
}


\donext

\end{frame}

\begin{frame}
  \frametitle{Evaluation}

  Two test suites

  \begin{itemize}

    \item \textbf{A}: From
      \emph{Productive Use of Failure in Inductive Proof}
      by Bundy and Ireland (1995)

    \item \textbf{B}: From
      \emph{Case-Analysis for Rippling and Inductive Proof}
      by Johansson, Dixon and Bundy (2010)

  \end{itemize}

\end{frame}

\begin{frame}[fragile]
  \frametitle{Results}

  \textbf{A}: 50 theorems, 38 equational. HipSpec proves 36.

  \vspace{\baselineskip}

  Unproved:
  \begin{center}
  \begin{tabular}{>{\small}c >{\small}l}
  No  & Conjecture \\
  \hline
  T14 & \verb!ordered (isort xs) = True! \\
  T50 & \verb!count x (isort xs) = count x xs! \\
  \end{tabular}
  \end{center}

  \pause

  \textbf{B}: 85 theorems, 71 equational. HipSpec proves 67.

  \vspace{\baselineskip}

  Unproved:
  \begin{center}
  \begin{tabular}{>{\small}c >{\small}l}
   No & Conjecture \\
   \hline
   53 & \verb!count n xs                  = count n (sort xs)! \\
   66 & \verb!len (filter p xs) <= len xs = True! \\
   68 & \verb!len (delete n xs) <= len xs = True! \\
   78 & \verb!sorted (sort xs)            = True! \\
  \end{tabular}
  \end{center}

\end{frame}

\begin{frame}[fragile]
  \frametitle{Results, Test Suite B continued}

  \begin{center}
  \begin{tabular}{l l}
    Tool       & Proved conjectures (of 85) \\
    \hline
    Zeno       & 82 \\[3pt]
    ACL2s      & 74 \\[3pt]
    IsaPlanner & 47 \\[3pt]
    Dafny      & 45 \\[3pt]
    HipSpec    & 67 (of 71)
  \end{tabular}
  \end{center}

\end{frame}

\begin{frame}[fragile]
  \frametitle{Test Suite B, continued}
  Two properties only proved by HipSpec!
  \begin{center}
  \begin{tabular}{>{\small}c >{\small}l}
   No & Conjecture \\
   \hline
  72 & \verb!rev (drop i xs) = take (len xs - i) (rev xs)! \\
  74 & \verb!rev (take i xs) = drop (len xs - i) (rev xs)! \\
  \end{tabular}
  \end{center}
\end{frame}

\begin{frame}[fragile]
  \frametitle{Proof: \Large\texttt{rev (drop i xs) = take (len xs-i) (rev xs)}}
  \begin{center}
  \begin{tabular}{>{\small}c >{\small}l}
   No & Conjecture \\
   \hline
  1 & \verb!len (drop x xs) = len xs-x!               \\[-1pt]
  2 & \verb!len xs          = len (rev xs)!           \\[-1pt]
  3 & \verb!xs              = take x xs++drop x xs!   \\[-1pt]
  4 & \verb!rev (ys++xs)    = rev xs++rev ys!         \\[-1pt]
  5 & \verb!xs              = take (len xs) (xs++ys)!
  \end{tabular}
  \end{center}

  \begin{align*}
  & \hspace{-2em} \small \hs{rev (drop x xs)}                                                   & \small = \left\{ 5 \right\} \\[-3pt]
  & \hspace{-2em} \small \hs{take (len (rev (drop x xs))) (rev (drop x xs)++rev (take x xs))}   & \small = \left\{ 2 \right\} \\[-3pt]
  & \hspace{-2em} \small \hs{take (len (drop x xs)) (rev (drop x xs)++rev (take x xs))}         & \small = \left\{ 1 \right\} \\[-3pt]
  & \hspace{-2em} \small \hs{take (len xs-x) (rev (drop x xs)++rev (take x xs))}                & \small = \left\{ 4 \right\} \\[-3pt]
  & \hspace{-2em} \small \hs{take (len xs-x) (rev (take x xs++drop x xs))}                      & \small = \left\{ 3 \right\} \\[-3pt]
  & \hspace{-2em} \small \hs{take (len xs-x) (rev xs)}
  \end{align*}
\end{frame}

\begin{frame}
  \frametitle{Current Limitations/Future Work}

  \begin{itemize}

    \item Conditional properties \pause

    \item Irrelevant lemmas \pause

    \item Number of functions in the module \pause

    \item Expensive computations \pause

    \item Number of variables \pause

    \item Non-termination \pause

    \item Feedback on Failure

  \end{itemize}

\end{frame}










\begin{frame}
  \frametitle{Behind the Scenes}
%  \begin{itemize}
%    \item Translation to first order logic\\
%    \item Trying different induction techniques\\
%    \item Running automated theorem provers\\
%  \end{itemize}
\end{frame}

%\begin{frame}
%\frametitle{Overview}
%
%Add an overview picture of the project. Highlight the parts that
%
%\begin{itemize}
%\item The programmer supplies (source + properties)
%\item What my tool does (translation, instantiating, displaying results)
%\item ATPs job
%\end{itemize}
%
%It is possible that this picture could make the previous slide superfluous
%
%\end{frame}


\begin{frame}[fragile]
\frametitle{Translation of bind}
\label{sec-2}

\begin{code}
Fork u v >>= f = Fork (u >>= f) (v >>= f)
Leaf x   >>= f = f x
\end{code}

\begin{align*}
\onslide<2->{\rom{1} &&& \bind{\fork{u}{v}}{f} = \fork{\bind{u}{f}}{\bind{v}{f}}} \\
\onslide<3->{\rom{2} &&& \bind{\leaf{x}}{f}    = \app{f}{x}} \\
\onslide<5->{\rom{3} &&& t \neq \fork{\forkl{t}}{\forkr{t}} \wedge t \neq\leaf{\unleaf{t}} \rightarrow \bind{t}{f} = \bot}
\end{align*}

\begin{align*}
\onslide<4->{\rom{4} &&& \leaf{x}    \neq \fork{u}{v}} & \onslide<7->{\rom{7} &&& \forkl{\fork{u}{v}} = u}\\
\onslide<6->{\rom{5} &&& \leaf{x}    \neq \bot       } & \onslide<7->{\rom{8} &&& \forkr{\fork{u}{v}} = v}\\
\onslide<6->{\rom{6} &&& \fork{u}{v} \neq \bot       } & \onslide<7->{\rom{9} &&& \unleaf{\leaf{x}}   = x}
\end{align*}
\end{frame}

%% Describe this better

\begin{frame}[fragile]
\frametitle{Function pointers}
\begin{center}
$\rom{2} \quad \faa{x}{f} \bind{\leaf{x}}{f} = \app{f}{x}$

\vspace{2\baselineskip}
\end{center}
For each function in the theory, make a new constant
representing the function with an axiom:
\begin{center}
\begin{align*}
\app{\ptr{return}}{x}         & = \fn{return}(x) \\
\appp{\app{\ptr{bind}}{t}}{f} & = \bind{t}{f}
\end{align*}

\end{center}
\end{frame}

\begin{frame}[fragile]
  \frametitle{Limitations in the Haskell translation}
  \begin{itemize}
    \item Difficult
      \begin{itemize}
        \item Type classes \\
        \item Built-in types (\hs{Integer}, \hs{Int}, \hs{Char} $\ldots$)
      \end{itemize}
    \item Sugar etc
      \begin{itemize}
        \item list comprehensions \\
        \item \hs{do}-notation \\
        \item no \hs{seq}
      \end{itemize}
  \end{itemize}
\end{frame}

%% Remove domino slide
%
%\begin{frame}
%  \frametitle{Induction}
%  \begin{center}
%    \includegraphics[height=6cm]{dominofalling} \\
%  \end{center}
%\end{frame}

\begin{frame}[fragile]
\frametitle{Supported Techniques}
\label{sec-4}
\begin{itemize}

\item Plain Equality \\
\item Structural Induction \\
\item Fixed Point Induction \\
\item Approximation Lemma

%\item Structural Induction\\
%\begin{code}
%prop_return_right t = t >>= return =:= t
%\end{code}
%
%\pause
%
%\item Approximation Lemma\\
%\begin{code}
%prop_return_left f x = return x >>= f =:= f x
%\end{code}
%
%\pause
%
%\item Fixed Point Induction\\
%\begin{code}
%prop_assoc t f g = (t >>= f) >>= g =:=
%                   t >>= (\x -> f x >>= g)
%\end{code}

\end{itemize} % ends low level
\end{frame}

\begin{frame}[fragile]
\frametitle{Plain Equality Example}

\begin{code}
return x >>= f =:= f x
\end{code}

\begin{align*}
lhs = & \; \bind{\return{x}}{f}        \\
    = & \; \bind{\leaf{x}}{f}          \\
    = & \; f @ x                       \\
    = & \; rhs
\end{align*}


\end{frame}

\begin{frame}[fragile]
\frametitle{Structural Induction Example}

\begin{mathpar}
  \inferrule*
    {
      P(\leaf{x})
      \and
      P(u)\wedge P(v)\rightarrow P(\fork{u}{v})
    }
    { P(t) }
\end{mathpar}

\pause

\begin{code}
t >>= return =:= t
\end{code}

We define $P$ as:

$$P(t) \Leftrightarrow \bind{t}{\ptr{return}} = t$$

\end{frame}

\begin{frame}[fragile]
\frametitle{Structural Induction Example}

\begin{code}
t >>= return =:= t
\end{code}

\vspace{2\baselineskip}

Case $\leaf{x}$:

\begin{align*}
lhs = & \; \bind{\leaf{x}}{\ptr{return}} \\
    = & \; \app{\ptr{return}}{x}         \\
    = & \; \return{x}                    \\
    = & \; \leaf{x}                      \\
    = & \; rhs
\end{align*}

\end{frame}

\begin{frame}[fragile]
\frametitle{Structural Induction Example}

\begin{code}
t >>= return =:= t
\end{code}

\vspace{2\baselineskip}

Case $\fork{u}{v}$:

\vspace{1\baselineskip}

Hypotheses: $\bind{u}{\ptr{return}} = u$ and $\bind{v}{\ptr{return}} = v$

\begin{align*}
lhs = & \; \bind{\fork{u}{v}}{\ptr{return}}                      \\
    = & \; \fork{\bind{u}{\ptr{return}}}{\bind{v}{\ptr{return}}} \\
    = & \; \fork{u}{v}                                           \\
    = & \; rhs
\end{align*}

\end{frame}

\begin{frame}[fragile]
\frametitle{Structural Induction for Infinite Values}

\begin{code}
t >>= return =:= t
\end{code}

\vspace{1\baselineskip}

But, if we also show the $\bot$ case...

$$\bind{\bot}{\ptr{return}} = \bot$$

\vspace{1\baselineskip}

By a theorem from domain theory then it holds for \emph{infinite trees}:

\begin{mathpar}
  \inferrule*
    {
      P(\bot)
      \and
      P(\leaf{x})
      \and
      P(u)\!\wedge\!P(v)\!\!\rightarrow\!\!P(\fork{u}{v})
      \and
      P \, \mathrm{admissible}
    }
    { P(t) }
\end{mathpar}

\end{frame}

\begin{frame}[fragile]
\frametitle{Fixed Point Induction}
\begin{code}
bind' b (Fork u v) f = Fork (b u f) (b v f)
bind' b (Leaf x)   f = f x
\end{code}

\hs{(>>=) = fix bind'}

\pause

\begin{mathpar}
  \inferrule*
    {
      P(\bot)
      \\
      P(x) \rightarrow P(f \w x)
      \\
      P \, \mathrm{admissible}
    }
    { P(\fn{fix} \w f) }
\end{mathpar}

\pause

$$\bindp{b}{\fork{u}{v}}{f} = \fork{\appp{\app{b}{u}}{f}}{\appp{\app{b}{v}}{f}}$$

\end{frame}

\begin{frame}[fragile]
\frametitle{Fixed Point Induction Redux}

\begin{code}[mathescape]
Fork u v >>=$\fixb$ f = Fork (u >>=$\fixw$ f) (v >>=$\fixw$ f)
Leaf x   >>=$\fixb$ f = f x
\end{code}

\begin{mathpar}
  \inferrule*
     {
       P(\bot)
       \\
       P(\unfix{\bindname}) \rightarrow P(\tofix{\bindname})
       \\
       P \, \mathrm{admissible}
     }
     { P(\bindname) }
\end{mathpar}

\pause

\begin{code}
(t >>= f) >>= g =:= t >>= (\x -> f x >>= g)
\end{code}

Define $P$:

$$P(h) :\Leftrightarrow \bind{h(t,f)}{g} = h(t,\lambda x . \bind{\app{f}{x}}{g})$$

\end{frame}

\newcommand\lamptr[2]{(\appp{\app{\ptr{lam}}{#1}}{#2})}
\newcommand\lam[3]{\fn{lam}(#1, \, #2, \, #3)}

\begin{frame}[fragile]
\frametitle{Bind Associativity with Fixed Point Induction}

\begin{code}
(t >>= f) >>= g =:= t >>= (\x -> f x >>= g)
\end{code}


To show:

$$\bind{(\bindb{t}{f})}{g} = \bindb{t}{\lamptr{f}{g}}$$

where

$$\lam{f}{g}{x} = \bind{\app{f}{x}}{g}$$

\pause

\vspace{2\baselineskip}

Cases: $t$ is one of $\bot$, $\leaf{x}$, $\fork{u}{v}$
\pause

\vspace{2\baselineskip}

$t \neq \fork{\forkl{t}}{\forkr{t}} \wedge t \neq\leaf{\unleaf{t}} \rightarrow \bindb{t}{f} = \bot$

\end{frame}

\begin{frame}[fragile]
\frametitle{Bind Associativity with Fixed Point Induction}

Case $\bot$. To show:
$$\bind{(\bindb{\bot}{f})}{g} = \bindb{\bot}{\lamptr{f}{g}}$$

\begin{align*}
lhs = & \; \bind{(\bindb{\bot}{f})}{g} \\
    = & \; \bind{\bot}{g}              \\
    = & \; \bot                        \\
    = & \; \bindb{\bot}{\lamptr{f}{g}} \\
    = & \; rhs
\end{align*}

\end{frame}

\begin{frame}[fragile]
\frametitle{Bind Associativity with Fixed Point Induction}

Case $\leaf{x}$. To show:
$$\bind{(\bindb{\leaf{x}}{f})}{g} = \bindb{\leaf{x}}{\lamptr{f}{g}}$$

where
$$\lam{f}{g}{x} = \bind{\app{f}{x}}{g}$$

\begin{align*}
lhs = & \; \bind{(\bindb{\leaf{x}}{f})}{g} \\
    = & \; \bind{\app{f}{x}}{g}            \\
    = & \; \lam{f}{g}{x}              \\
    = & \; \app{\lamptr{f}{g}}{x}          \\
    = & \; \bindb{\leaf{x}}{\lamptr{f}{g}} \\
    = & \; rhs
\end{align*}

\end{frame}

\begin{frame}[fragile]
\frametitle{Bind Associativity with Fixed Point Induction}

Case $\fork{u}{v}$. To show:
$$\bind{(\bindb{\fork{u}{v}}{f})}{g} = \bindb{\fork{u}{v}}{\lamptr{f}{g}}$$

Induction hypothesis:
$$\bind{(\bindw{t}{f})}{g} = \bindw{t}{\lamptr{f}{g}}$$


\begin{align*}
lhs = & \; \bind{(\bindb{\fork{u}{v}}{f})}{g} \\
    = & \; \fork{\bind{(\bindw{u}{f})}{g}}{\bind{(\bindw{v}{f})}{g}}\\
    = & \; \fork{\bindw{u}{\lamptr{f}{g}}}{\bindw{v}{\lamptr{f}{g}}} \\
    = & \; \bindb{\fork{u}{v}}{\lamptr{f}{g}} \\
    = & \; rhs
\end{align*}

\end{frame}

%\begin{frame}[fragile]
%\frametitle{Approximation Lemma - iterate}
%
%\begin{code}
%iterate f x = x : iterate f (f x)
%\end{code}
%
%\begin{code}
%iterate (^2) 1 = 1 : 2 : 4 : 8 : 16 : ...
%\end{code}
%
%\end{frame}

\begin{frame}[fragile]
\frametitle{Bind Associativity with Fixed Point Induction}

We have shown
\begin{align*}
&\bind{(\bindw{t}{f})}{g} = \bindw{t}{\lamptr{f}{g}} \rightarrow \\
&\bind{(\bindb{t}{f})}{g} = \bindb{t}{\lamptr{f}{g}}
\end{align*}

Base case is trivial
$$\bind{\bot}{g} = \bot(\lamptr{f}{g})$$

Fixed Point Induction gives
$$\bind{(\bind{t}{f})}{g} = \bind{t}{\lamptr{f}{g}}$$

\begin{code}
(t >>= f) >>= g =:= t >>= (\x -> f x >>= g)
\end{code}

\end{frame}

\begin{frame}[fragile]
\frametitle{Approximation Lemma - map-iterate}

\pause

\begin{code}
map f (iterate f x) =:= iterate f (f x)
\end{code}

\begin{code}
approx :: Nat -> [a] -> [a]
approx (1 + n) []     = []
approx (1 + n) (x:xs) = x : approx n xs
\end{code}

\pause

To show $xs = ys$, use induction over $\mathbb{N}$ to prove

$$\fa{n}\fn{approx}(n,xs) = \fn{approx}(n,ys)$$

\pause

\begin{code}
ind :: [a] -> [a]
ind []     = []
ind (x:xs) = x : ind xs
\end{code}

Now we can use Fixed Point Induction!

\end{frame}

\newcommand\iter[2]{\fn{iterate}(#1,\,#2)}
\newcommand\map[2]{\fn{map}(#1,\,#2)}
\newcommand\appw[1]{\fn{\unfix{ind}}(#1)}
\newcommand\appb[1]{\fn{\tofix{ind}}(#1)}

\begin{frame}[fragile]
\frametitle{Approximation Lemma - map-iterate}

\small{
\begin{code}[mathescape]
ind$\fixb$ :: [a] -> [a]
ind$\fixb$ []     = []
ind$\fixb$ (x:xs) = x : ind$\fixw$ xs
\end{code}
}

$$\map{f}{\iter{f}{x}} = \iter{f}{\app{f}{x}}$$

\pause

Induction hypothesis:
$$\appw{\map{f}{\iter{f}{x}}} = \appw{\iter{f}{\app{f}{x}}}$$

\pause

\begin{align*}
lhs = & \; \appb{\map{f}{\iter{f}{x}}} \\
    = & \; \appb{\map{f}{x:\iter{f}{\app{f}{x}}}}\\
    = & \; \appb{\app{f}{x} : \map{f}{\iter{f}{\app{f}{x}}}}\\
    = & \; \app{f}{x} : \appw{\map{f}{\iter{f}{\app{f}{x}}}}\\
    = & \; \app{f}{x} : \appw{\iter{f}{\app{f}{(\app{f}{x})}}}\\
    = & \; \appb{\app{f}{x} : \iter{f}{\app{f}{(\app{f}{x})}}}\\
    = & \; \appb{\iter{f}{\app{f}{x}}} \\
    = & \; rhs
\end{align*}

\end{frame}

%% DRAW THIS

\begin{frame}[fragile]
\frametitle{Test Suite}
\begin{table}[p]
\centering
%\begin{tabular}{>{\small}l | >{\small}l | >{\small}p{6cm} }
Group              & File                   & Description \\
\hline
Fundamental        & $\ts{Bool                  }$ & Simple properties about and, or and not \\
                   & $\ts{Expr                  }$ & Expressions with mirror, size and eval \\
                   & $\ts{Functions             }$ & Function composition, currying \\
                   & $\ts{Integers              }$ & Integers as a disjoint union of Nats \\
                   & $\ts{Reverse               }$ & Reverse with and without accumulator \\
                   & $\ts{Queues                }$ & Queues with $O(1)$ pop and enqueue \\
Infinite values    & $\ts{Infinite              }$ & Infinite list and tree properties \\
                   & $\ts{Sequences             }$ & Infinite sequences \\
                   & $\ts{Streams               }$ & Streams from \citep{streams} \\
Miscellaneous      & $\ts{PatternMatching       }$ & \hs{||} and \hs{mirror} in different ways \\
                   & $\ts{Fix                   }$ & Even and odd defined using \hs{fix} \\
                   & $\ts{Tricky                }$ & Properties that hold for total infinite lists \\
Monads             & $\ts{MonadEnv              }$ & The environment monad\\
                   & $\ts{MonadMaybe            }$ & The maybe monad \\
                   & $\ts{MonadState            }$ & The state monad \\
Natural numbers    & $\ts{Nat                   }$ & Natural numbers from the Zeno test suite \\
                   & $\ts{Nat2ndArg             }$ & Induction on the second argument \\
                   & $\ts{NatAcc                }$ & Accumulator definitions\\
                   & $\ts{NatDouble             }$ & Depth two definitions \\
                   & $\ts{NatDoubleSlow         }$ & Depth two, recursion in one depth \\
                   & $\ts{NatStrict             }$ & Strict in both arguments \\
                   & $\ts{NatSwap               }$ & Arguments swapped in the recursive call \\
Other Work         & $\ts{IWC                   }$ & Inductive challenge problems\footnote{http://www.csc.liv.ac.uk/~lad/research/challenges} \\
                   & $\ts{ProductiveUseOfFailure}$ & From \citep{productiveuse} \\
                   & $\ts{ZenoLists             }$ & The list part of Zeno's test suite \\
                   & $\ts{Ordinals              }$ & Ordinals as defined by \cite{dixonphd} \\
Sorting            & $\ts{InsertionSort         }$ & Insertion sort \\
                   & $\ts{MergeSort             }$ & Merge sort from \hs{Data.List} \\
\end{tabular}

\end{table}
\end{frame}


%%% REWRITE

\begin{frame}
\frametitle{Results}
\label{sec-5}

540 properties in the test suite

\begin{itemize}
\item 214 Theorems
      \begin{itemize}
      \item Definitional Equality: 74
      \item Structural Induction:  124
      \item Approximation Lemma:   145
      \item Fixed Point Induction: 26
      \end{itemize}
\item 111 Finite Theorems
\end{itemize}

%\begin{tabular}{ >{\small}r@{/}>{\small}l | >{\small}r@{/}>{\small}l | >{\small}r@{/}>{\small}l | >{\small}r@{/}>{\small}l | >{\small}r@{/}>{\small}l || >{\small}r@{/}>{\small}l }
%\multicolumn{2}{>{\small}l|}{Theorem} & \multicolumn{2}{>{\small}l|}{plain} & \multicolumn{2}{>{\small}l|}{induction} & \multicolumn{2}{>{\small}l|}{approx} & \multicolumn{2}{>{\small}l||}{fixpoint} & \multicolumn{2}{>{\small}l}{Finite Thm.}  \\
%214&540 & 74&214 & 124&214 & 145&214 & 26&214 & 111&540 \\
%\end{tabular}
\end{frame}


\begin{frame}
\frametitle{Future Work: Lemma system - Totality}
\label{sec-6}


\centering
\hs{data Nat = Succ Nat | Zero}

\vspace{\baselineskip}

\begin{align*}
\rom{1} &&        & \neg \, \Total(\bot) \\
\rom{2} &&        & \Total(\fn{zero}) \\
\rom{3} && \fa{x} & \Total(x) \rightarrow \Total(\fn{succ}(x))
\end{align*}

\pause

\begin{equation*}
\faa{x}{y} \Total(x) \wedge \Total(y) \rightarrow x + y = y + x
\end{equation*}
\end{frame}
\begin{frame}
\frametitle{Future Work: Lemma system - Finiteness}
\label{sec-7}


\begin{center}
$\fa{xs} \Fin(xs) \rightarrow \fn{reverse}(\fn{reverse}(xs)) = xs$
\end{center}
\end{frame}
\begin{frame}
\frametitle{Future Work: Lemma system - Infiniteness}
\label{sec-8}


\begin{center}
$\fa{x} \Inf(x) \leftrightarrow (x = \fn{succ}(x))$
\end{center}

\vspace{1\baselineskip}

\pause

\begin{center}
$\Fin(x) \leftrightarrow \Total(x) \wedge \neg \Inf(x)$
\end{center}

\pause

\begin{align*}
\rom{1} && \faa{x}{y} & \Fin(x) \wedge \Fin(y)                & \leftrightarrow &&& \Fin(x + y) \\
\rom{2} && \faa{x}{y} & \Inf(x) \vee (\Fin(x) \wedge \Inf(y)) & \leftrightarrow &&& \Inf(x + y)
\end{align*}
\end{frame}

\begin{frame}[fragile]
\frametitle{Future Work: Lemma synthesis from QuickSpec}

\begin{verbatim}
1: return x := x:[]
2: x:xs := return x++xs
== equations ==
1: x:[] == return x
2: xs++[] == xs
3: []++xs == xs
4: reverse [] == []
5: (x:xs)++ys == x:(xs++ys)
6: (xs++ys)++zs == xs++(ys++zs)
7: reverse (return x) == return x
8: reverse (reverse xs) == xs
9: reverse xs++return x == reverse (x:xs)
10: reverse xs++reverse ys == reverse (ys++xs)
\end{verbatim}
\end{frame}

%% MOVE TO END
\begin{frame}
  \frametitle{Related Work}
  \begin{itemize}
    \item Zeno (Sonnex, Drossopoulou, Eisenbach) {\footnotesize(Imperial College)}

    \item Combining Interactive and Automatic Reasoning about Functional Programs \\
          (Bove, Dybjer, Sicard)
    \item Dependently Typed Programming based on Automated Theorem Proving \\
          (Armstrong, Foster, Struth) {\footnotesize(Univ. of Sheffield)}
  \end{itemize}
\end{frame}

\begin{frame}
\frametitle{Conclusion}

\begin{itemize}
\item FOL is expressible enough for various Haskell concepts; pattern
  matching, infinite data structures and higher order functions
\item Thanks to referential transparency
\item Possible to prove many properties without proving termination
\end{itemize}
\end{frame}


\begin{frame}
\frametitle{Old slides}
these are slides from the old presentation
\end{frame}


\begin{frame}
\frametitle{Future Work: GHC Core}
\label{sec-9}
\begin{itemize}

\item Mimic GHC Semantics\\
\label{sec-9-1}%
\item Type classes and desugaring for free\\
\label{sec-9-2}%
\end{itemize} % ends low level
\end{frame}

\begin{frame}
\frametitle{Name Contest!}
\label{sec-11}
\begin{itemize}
\item hip - Haskell Inductive Prover\\
\item auhapp - AUtomated HAskell Property Prover\\
\item gluppe - GLUppe Proves Properties Equationally\\
\item glapp - GLapp Automatically Proves Properties\\
\item herp - Haskell Equational Reasoning Prover\\
\end{itemize}
\end{frame}

\begin{frame}[fragile]
\frametitle{Dictionary passing}
\label{sec-12}


\begin{verbatim}
class Monoid a where
  mappend :: a -> a -> a
  mempty  :: a

mconcat :: Monoid a => [a] -> a
mconcat = foldr mappend mempty
\end{verbatim}

\pause

$\Rightarrow$

\begin{verbatim}
data Monoid a = Monoid { mappend :: a -> a -> a
                       , mempty  :: a }

mconcat :: Monoid a -> [a] -> a
mconcat m = foldr (mappend m) (mempty m)
\end{verbatim}
\end{frame}

\end{document}


%\begin{frame}
%\frametitle{HALT - HAskell to Logic Translator}
%target logic is untyped.
%
%handle constructors as functions, and use function pointers
%\end{frame}


%\begin{frame}
%  \frametitle{Motivation}
%  \begin{itemize}
%    \item Testing can only show presence of bugs, \\
%          proving shows absence of bugs
%    \item Theorem provers are good at proving
%    \item Infinite values from lazy data structures
%  \end{itemize}
%\end{frame}


%\begin{frame}
%  \frametitle{Induction step}
%  \begin{center}
%    \includegraphics[height=6cm]{dominostop}
%  \end{center}
%\end{frame}

%\begin{frame}[fragile]
%  \frametitle{Induction Example}
%  \begin{center}
%    \begin{verbatim}
%count :: Nat -> [Nat] -> Nat
%count n (x:xs) = if n == x then S (count n xs) else count n xs
%count n []    = Z
%
%prop_count :: Nat -> [Nat] -> [Nat] -> Prop Nat
%prop_count n xs ys
%  = count n xs + count n ys =:= count n (xs ++ ys)
%    \end{verbatim}
%  \end{center}
%\end{frame}
%
%\begin{frame}[fragile]
%  \frametitle{Infinite Lists}
%  \begin{verbatim}
%    natsFrom x = x : natsFrom (S x)
%  \end{verbatim}
%
%  \hs{natsFrom 0} is the infinite list \hs{0 : 1 : 2 : 3 : ...}
%  \\
%  How can we prove properties about infinite lists?
%\end{frame}
%
%\begin{frame}
%  \begin{center}
%    \includegraphics[height=8cm]{crashpdf}
%  \end{center}
%\end{frame}
%
%\begin{frame}[fragile]
%  \frametitle{Crashes}
%  \begin{center}



%  \Huge\Huge$$\bot$$
%  \end{center}
%  \begin{center}
%  ``bottom''
%  \begin{verbatim}
%         loop = loop
%         head []
%         undefined
%  \end{verbatim}
%  \end{center}
%\end{frame}

%\begin{frame}
%  \frametitle{Properties related to bottom}
%    \begin{itemize}
%      \item Monotonicity \\
%            $$x \sqsubseteq y \rightarrow f(x) \sqsubseteq f(y)$$ \\
%            $f$ cannot return more information from $\hs{1:}\bot$ than for $\hs{1:[]}$ \\
%\vspace{0.5cm}
%      \item Continuity \\
%            \hs{finite :: [a] -> Bool}
%  \end{itemize}
%\end{frame}

